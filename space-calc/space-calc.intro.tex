\section{Introduction}\label{sec:introduction} % (fold)

One of the notable things about the Einstein field equations
\cite{Misner1973} is their mutual recursion. The stress energy tensor
describing the distribution of matter and energy informs the metric
tensor, which in turn informs the stress energy tensor. Any red
blooded computer scientist sees this and immediately wants to express
it as a couple of mutually recursive functions and/or data structures.

Meanwhile, in computer science a revolution in the notion of location
has happened and all but gone unnoticed. On the one hand, Huet's
zipper \cite{DBLP:journals/jfp/Huet97} generalizes the intuitions of
the Dedekind cut for all differentiable data structures. Specifically,
for a data structure described by a differentiable functor, $T$, an
element of the data type $\partial \; T \times T$ describes a location
in an element of $T$ \cite{DBLP:conf/popl/McBride08}.

On the other, Milner's $\pi$-calculus transforms the
$\lambda$-calculus by reifying the site of interaction as a channel,
and introducing parallel composition \cite{milner91polyadicpi}. This
change simultaneously gives computations autonomy and a means of
finding each other. Meredith's rho-calculus takes Milner's insight a
step further \cite{DBLP:journals/entcs/MeredithR05}, making the
reified site of interaction into a fully first class entity with a
complexity of structure equivalent to computations.

A question worth exploring is whether these two notions of location
can be combined, yielding a site of interaction corresponding to a
location in a computation. The thought is that such a reconciliation
of the two ideas would allow for a computational calculus enjoying
something like the intuitions underlying the dynamics in Einstein's
field equations.

\subsection{Summary of contributions and outline of paper}

\subsubsection{Summary of contributions}
We present a calculus, dubbed the space calculus (from outer space!)
whose primitives are channels built from locations in computations
reified as data structures. This gives us the ability to model a
variety of space-like phenomena.

\subsubsection{Outline of paper}
TBD

% section introduction (end)
