\section{From calculus to efficient implementation}

It turns out that a variant of the $\mathsf{Linda}$ tuple space
\cite{DBLP:conf/parle/Gelernter89} where input is not blocking is the
critical innovation to an efficient implementation. Instead of
blocking we turn input from a key into the storage of a continuation
at the key. As the astute reader has already guessed, (the hashes of)
channels become keys.\\

%\begin{center}
\begin{figure}
  \centering
  \includegraphics[scale=0.5]{MeTTaCalcImpl1.pdf} \\
  \caption{Compiler from $\mathsf{MeTTa}$-calculus to $\mathsf{RSpace}$}
\end{figure}
%\end{center}

Thus, the diagram above indicates how to turn a given process state
into an $\mathsf{RSpace}$ instance.

\begin{itemize}
  \item the first equation says that the $\pzero$ process corresponds
    to the empty $\mathsf{RSpace}$;
  \item the second says that a comprehension corresponds to an
    $\mathsf{RSpace}$ consisting of a single key-value pair; the key
    being the hash of the channel on which the comprehension is
    listening (the \emph{source} of the comprehension); and the value
    being the (multiset containing the single) continuation formed by
    created a pattern-matching style lambda from \emph{target} of the
    comprehension to its \emph{body}; \footnote{Sometimes we abuse the
      terminology and call the body the continuation.}
  \item the third equation translates the concurrent composition of
    two processes, say $P$ and $Q$, into the combination of their two
    $\mathsf{RSpace}$s using an operation $\mathsf{RSpace}$s on we
    define below.
\end{itemize}

\subsubsection{Parallel composition of $\mathsf{RSpace}$s}

If we are combining two $\mathsf{RSpaces}$ that only have a single
key-value pair, each; and the keys are not equal, then we simply
combing them into a single $\mathsf{RSpace}$ containing both key-value
pairs. More generally, when combining $\mathsf{RSpace}$s that have no overlap
in their key-sets, we simply return the $\mathsf{RSpace}$ whose
key-value pairs is the union of the key-value pairs of the
$\mathsf{RSpace}$s being combined.\\
 
\begin{figure}
  \centering
  \includegraphics[scale=0.5]{MeTTaCalcImpl2.pdf}
  \caption{Compiler from $\mathsf{MeTTa}$-calculus to $\mathsf{RSpace}$ continued}
\end{figure}

When combining two $\mathsf{RSpaces}$ that only have a single
key-value pair, each, their keys are the same, but the patterns of
their continuations do not unify, then we simply combine them into a
single $\mathsf{RSpace}$ containing a single key-value pair, the key
of which is the key of each pair (remember, both pairs share a key)
and the value of which is the multiset containing the respective
values (continuations).

More generally, when combining $\mathsf{RSpace}$s that have overlap in
their key-sets, but the overlapping keys contain no continuations
whose patterns unify, we simply return the $\mathsf{RSpace}$ whose
key-value pairs is

\begin{itemize}
  \item the union of the key-value pairs of the $\mathsf{RSpace}$s
    being combined for the key-value pairs whose keys do not overlap;
  \item and constructs a new key-value pair for each pair in the
    overlap that shares a key, whose key is the key they share and
    whose value is the union of their respective multisets.
\end{itemize}

\begin{figure}
  \centering
  \includegraphics[scale=0.5]{MeTTaCalcImpl3.pdf}
  \caption{Compiler from $\mathsf{MeTTa}$-calculus to $\mathsf{RSpace}$ continued again}
\end{figure}

What remains is the case when combining two key-value pairs that have
unifying patterns in their multisets. The naive algorithm is quadratic
in the number of unification checks. However, $\mathsf{MORK}$ provides
a more efficient way to parallelize these checks.

[$\mathsf{MORK}$ algorithm description goes here.]

\subsubsection{Procedural reflection implementation}

The astute reader will have noticed that we have not addressed the
execution of procedurally reflective processes. The naive
implementation instantiates a \emph{copy} of the $\mathsf{RSpace}$ in
which $x\mathsf{?}P$ is being evaluated and allows one transactonal
step in the future of $P$, i.e. one $\mathsf{COMM}$ event in the
future of $P$. Then it makes that state available at $x$ in the
original $\mathsf{RSpace}$.

Unfortunately, there is no way to avoid the cost of copying the
original $\mathsf{RSpace}$. We should note that there is a natual way
to ship all processes into a different namespace.

\begin{mathpar}
  \inferrule* {}{u * P = P\{ u * x / x | x \in \mathsf{FN}(P) \}} \\
  \and
  \inferrule* {}{\mathsf{@}P * \mathsf{@}Q = \mathsf{@}(P\mathsf{|}Q)}
\end{mathpar}

Thus, all the key-value pairs could remain in the original
$\mathsf{RSpace}$ and merely be shifted into a distinguished
namespace. However, this would actually be more costly than a deep
copy of the original $\mathsf{RSpace}$ because the key-value pairs
still have to be copied and then shifted.

\subsection{The transactional semantics of $\mathsf{RSpace}$}

A given $\mathsf{RSpace}$ represents a collection of $\mathsf{MeTTa}$
spaces that share a transactional semantics. Effectively, the
collection of channels served by the $\mathsf{RSpace}$, that is the
\emph{namespace} served by a given $\mathsf{RSpace}$, all participate
in coordinated transactions in the same way that all the tables served
by a $\mathsf{SQL}$ server participate in coordinated
transactions. However, unlike $\mathsf{SQL}$'s architecture, the
$\mathsf{RSpace}$ architecture naturally composes, allowing for a
hierarchy of transactional coordination over a tree of namespaces served by a network of $\mathsf{RSpace}$s.
