\section{From formalism to physical intuition and back}
The guiding intuition is that information about a physical entity must
be present at a name. To model the non-deterministic character of
quantum mechanics the information and its variants are spread out over
a collection of names, i.e. a namespace, $\quotep{\phi}$. In symbols,
we represent the possible states of a physical entity, like the spin
of a particle, as expressions of the form

\begin{mathpar}
  S = \Sigma_{u \in \quotep{\phi}, {s \in S}}\outputp{u}{Q_{s}}
\end{mathpar}

We represent a test for a property as input at a name.

\begin{mathpar}
  T = \Sigma_{u \in \quotep{\phi}, {c \in C}}\prefix{u}{v}{P_{c}}
\end{mathpar}

Then a measurement at a specific name, $x$, is given by

\begin{mathpar}
  \testnp{T}{x}{S} = x \cdot (T^{\bot}[S]) = x \cdot (T|S)
\end{mathpar}

\subsection{The Born rule as an interpretation of non-determinism in the evolution of states}

The $\mathsf{COMM}$-rule clearly admits non-deterministic evolution of
states. Specifically, there are two basic forms of non-determinism
typically referred to as \emph{races} in the classical interpretation
of the rho-calculus. The first one is when there are more outputs than
there are inputs, and the second is when there are more inputs than
there are outputs. Symbolically,

\begin{mathpar}
  \prefix{x}{y_{1}}{P_{1}} \mathsf{|}\outputp{x}{Q}\mathsf{|} \prefix{x}{y_{2}}{P_{2}}
  \and
  \outputp{x}{Q_{1}} \mathsf{|}\prefix{x}{y}{P}\mathsf{|}\outputp{x}{Q_{2}}
\end{mathpar}

In the first case the two states reachable by reduction are
$P_{1}\substn{\quotep{Q}}{y_{1}}$ and
$P_{2}\substn{\quotep{Q}}{y_{2}}$; and in the second case the two
states reachable by reduction are $P\substn{\quotep{Q_{1}}}{y}$ and
$P\substn{\quotep{Q_{2}}}{y}$. In many implementations of the
rho-calculus, such as in the RChain blockchain, the interpretation of
this non-determinism is as a race. One of the communications beats the
other on an either as an asynchronous communication over a standard
protocol (such as TCP), or one pair of threads are scheduled to work
over the other pair, in a threading library inside an executing
process.

In stochastic versions of these calculi the idea is to extend channels
with \emph{rates}. Thus,

\begin{mathpar}
  \prefix{x}{y}{P} \mapsto \prefix{(x,r)}{y}{P}
  \and
  \outputp{x}{Q} \mapsto \outputp{(x,r)}{Q}
\end{mathpar}

The rates are used to determine a 1-norm probability distribution over
the possible $\mathsf{COMM}$ events, at a given number of reduction
steps from the starting expression.

There is nothing in principle that prevents using 2-norm probability
distributions, where the rates are given by complex numbers and
contribute probability amplitudes. That's what we develop in the next
section.
