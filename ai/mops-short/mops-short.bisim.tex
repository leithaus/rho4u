\section{Bisimulation}
Since the operational semantics is expressed as a transition system we recover a notion of bisimulation. There are two possible ways to generate the notion of bisimulation in this context. One uses the Leifer-Milner-Sewell approach of deriving a bisimulation from the rewrite rules \cite{DBLP:conf/concur/LeiferM00}. However, the technical apparatus is very heavy to work with. The other is to adapt barbed bisimulation developed for the asynchronous $\pi$-calculus to this setting.

The reason we need to use some care in developing the notion of bisimulation is that there are substitutions being generated and applied in many of the rules. So, a single label will not suffice. However, taking a query in the input space as a barb will. This notion of barbed bisimulation will provide a means of evaluating the correctness of compilation schemes to other languages. We illustrate this idea in the section on compiling MeTTa to the rho-calculus.
