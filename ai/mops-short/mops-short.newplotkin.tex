\section{Examples from computer science}
Since Milner's seminal Functions as processes paper, the gold standard for a presentation of an operational semantics is to present the algebra of states via a grammar (a monad) and a structural congruence (an algebra of the monad), and the rewrite rules in Plotkin-style SOS format \cite{DBLP:journals/mscs/Milner92} \cite{Plotkin04theorigins}.

\subsubsection{$\lambda$-calculus}

\paragraph{Algebra of States}
\begin{mathpar}
  \inferrule*[lab=Program]{}{ Term[V] \bc V \;\bm\; \lambda V.Term[V]  \;\bm\; \mathsf{(} Term[V] \; Term[V] \mathsf{)} }
\end{mathpar}

The structural congruence is the usual $\alpha$-equivalence, namely that $\lambda x.M \scong \lambda y.(M\{y / x\})$ when $y$ not free in $M$.

It is evident that $Term[V]$ is a monad and imposing $\alpha$-equivalence gives an algebra of the monad.

\paragraph{Transitions}
The rewrite rule is the well know $\beta$-reduction.
\begin{mathpar}
  \inferrule* [lab=Beta]{}{((\lambda x.M)N) \to M\{N / x\}}
\end{mathpar}

\subsubsection{$\pi$-calculus}

\paragraph{Algebra of States}
\begin{mathpar}
  \inferrule*[lab=Process]{}{Term[N] \bc \mathsf{0} \;\bm\; \mathsf{for}\mathsf{(}N \leftarrow N\mathsf{)}Term[N] \;\bm\; N\mathsf{!}\mathsf{(}N\mathsf{)} \;\bm\; \mathsf{(}\mathsf{new}\; N\mathsf{)}Term[N] \\ \;\bm\; Term[N] \mathsf{|} Term[N] \;\bm\; \mathsf{!}Term[N]}
\end{mathpar}

The structural congruence is the smallest equivalence relation including $\alpha$-equivalence making $(Term[N],\;\mathsf{|}\;,\mathsf{0})$ a commutative monoid, and respecting

\begin{eqnarray*}
  \mathsf{(}\mathsf{new}\; x\mathsf{)}\mathsf{(}\mathsf{new}\; x\mathsf{)}P & \scong & \mathsf{(}\mathsf{new}\; x\mathsf{)}P \\
  \mathsf{(}\mathsf{new}\; x\mathsf{)}\mathsf{(}\mathsf{new}\; y\mathsf{)}P & \scong & \mathsf{(}\mathsf{new}\; y\mathsf{)}\mathsf{(}\mathsf{new}\; x\mathsf{)}P \\
  (\mathsf{(}\mathsf{new}\; x\mathsf{)}P)\mathsf{|}Q & \scong & \mathsf{(}\mathsf{new}\; x\mathsf{)}(P\mathsf{|}Q), x \notin \freenames{Q}
\end{eqnarray*}

Again, it is evident that $Term[N]$ is a monad and imposing the structural congruence gives an algebra of the monad.

\paragraph{Transitions}
The rewrite rules divide into a core rule, and when rewrites apply in
context.
\begin{mathpar}
  \inferrule* [lab=COMM]{}{\mathsf{for}\mathsf{(}y \leftarrow x\mathsf{)}P \mathsf{|} x\mathsf{!}\mathsf{(}z\mathsf{)} \to P\{ z / y \}} \\
  \and  
  \inferrule* [lab=PAR]{P \to P'}{(\mathsf{new}\; x)P \to (\mathsf{new}\; x)P'}
  \and  
  \inferrule* [lab=PAR]{P \to P'}{P\mathsf{|}Q \to P'\mathsf{|}Q} \\
  \and
  \inferrule* [lab=STRUCT]{P \scong P' \to Q' \scong Q}{P \to Q}
\end{mathpar}

\cite{DBLP:journals/mscs/Milner92}

\subsubsection{rho-calculus}

\paragraph{Algebra of States}
Note that the rho-calculus is different from the $\lambda$-calculus and the $\pi$-calculus because it is \emph{not} dependent on a type of variables or names. However, it does give us the opportunity to expose how ground types, such as Booleans, numeric and string operations are imported into the calculus. The calculus does depend on the notion of a $0$ process. In fact, this could be any builtin functionality. The language rholang, derived from the rho-calculus, imports all literals as \emph{processes}. Note that this is in the spirit of the $\lambda$-calculus and languages derived from it: Booleans, numbers, and strings are terms, on the level with $\lambda$ terms.

So, the parameter to the monad for the rho-calculus takes the collection of builtin processes. Naturally, for all builtin processes other than $0$ there have to be reduction rules. For brevity, we take $Z = \{0\}$.
\begin{mathpar}
\inferrule* [lab=process] {} {Term[Z] \bc Z \;\bm\; \mathsf{for}(
  Name[Z] \leftarrow Name[Z] )Term[Z] \;\bm\; Name[Z]\mathsf{!}(Term[Z]) 
  \\ \and \;\bm\; \mathsf{*}Name[Z] \;\bm\; Term[Z]\mathsf{|}Term[Z] } \\
\and \inferrule* [lab=name] {}
     {Name[Z] \bc \mathsf{@}Term[Z] }
\end{mathpar}

The structural congruence is the smallest equivalence relation including $\alpha$-equivalence making $(P,\;\mathsf{|}\;,\mathsf{0})$ a commutative monoid.

Again, it is evident that $Term[Z]$ is a monad and imposing the structural congruence gives an algebra of the monad.

\paragraph{Transitions}
The rewrite rules divide into a core rule, and when rewrites apply in
context.
\begin{mathpar}
  \inferrule* [lab=COMM] {x_{t} \;\nameeq\; x_{s}} {\mathsf{for}(y \leftarrow x_{t} )P \;\mathsf{|}\; x_{s}!(Q)
  \red P\substn{\quotep{Q}}{y}} \\
  \and
  \inferrule* [lab=PAR]{P \red P'}{P\mathsf{|}Q \red P'\mathsf{|}Q}
  \and
  \inferrule* [lab=STRUCT]{{P \;\scong\; P'} \andalso {P' \red Q'} \andalso {Q' \;\scong\; Q}}{P \red Q}
\end{mathpar}

\cite{DBLP:journals/entcs/MeredithR05}

\subsubsection{Register machines and WYSIWYG semantics}One important point about the previous three
examples is that they are all examples of WYSIWYG operational
semantics in the sense that the states \emph{are} the terms of the
calculi. Register machines, such as the famous SECD machine, or more industrially robust machines, like the JVM, are not strictly WYSIWYG.In fact, the register based states are not strictly a monad, but contain a monad. In particular, the terms in the language are only
part of the state, which includes elements such as a stack, a heap, or other registers.

WYSIWYG models make static analysis dramatically
simpler. Specifically, an analyzer only has to look at terms in the
language. Language designs that do not offer a WYSIWYG presentation
typically cannot be analyzed entirely through the terms of the
language. In those cases properties like safety, liveness, or security
are properties of the states, not just the program terms.
