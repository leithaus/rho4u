\section{Towards a common language}
Three of the most successful branches of scientific discourse all agree on the shape of a model adequate for expressing and effecting computation. Physics, computer science, and mathematics all use the same standard shape. A model adequate for computation comes with an algebra of states and “laws of motion.”

One paradigmatic example from physics is Hilbert spaces and the Schroedinger equation. In computer science and mathematics the algebra of states is further broken down into a monad (the free algebra of states) and an algebra of the monad recorded as some equations on the free algebra.

Computer science represents laws of motion, aka state transitions, as rewrite rules exploiting the structure of states to determine transitions to new states. Mainstream mathematics is a more recognizable generalization of physics, coding state transitions, aka behavior, via morphisms (including automorphisms) between state spaces.

But all three agree to a high degree of specificity on what ingredients go into a formal presentation adequate for effecting computation.

\subsection{Examples from computer science}

\subsubsection{$\lambda$-calculus}

TBD

\subsubsection{$\pi$-calculus}

TBD

\subsubsection{rho-calculus}

\subsubsection{The JVM}

TBD

