\section{Monoidal logic: a motivating example}
In this section we develop a (not so) toy example that motivated your
author's search for an algorithm of this generality. We present a
logic for reasoning about monoids.

\subsection{Freely generated monoids}
\begin{mathpar}
  \inferrule* [lab=Monoid-Expr] {} {M[G] \bc \mathsf{e} \;\bm\; g \;\bm\; M[G] \mathsf{*} M[G]}
\end{mathpar}

where $g \in G$. The syntax is only slightly unusual. It's almost
standard $\mathsf{BNF}$ except that the grammar is parametric in a set
$G$ of generators. Readers familiar with parametric types found in
languages like OCaml, F$\#$, Scala, Haskell, Java, etc should be very
comfortable with this notation. The grammar freely generates a
collection of expressions over $G$ involving $\mathsf{*}$ and $e$,
which we write $\mathcal{L}(M[G])$. We must whack this collection down
by the familiar relations.

\begin{eqnarray*}
  e \mathsf{*} m & = m = & m \mathsf{*} e \\
  (m_{1} \mathsf{*} m_{2}) \mathsf{*} m_{3} & = & m_{1} \mathsf{*} (m_{2} \mathsf{*} m_{3})
\end{eqnarray*}

In other words, formally the monoid is $\mathcal{L}(M[G])/=$.

\subsection{Monoid logic: syntax and semantics}
\begin{eqnarray*}
  \phi,\psi & \bc & \mathsf{true} \;\bm\; \neg \phi \;\bm\; \phi \& \psi \\
  & \;\bm\; & \mathbf{e} \;\bm\; \mathbf{g} \;\bm\; \phi \mathbf{*} \psi
\end{eqnarray*}

\begin{eqnarray*}
  & collection \; operations & \\
  \meaningof{\mathsf{true}} & = & \mathcal{L}(M[G]) \\
  \meaningof{\neg \phi} & = & \mathcal{L}(M[G]) \backslash \meaningof{\phi} \\
  \meaningof{\phi \& \psi} & = & \meaningof{\phi} \cap \meaningof{\psi} \\
  & spatial \; operations & \\
  \meaningof{\mathbf{e}} & = & \{ m \in \mathcal{L}(M[G]) : m = \mathsf{e} \} \\
  \meaningof{\mathbf{g}} & = & \{ m \in \mathcal{L}(M[G]) : m = g \} \\
  \meaningof{\phi \mathbf{*} \psi} & = & \{ m \in \mathcal{L}(M[G]) : \exists m_{1}, m_{2}.m = m_{1} \mathsf{*} m_{2}, m_{1} \in \meaningof{\phi}, m_{2} \in \meaningof{\psi} \}
\end{eqnarray*}

The logic neatly divides into operations on collections of witnesses
(the usual boolean connectives, interpreted as operations on sets),
and spatial operations, i.e. logical operations derived from the
syntax of monoid \emph{expressions}. It is completely algorithmically
generated. Any theory presented via generators and relations like this
(and hence any Lawvere theory, and hence a certain class of monads)
has a corresponding logic.

Yet, it is is very expressive. Consider the following formula.
\begin{eqnarray*}
  prime & := & \neg \mathbf{e} \; \& \; \neg (\neg \mathbf{e} \; \mathbf{*}\; \neg \mathbf{e})
\end{eqnarray*}

As the reader may verify, this picks out exactly the primes of any
monoid generated from a set $G$ of generators. In fact, an extension
of this algorithm was used to \emph{generate} namespace logic. The
extension adds a third section to the logic that provides an
algorithmic treatment of the Hennessy-Milner style modal connectives
corresponding to the rewrite rules of the rho-calculus. Monoidal logic
is a proper fragement of namespace logic and the formula for primes
defined above can also be used to pick out \emph{single-threaded}
processes in the rho-calculus, illustrating an unexpected connection
between primality and sequentiality.

This analysis was the beginning of a search for the proper expression
of the algorithm that captures not only model-checking semantics like
the one above, but a type and judgment system. The search lead to a
category theoretic treatment. Originally, it seemed that the approach
might be captured by distributive laws amongst monads. However, some
recent no-go theorems for distributive laws showed that the approach
could not work for the parameterization contemplated.

In this connection, it is worth describing the intuitions that have
lead to the algorithm in its current form and scope. The reality is
that when taken together Curry-Howard and realizability force a
particular shape to the semantics of logical formulae. They must be
collections of witnesses, where the witnesses are computations
evincing the properties they bear witness to. Traditionally, logicians
and computer scientists alike have chosen sets as the natural form of
collection. However, that is not the only choice. In fact, it is not
the only natural choice.

This perspective becomes clear when we think about the other role that
logic plays. Logic is also the basis of query. Currently, Github is
only searchable in terms meta-data. Imagine querying it using types:
given a type, a program that type checks is a match, otherwise a
miss. The developers of Hoogle \cite{mitchell_2011} tried this for
searching Haskell's Cabal package system. The problem with Hoogle is
that the type system is too coarse. Both
\begin{mathpar}
  \inferrule* {}{\mathsf{sort} : \mathsf{list}\; \mathsf{int} \to \mathsf{list} \; \mathsf{int}}\\
  \and
  and \\
  \and
  \inferrule* {} {\; \mathsf{shuffle} : \mathsf{list}\;\mathsf{int} \to \mathsf{list} \; \mathsf{int}}
\end{mathpar}

bear the same type signature, but they do opposite things. Spatial
behavioral types, however, yield a considerably more expressive query
language.

Note, however, that Github repos are not \emph{sets}. They undergo
update, including insertion, deletion, and modification. Being able to
parameterize the algorithm to take as input the kind of collections
used to collect witnesses, and thereby generate the corresponding
logical connectives is vital to be able to consider query applications
like the one contemplated here.

Christian Williams and Mike Stay devised the core of the formalization
of the algorithm considered here. However, there are some key
differences.


