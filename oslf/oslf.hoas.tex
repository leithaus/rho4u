\section{Specifying a model of computation: lambda theories and higher order abstract syntax}

We follow Fiore and Hur \cite{DBLP:conf/csl/FioreH10} closely.

\begin{definition}
  A second order signature $\Sigma$ enjoys the following data $\Sigma = (T,O,|-|)$ with
  \begin{itemize}
    \item $T$ a set of types
    \item $O$ a set of operators
      \item $|-| : (T^{*} \times T^{*} \to T)$ and arity function.
  \end{itemize}  
\end{definition}

We write $o : (\arrvec{\sigma_{1}})\tau_{1} \times \ldots \times (\arrvec{\sigma_{n}})\tau_{n} \to \tau$ when $|o| = ((\arrvec{\sigma_{1}})\tau_{1}, \ldots, (\arrvec{\sigma_{n}})\tau_{n}, \tau)$

\subsection{Typing contexts}
We will consider terms in typing contexts. Typing contexts have two
zones, each respectively typing variables and metavariables. Variable
typings are types. Metavariable typings are parameterised types: a
metavariable of type $[\sigma_{1},\ldots,\sigma_{n}]\tau$, when
parameterised by terms of type $\sigma_{1},\ldots,\sigma_{n}$ yield a
term of type $\tau$. Thus, we use the following representation for
typing contexts $m_{1} : [\arrvec{\sigma_{1}}]\tau_{1}, \ldots, m_{j}: [\arrvec{\sigma_{j}}]\tau_{j} \rhd x_{1} : \sigma'_{1}, \ldots, x_{k}: \sigma'_{k}$.

\subsection{Terms}
 Signatures give rise to terms. These are built up by means of operators
from both variables and metavariables, and hence referred to as second-order.
Terms are considered up to $\alpha$-equivalence by stipulating that for every operator $o$ in the term $o(\ldots,(\arrvec{x}_{i})t_{i},\ldots)$ the $\arrvec{x}_{i}$ are bound in $t_{i}$.
