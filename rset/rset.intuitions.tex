\section{Intuitions}
We imagine two copies of FM set theory, say a \textcolor{red}{red one}
and a black one. The \textcolor{red}{red theory} has red
comprehensions,\textcolor{red}{$\{ \ldots \}$} and a red element-of
relation, \textcolor{red}{$\in$}, while the black theory,
correspondingly has black comprehensions, $\{ \ldots \}$ and a black
element-of relation, $\in$. The red element-of relation,
\textcolor{red}{$\in$}, can ``see into'' red sets, \textcolor{red}{$\{
  \ldots \}$}, but not black ones. Meanwhile, the black element-of
relation, $\in$, can ``see into'' black sets $\{ \ldots \}$, but not
red ones. Thus, the black sets can serve as atoms for the red sets,
while the red sets can serve as atoms for the black sets.

So, the two theories are mutually recursively defined. This mutual
recursion is closely linked to game semantics in the sense of
Abramksy, Hyland, and Ong, et al. Without loss of generality, we can
think of \textcolor{red}{red} as player and black as opponent. A
well-formed red set, \textcolor{red}{$S$}, is a winning strategy for
player, while a well-formed black set, $S$, is a winning strategy for
opponent.
