\section{Formal theory}
The judgment \(\Sigma; \Gamma \vdash S\) is pronounced
``\(\Sigma\) thinks that \(S\) is well-formed, given
dependencies, \(\Gamma\) .'' Similarly,
\(\Sigma; \Gamma \vdash x\) is pronounced
``\(\Sigma\) thinks that \(x\) is an admissible atom, given
dependencies, \(\Gamma\).''

We think of $X$ as some infinite supply of ``atoms'' and
$\mathsf{S}[X]$ as a \emph{stream} of atoms drawn from $X$. We assume
basic stream operations on $\mathsf{S}[X]$, such as
$\mathsf{take}(\mathsf{S}[X],n)$ which returns the pair
$(\sigma,\Sigma)$ where $\sigma = x_{1}:\ldots:x_{n}$ and $\Sigma = \mathsf{drop}(\mathsf{S}[X],n)$. Given $\rho = \{j_{1}/i_{1}:\ldots:j_{n}/i_{n}\}$ the operation
$\mathsf{swap}(\mathsf{S}[X],\rho)$ denotes the application of the
permutation $\rho$ to $\mathsf{S}[X]$.

Thus, $\mathsf{S}[X]$ may be thought of as an ordering on $X$.

A rule of the form

\[\frac{ H_1, \ldots , H_n }{ C }R\]

is pronounced ``\(R\) concludes that \(C\) given \(H_1\), \(\ldots\),
\(H_n\)''.

We use $e,f,\ldots$ to range over sets and atoms.

The rules for judging when an atom is admissible or set is well
formed are as follows

\[\frac{ }{ \mathsf{S}[X]; () \vdash \emptyset}Foundation\]

\[\frac{ (x,\Sigma) = \mathsf{take}(\mathsf{S}[X],1)}{ \Sigma; x \vdash \{ x \}}Atomicity\]

\[\frac{ \Sigma; \Gamma \vdash S }{ \Sigma; \Gamma \vdash \{ S \}}Nesting\]

\[\frac{ \Sigma; \Gamma_1 \vdash S_1 \; \Sigma; \Gamma_2 \vdash S_2}{ \Sigma; \Gamma_1, \Gamma_2 \vdash S_1 \cup S_2}Union\]

\[\frac{ \Sigma; \Gamma_1 \vdash S_1 \; \Sigma; \Gamma_2 \vdash S_2}{ \Sigma; \Gamma_1, \Gamma_2 \vdash S_1 \cap S_2}Intersection\]

\[\frac{ \Sigma; \Gamma_1 \vdash S_1 \; \Sigma; \Gamma_2 \vdash S_2}{ \Sigma; \Gamma_1, \Gamma_2 \vdash S_1 \setminus S_2}Subtraction\]

\[\frac{ \Sigma; \Gamma \vdash S \; \Sigma; \Gamma_{1} \vdash S_{1} \;\ldots\; \Sigma; \Gamma_{n} \vdash S_{2}\; \Sigma; \Delta \vdash f : S_{1} \times \ldots \times S_{n} \to S }{ \Sigma; \Gamma,\Gamma_{1},\ldots, \Gamma_{n},\Delta \vdash \{ f(s_{1},\ldots,s_{n}) \; : \; s_1 \in S_{1},\; \ldots\;, s_{n} \in S_{n}\}}Comprehension\]

\subsection{Equations}\label{equations}

The syntactic theory is too fine grained. It makes syntactic
distinctions that do not correspond to distinct sets. We erase these
syntactic distinctions with a set of equations on set expressions.

\[\frac{ \Sigma; \Gamma \vdash e }{ \Sigma; \Gamma  \vdash e = e}Identity\]
\[\frac{\Sigma; \Gamma_{1} \vdash _{1}\; \Sigma; \Gamma_{2} \vdash e_{2}\; \Sigma; \Gamma_{3}  \vdash e_{1} = e_{2}}{ \Sigma; \Gamma_{3}  \vdash e_{1} = e_{2}}Symmetry\]
\[\frac{\Sigma; \Gamma_{1}  \vdash e_{1}\; \Sigma; \Gamma_{2}  \vdash e_{2}\; \Sigma; \Gamma_{3} \vdash e_{3}\; \Sigma; \Gamma_{4}  \vdash e_{1} = e_{2} \; \Sigma; \Gamma_{5}  \vdash e_{2} = e_{3}}{ \Sigma; \Gamma_{4},\Gamma_{5}  \vdash e_{1} = e_{3}}Transitivity\]

\[\frac{\Sigma; \Gamma  \vdash S}{\Sigma; \Gamma \vdash S \cup S =  S}Idempotence_{\cup}\]
\[\frac{\Sigma; \Gamma  \vdash S}{\Sigma; \Gamma \vdash S \cap S =  S}Idempotence_{\cap}\]


