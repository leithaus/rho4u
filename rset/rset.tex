\documentclass[12pt]{llncs}

\usepackage[pdftex]{hyperref}                   
\usepackage {mathpartir}
\usepackage{bigpage}
\usepackage{bcprules}
\usepackage {listings}
\usepackage{amsmath}
\usepackage{amssymb}
\usepackage{amsfonts}
\usepackage{amstext}
\usepackage{latexsym}
\usepackage{longtable}
\usepackage{stmaryrd}
%\usepackage{fdsymbol}
\usepackage{graphicx} 
%\usepackage[margins=2.5cm,nohead,nofoot]{geometry}
%\usepackage{geometry}
\usepackage{color}
\usepackage{xcolor}


%\include{myPreamble}
\documentclass[12pt]{llncs}

\usepackage[pdftex]{hyperref}                   
\usepackage {mathpartir}
\usepackage{bigpage}
\usepackage{bcprules}
\usepackage {listings}
\usepackage{amsmath}
\usepackage{amssymb}
\usepackage{amsfonts}
\usepackage{amstext}
\usepackage{latexsym}
\usepackage{longtable}
\usepackage{stmaryrd}
%\usepackage{fdsymbol}
\usepackage{graphicx} 
%\usepackage[margins=2.5cm,nohead,nofoot]{geometry}
%\usepackage{geometry}
\usepackage{color}
\usepackage{xcolor}


%\include{myPreamble}
\documentclass[12pt]{llncs}

\usepackage[pdftex]{hyperref}                   
\usepackage {mathpartir}
\usepackage{bigpage}
\usepackage{bcprules}
\usepackage {listings}
\usepackage{amsmath}
\usepackage{amssymb}
\usepackage{amsfonts}
\usepackage{amstext}
\usepackage{latexsym}
\usepackage{longtable}
\usepackage{stmaryrd}
%\usepackage{fdsymbol}
\usepackage{graphicx} 
%\usepackage[margins=2.5cm,nohead,nofoot]{geometry}
%\usepackage{geometry}
\usepackage{color}
\usepackage{xcolor}


%\include{myPreamble}
\documentclass[12pt]{llncs}

\usepackage[pdftex]{hyperref}                   
\usepackage {mathpartir}
\usepackage{bigpage}
\usepackage{bcprules}
\usepackage {listings}
\usepackage{amsmath}
\usepackage{amssymb}
\usepackage{amsfonts}
\usepackage{amstext}
\usepackage{latexsym}
\usepackage{longtable}
\usepackage{stmaryrd}
%\usepackage{fdsymbol}
\usepackage{graphicx} 
%\usepackage[margins=2.5cm,nohead,nofoot]{geometry}
%\usepackage{geometry}
\usepackage{color}
\usepackage{xcolor}


%\include{myPreamble}
\include{rset.local} 

%\ifpdf
%\usepackage[pdftex]{graphicx}
%\else
%\usepackage{graphicx}
%\fi

 % \ifpdf
%  \usepackage{pdfsync}
%  \if


%\title{Brief Article}
%\author{David F. Snyder}
%\author{L.G. Meredith}

%\address{Dept. of Math., Texas State University--San Marcos, San Marcos, TX 78666}
       
\pagestyle{empty}


\begin{document}

\lstset{language=[Objective]Caml,frame=shadowbox}


\input{rset.front}

In the early 2000's Pitts and Gabbay's use of Fraenkl-Mostowski set
theory (FM set theory), a theory of sets with atoms, to model nominal phenomena,
sparked a vibrant line of research. \cite{DBLP:journals/fac/GabbayP02}
\cite{DBLP:journals/jcss/Clouston14}. As Chaitin points out, the
axioms and assumptions of a theory comprise its risk
\cite{chaitin1999unknowable}. This observation raises an interesting
point: is it possible to have a set theory with atoms without taking
on \emph{infinite} risk? It turns out it is.

We describe a theory of sets that enjoys both the cleanliness of
$\mathsf{ZF}$, that is arising only from the empty set, and yet with
an infinite supply of atoms.

\input{rset.intuitions}

\input{rset.presentation}

\input{rset.stdfmset}

\input{rset.mainthm}

\input{rset.conc}

\input{rset.ack}

\bibliographystyle{plain}   
\bibliography{rset.bib}

\end{document}
 

%\ifpdf
%\usepackage[pdftex]{graphicx}
%\else
%\usepackage{graphicx}
%\fi

 % \ifpdf
%  \usepackage{pdfsync}
%  \if


%\title{Brief Article}
%\author{David F. Snyder}
%\author{L.G. Meredith}

%\address{Dept. of Math., Texas State University--San Marcos, San Marcos, TX 78666}
       
\pagestyle{empty}


\begin{document}

\lstset{language=[Objective]Caml,frame=shadowbox}


\def\lastname{Meredith}

\title{Notes on a reflective theory of sets}
\titlerunning{oslf}

\author{ Lucius Gregory Meredith\inst{1}
}
\institute{
        {Managing Partner, RTech Unchained 9336 California Ave SW, Seattle, WA 98103, USA} \\
  \email{ lgreg.meredith@gmail.com } 
}
 

\maketitle              % typeset the title of the contribution

%%% ----------------------------------------------------------------------

\begin{abstract}

  We describe a universe of set theories enjoying atoms without enduring infinite risk.

\end{abstract}

%\keywords{Process calculi, knots, invariants}

% \begin{keyword}
% concurrency, message-passing, process calculus, reflection, program logic
% \end{keyword}

%\end{frontmatter}


In the early 2000's Pitts and Gabbay's use of Fraenkl-Mostowski set
theory (FM set theory), a theory of sets with atoms, to model nominal phenomena,
sparked a vibrant line of research. \cite{DBLP:journals/fac/GabbayP02}
\cite{DBLP:journals/jcss/Clouston14}. As Chaitin points out, the
axioms and assumptions of a theory comprise its risk
\cite{chaitin1999unknowable}. This observation raises an interesting
point: is it possible to have a set theory with atoms without taking
on \emph{infinite} risk? It turns out it is.

We describe a theory of sets that enjoys both the cleanliness of
$\mathsf{ZF}$, that is arising only from the empty set, and yet with
an infinite supply of atoms.

\section{Intuitions}
We imagine two copies of FM set theory, say a \textcolor{red}{red one}
and a black one. The \textcolor{red}{red theory} has red
comprehensions,\textcolor{red}{$\{ \ldots \}$} and a red element-of
relation, \textcolor{red}{$\in$}, while the black theory,
correspondingly has black comprehensions, $\{ \ldots \}$ and a black
element-of relation, $\in$. The red element-of relation,
\textcolor{red}{$\in$}, can ``see into'' red sets, \textcolor{red}{$\{
  \ldots \}$}, but not black ones. Meanwhile, the black element-of
relation, $\in$, can ``see into'' black sets $\{ \ldots \}$, but not
red ones. Thus, the black sets can serve as atoms for the red sets,
while the red sets can serve as atoms for the black sets.

So, the two theories are mutually recursively defined. This mutual
recursion is closely linked to game semantics in the sense of
Abramksy, Hyland, and Ong, et al. Without loss of generality, we can
think of \textcolor{red}{red} as player and black as opponent. A
well-formed red set, \textcolor{red}{$S$}, is a winning strategy for
player, while a well-formed black set, $S$, is a winning strategy for
opponent.


\section{Formal theory}
The judgment \(\Sigma; \Gamma \vdash S\) is pronounced
``\(\Sigma\) thinks that \(S\) is well-formed, given
dependencies, \(\Gamma\) .'' Similarly,
\(\Sigma; \Gamma \vdash x\) is pronounced
``\(\Sigma\) thinks that \(x\) is an admissible atom, given
dependencies, \(\Gamma\).''

We think of $X$ as some infinite supply of ``atoms'' and
$\mathsf{S}[X]$ as a \emph{stream} of atoms drawn from $X$. We assume
basic stream operations on $\mathsf{S}[X]$, such as
$\mathsf{take}(\mathsf{S}[X],n)$ which returns the pair
$(\sigma,\Sigma)$ where $\sigma = x_{1}:\ldots:x_{n}$ and $\Sigma = \mathsf{drop}(\mathsf{S}[X],n)$. Given $\rho = \{j_{1}/i_{1}:\ldots:j_{n}/i_{n}\}$ the operation
$\mathsf{swap}(\mathsf{S}[X],\rho)$ denotes the application of the
permutation $\rho$ to $\mathsf{S}[X]$.

Thus, $\mathsf{S}[X]$ may be thought of as an ordering on $X$.

A rule of the form

\[\frac{ H_1, \ldots , H_n }{ C }R\]

is pronounced ``\(R\) concludes that \(C\) given \(H_1\), \(\ldots\),
\(H_n\)''.

We use $e,f,\ldots$ to range over sets and atoms.

The rules for judging when an atom is admissible or set is well
formed are as follows

\[\frac{ }{ \mathsf{S}[X]; () \vdash \emptyset}Foundation\]

\[\frac{ (x,\Sigma) = \mathsf{take}(\mathsf{S}[X],1)}{ \Sigma; x \vdash \{ x \}}Atomicity\]

\[\frac{ \Sigma; \Gamma \vdash S }{ \Sigma; \Gamma \vdash \{ S \}}Nesting\]

\[\frac{ \Sigma; \Gamma_1 \vdash S_1 \; \Sigma; \Gamma_2 \vdash S_2}{ \Sigma; \Gamma_1, \Gamma_2 \vdash S_1 \cup S_2}Union\]

\[\frac{ \Sigma; \Gamma_1 \vdash S_1 \; \Sigma; \Gamma_2 \vdash S_2}{ \Sigma; \Gamma_1, \Gamma_2 \vdash S_1 \cap S_2}Intersection\]

\[\frac{ \Sigma; \Gamma_1 \vdash S_1 \; \Sigma; \Gamma_2 \vdash S_2}{ \Sigma; \Gamma_1, \Gamma_2 \vdash S_1 \setminus S_2}Subtraction\]

\[\frac{ \Sigma; \Gamma \vdash S \; \Sigma; \Gamma_{1} \vdash S_{1} \;\ldots\; \Sigma; \Gamma_{n} \vdash S_{2}\; \Sigma; \Delta \vdash f : S_{1} \times \ldots \times S_{n} \to S }{ \Sigma; \Gamma,\Gamma_{1},\ldots, \Gamma_{n},\Delta \vdash \{ f(s_{1},\ldots,s_{n}) \; : \; s_1 \in S_{1},\; \ldots\;, s_{n} \in S_{n}\}}Comprehension\]

\subsection{Equations}\label{equations}

The syntactic theory is too fine grained. It makes syntactic
distinctions that do not correspond to distinct sets. We erase these
syntactic distinctions with a set of equations on set expressions.

\[\frac{ \Sigma; \Gamma \vdash e }{ \Sigma; \Gamma  \vdash e = e}Identity\]
\[\frac{\Sigma; \Gamma_{1} \vdash _{1}\; \Sigma; \Gamma_{2} \vdash e_{2}\; \Sigma; \Gamma_{3}  \vdash e_{1} = e_{2}}{ \Sigma; \Gamma_{3}  \vdash e_{1} = e_{2}}Symmetry\]
\[\frac{\Sigma; \Gamma_{1}  \vdash e_{1}\; \Sigma; \Gamma_{2}  \vdash e_{2}\; \Sigma; \Gamma_{3} \vdash e_{3}\; \Sigma; \Gamma_{4}  \vdash e_{1} = e_{2} \; \Sigma; \Gamma_{5}  \vdash e_{2} = e_{3}}{ \Sigma; \Gamma_{4},\Gamma_{5}  \vdash e_{1} = e_{3}}Transitivity\]

\[\frac{\Sigma; \Gamma  \vdash S}{\Sigma; \Gamma \vdash S \cup S =  S}Idempotence_{\cup}\]
\[\frac{\Sigma; \Gamma  \vdash S}{\Sigma; \Gamma \vdash S \cap S =  S}Idempotence_{\cap}\]




\documentclass[12pt]{llncs}

\usepackage[pdftex]{hyperref}                   
\usepackage {mathpartir}
\usepackage{bigpage}
\usepackage{bcprules}
\usepackage {listings}
\usepackage{amsmath}
\usepackage{amssymb}
\usepackage{amsfonts}
\usepackage{amstext}
\usepackage{latexsym}
\usepackage{longtable}
\usepackage{stmaryrd}
%\usepackage{fdsymbol}
\usepackage{graphicx} 
%\usepackage[margins=2.5cm,nohead,nofoot]{geometry}
%\usepackage{geometry}
\usepackage{color}
\usepackage{xcolor}


%\include{myPreamble}
\include{rset.local} 

%\ifpdf
%\usepackage[pdftex]{graphicx}
%\else
%\usepackage{graphicx}
%\fi

 % \ifpdf
%  \usepackage{pdfsync}
%  \if


%\title{Brief Article}
%\author{David F. Snyder}
%\author{L.G. Meredith}

%\address{Dept. of Math., Texas State University--San Marcos, San Marcos, TX 78666}
       
\pagestyle{empty}


\begin{document}

\lstset{language=[Objective]Caml,frame=shadowbox}


\input{rset.front}

In the early 2000's Pitts and Gabbay's use of Fraenkl-Mostowski set
theory (FM set theory), a theory of sets with atoms, to model nominal phenomena,
sparked a vibrant line of research. \cite{DBLP:journals/fac/GabbayP02}
\cite{DBLP:journals/jcss/Clouston14}. As Chaitin points out, the
axioms and assumptions of a theory comprise its risk
\cite{chaitin1999unknowable}. This observation raises an interesting
point: is it possible to have a set theory with atoms without taking
on \emph{infinite} risk? It turns out it is.

We describe a theory of sets that enjoys both the cleanliness of
$\mathsf{ZF}$, that is arising only from the empty set, and yet with
an infinite supply of atoms.

\input{rset.intuitions}

\input{rset.presentation}

\input{rset.stdfmset}

\input{rset.mainthm}

\input{rset.conc}

\input{rset.ack}

\bibliographystyle{plain}   
\bibliography{rset.bib}

\end{document}


\documentclass[12pt]{llncs}

\usepackage[pdftex]{hyperref}                   
\usepackage {mathpartir}
\usepackage{bigpage}
\usepackage{bcprules}
\usepackage {listings}
\usepackage{amsmath}
\usepackage{amssymb}
\usepackage{amsfonts}
\usepackage{amstext}
\usepackage{latexsym}
\usepackage{longtable}
\usepackage{stmaryrd}
%\usepackage{fdsymbol}
\usepackage{graphicx} 
%\usepackage[margins=2.5cm,nohead,nofoot]{geometry}
%\usepackage{geometry}
\usepackage{color}
\usepackage{xcolor}


%\include{myPreamble}
\include{rset.local} 

%\ifpdf
%\usepackage[pdftex]{graphicx}
%\else
%\usepackage{graphicx}
%\fi

 % \ifpdf
%  \usepackage{pdfsync}
%  \if


%\title{Brief Article}
%\author{David F. Snyder}
%\author{L.G. Meredith}

%\address{Dept. of Math., Texas State University--San Marcos, San Marcos, TX 78666}
       
\pagestyle{empty}


\begin{document}

\lstset{language=[Objective]Caml,frame=shadowbox}


\input{rset.front}

In the early 2000's Pitts and Gabbay's use of Fraenkl-Mostowski set
theory (FM set theory), a theory of sets with atoms, to model nominal phenomena,
sparked a vibrant line of research. \cite{DBLP:journals/fac/GabbayP02}
\cite{DBLP:journals/jcss/Clouston14}. As Chaitin points out, the
axioms and assumptions of a theory comprise its risk
\cite{chaitin1999unknowable}. This observation raises an interesting
point: is it possible to have a set theory with atoms without taking
on \emph{infinite} risk? It turns out it is.

We describe a theory of sets that enjoys both the cleanliness of
$\mathsf{ZF}$, that is arising only from the empty set, and yet with
an infinite supply of atoms.

\input{rset.intuitions}

\input{rset.presentation}

\input{rset.stdfmset}

\input{rset.mainthm}

\input{rset.conc}

\input{rset.ack}

\bibliographystyle{plain}   
\bibliography{rset.bib}

\end{document}


\section{Conclusions and future work}

We have presented a formal theory of sets that reconciles set theory with atoms with a set theory built only from the empty set.

% subsection other_calculi_other_bisimulations_and_geometry_as_behavior (end)




\paragraph{Acknowledgments.}
The author wishes to thank Jamie Gabbay for stimulating discussions
that led the author to devise this theory.


\bibliographystyle{plain}   
\bibliography{rset.bib}

\end{document}
 

%\ifpdf
%\usepackage[pdftex]{graphicx}
%\else
%\usepackage{graphicx}
%\fi

 % \ifpdf
%  \usepackage{pdfsync}
%  \if


%\title{Brief Article}
%\author{David F. Snyder}
%\author{L.G. Meredith}

%\address{Dept. of Math., Texas State University--San Marcos, San Marcos, TX 78666}
       
\pagestyle{empty}


\begin{document}

\lstset{language=[Objective]Caml,frame=shadowbox}


\def\lastname{Meredith}

\title{Notes on a reflective theory of sets}
\titlerunning{oslf}

\author{ Lucius Gregory Meredith\inst{1}
}
\institute{
        {Managing Partner, RTech Unchained 9336 California Ave SW, Seattle, WA 98103, USA} \\
  \email{ lgreg.meredith@gmail.com } 
}
 

\maketitle              % typeset the title of the contribution

%%% ----------------------------------------------------------------------

\begin{abstract}

  We describe a universe of set theories enjoying atoms without enduring infinite risk.

\end{abstract}

%\keywords{Process calculi, knots, invariants}

% \begin{keyword}
% concurrency, message-passing, process calculus, reflection, program logic
% \end{keyword}

%\end{frontmatter}


In the early 2000's Pitts and Gabbay's use of Fraenkl-Mostowski set
theory (FM set theory), a theory of sets with atoms, to model nominal phenomena,
sparked a vibrant line of research. \cite{DBLP:journals/fac/GabbayP02}
\cite{DBLP:journals/jcss/Clouston14}. As Chaitin points out, the
axioms and assumptions of a theory comprise its risk
\cite{chaitin1999unknowable}. This observation raises an interesting
point: is it possible to have a set theory with atoms without taking
on \emph{infinite} risk? It turns out it is.

We describe a theory of sets that enjoys both the cleanliness of
$\mathsf{ZF}$, that is arising only from the empty set, and yet with
an infinite supply of atoms.

\section{Intuitions}
We imagine two copies of FM set theory, say a \textcolor{red}{red one}
and a black one. The \textcolor{red}{red theory} has red
comprehensions,\textcolor{red}{$\{ \ldots \}$} and a red element-of
relation, \textcolor{red}{$\in$}, while the black theory,
correspondingly has black comprehensions, $\{ \ldots \}$ and a black
element-of relation, $\in$. The red element-of relation,
\textcolor{red}{$\in$}, can ``see into'' red sets, \textcolor{red}{$\{
  \ldots \}$}, but not black ones. Meanwhile, the black element-of
relation, $\in$, can ``see into'' black sets $\{ \ldots \}$, but not
red ones. Thus, the black sets can serve as atoms for the red sets,
while the red sets can serve as atoms for the black sets.

So, the two theories are mutually recursively defined. This mutual
recursion is closely linked to game semantics in the sense of
Abramksy, Hyland, and Ong, et al. Without loss of generality, we can
think of \textcolor{red}{red} as player and black as opponent. A
well-formed red set, \textcolor{red}{$S$}, is a winning strategy for
player, while a well-formed black set, $S$, is a winning strategy for
opponent.


\section{Formal theory}
The judgment \(\Sigma; \Gamma \vdash S\) is pronounced
``\(\Sigma\) thinks that \(S\) is well-formed, given
dependencies, \(\Gamma\) .'' Similarly,
\(\Sigma; \Gamma \vdash x\) is pronounced
``\(\Sigma\) thinks that \(x\) is an admissible atom, given
dependencies, \(\Gamma\).''

We think of $X$ as some infinite supply of ``atoms'' and
$\mathsf{S}[X]$ as a \emph{stream} of atoms drawn from $X$. We assume
basic stream operations on $\mathsf{S}[X]$, such as
$\mathsf{take}(\mathsf{S}[X],n)$ which returns the pair
$(\sigma,\Sigma)$ where $\sigma = x_{1}:\ldots:x_{n}$ and $\Sigma = \mathsf{drop}(\mathsf{S}[X],n)$. Given $\rho = \{j_{1}/i_{1}:\ldots:j_{n}/i_{n}\}$ the operation
$\mathsf{swap}(\mathsf{S}[X],\rho)$ denotes the application of the
permutation $\rho$ to $\mathsf{S}[X]$.

Thus, $\mathsf{S}[X]$ may be thought of as an ordering on $X$.

A rule of the form

\[\frac{ H_1, \ldots , H_n }{ C }R\]

is pronounced ``\(R\) concludes that \(C\) given \(H_1\), \(\ldots\),
\(H_n\)''.

We use $e,f,\ldots$ to range over sets and atoms.

The rules for judging when an atom is admissible or set is well
formed are as follows

\[\frac{ }{ \mathsf{S}[X]; () \vdash \emptyset}Foundation\]

\[\frac{ (x,\Sigma) = \mathsf{take}(\mathsf{S}[X],1)}{ \Sigma; x \vdash \{ x \}}Atomicity\]

\[\frac{ \Sigma; \Gamma \vdash S }{ \Sigma; \Gamma \vdash \{ S \}}Nesting\]

\[\frac{ \Sigma; \Gamma_1 \vdash S_1 \; \Sigma; \Gamma_2 \vdash S_2}{ \Sigma; \Gamma_1, \Gamma_2 \vdash S_1 \cup S_2}Union\]

\[\frac{ \Sigma; \Gamma_1 \vdash S_1 \; \Sigma; \Gamma_2 \vdash S_2}{ \Sigma; \Gamma_1, \Gamma_2 \vdash S_1 \cap S_2}Intersection\]

\[\frac{ \Sigma; \Gamma_1 \vdash S_1 \; \Sigma; \Gamma_2 \vdash S_2}{ \Sigma; \Gamma_1, \Gamma_2 \vdash S_1 \setminus S_2}Subtraction\]

\[\frac{ \Sigma; \Gamma \vdash S \; \Sigma; \Gamma_{1} \vdash S_{1} \;\ldots\; \Sigma; \Gamma_{n} \vdash S_{2}\; \Sigma; \Delta \vdash f : S_{1} \times \ldots \times S_{n} \to S }{ \Sigma; \Gamma,\Gamma_{1},\ldots, \Gamma_{n},\Delta \vdash \{ f(s_{1},\ldots,s_{n}) \; : \; s_1 \in S_{1},\; \ldots\;, s_{n} \in S_{n}\}}Comprehension\]

\subsection{Equations}\label{equations}

The syntactic theory is too fine grained. It makes syntactic
distinctions that do not correspond to distinct sets. We erase these
syntactic distinctions with a set of equations on set expressions.

\[\frac{ \Sigma; \Gamma \vdash e }{ \Sigma; \Gamma  \vdash e = e}Identity\]
\[\frac{\Sigma; \Gamma_{1} \vdash _{1}\; \Sigma; \Gamma_{2} \vdash e_{2}\; \Sigma; \Gamma_{3}  \vdash e_{1} = e_{2}}{ \Sigma; \Gamma_{3}  \vdash e_{1} = e_{2}}Symmetry\]
\[\frac{\Sigma; \Gamma_{1}  \vdash e_{1}\; \Sigma; \Gamma_{2}  \vdash e_{2}\; \Sigma; \Gamma_{3} \vdash e_{3}\; \Sigma; \Gamma_{4}  \vdash e_{1} = e_{2} \; \Sigma; \Gamma_{5}  \vdash e_{2} = e_{3}}{ \Sigma; \Gamma_{4},\Gamma_{5}  \vdash e_{1} = e_{3}}Transitivity\]

\[\frac{\Sigma; \Gamma  \vdash S}{\Sigma; \Gamma \vdash S \cup S =  S}Idempotence_{\cup}\]
\[\frac{\Sigma; \Gamma  \vdash S}{\Sigma; \Gamma \vdash S \cap S =  S}Idempotence_{\cap}\]




\documentclass[12pt]{llncs}

\usepackage[pdftex]{hyperref}                   
\usepackage {mathpartir}
\usepackage{bigpage}
\usepackage{bcprules}
\usepackage {listings}
\usepackage{amsmath}
\usepackage{amssymb}
\usepackage{amsfonts}
\usepackage{amstext}
\usepackage{latexsym}
\usepackage{longtable}
\usepackage{stmaryrd}
%\usepackage{fdsymbol}
\usepackage{graphicx} 
%\usepackage[margins=2.5cm,nohead,nofoot]{geometry}
%\usepackage{geometry}
\usepackage{color}
\usepackage{xcolor}


%\include{myPreamble}
\documentclass[12pt]{llncs}

\usepackage[pdftex]{hyperref}                   
\usepackage {mathpartir}
\usepackage{bigpage}
\usepackage{bcprules}
\usepackage {listings}
\usepackage{amsmath}
\usepackage{amssymb}
\usepackage{amsfonts}
\usepackage{amstext}
\usepackage{latexsym}
\usepackage{longtable}
\usepackage{stmaryrd}
%\usepackage{fdsymbol}
\usepackage{graphicx} 
%\usepackage[margins=2.5cm,nohead,nofoot]{geometry}
%\usepackage{geometry}
\usepackage{color}
\usepackage{xcolor}


%\include{myPreamble}
\include{rset.local} 

%\ifpdf
%\usepackage[pdftex]{graphicx}
%\else
%\usepackage{graphicx}
%\fi

 % \ifpdf
%  \usepackage{pdfsync}
%  \if


%\title{Brief Article}
%\author{David F. Snyder}
%\author{L.G. Meredith}

%\address{Dept. of Math., Texas State University--San Marcos, San Marcos, TX 78666}
       
\pagestyle{empty}


\begin{document}

\lstset{language=[Objective]Caml,frame=shadowbox}


\input{rset.front}

In the early 2000's Pitts and Gabbay's use of Fraenkl-Mostowski set
theory (FM set theory), a theory of sets with atoms, to model nominal phenomena,
sparked a vibrant line of research. \cite{DBLP:journals/fac/GabbayP02}
\cite{DBLP:journals/jcss/Clouston14}. As Chaitin points out, the
axioms and assumptions of a theory comprise its risk
\cite{chaitin1999unknowable}. This observation raises an interesting
point: is it possible to have a set theory with atoms without taking
on \emph{infinite} risk? It turns out it is.

We describe a theory of sets that enjoys both the cleanliness of
$\mathsf{ZF}$, that is arising only from the empty set, and yet with
an infinite supply of atoms.

\input{rset.intuitions}

\input{rset.presentation}

\input{rset.stdfmset}

\input{rset.mainthm}

\input{rset.conc}

\input{rset.ack}

\bibliographystyle{plain}   
\bibliography{rset.bib}

\end{document}
 

%\ifpdf
%\usepackage[pdftex]{graphicx}
%\else
%\usepackage{graphicx}
%\fi

 % \ifpdf
%  \usepackage{pdfsync}
%  \if


%\title{Brief Article}
%\author{David F. Snyder}
%\author{L.G. Meredith}

%\address{Dept. of Math., Texas State University--San Marcos, San Marcos, TX 78666}
       
\pagestyle{empty}


\begin{document}

\lstset{language=[Objective]Caml,frame=shadowbox}


\def\lastname{Meredith}

\title{Notes on a reflective theory of sets}
\titlerunning{oslf}

\author{ Lucius Gregory Meredith\inst{1}
}
\institute{
        {Managing Partner, RTech Unchained 9336 California Ave SW, Seattle, WA 98103, USA} \\
  \email{ lgreg.meredith@gmail.com } 
}
 

\maketitle              % typeset the title of the contribution

%%% ----------------------------------------------------------------------

\begin{abstract}

  We describe a universe of set theories enjoying atoms without enduring infinite risk.

\end{abstract}

%\keywords{Process calculi, knots, invariants}

% \begin{keyword}
% concurrency, message-passing, process calculus, reflection, program logic
% \end{keyword}

%\end{frontmatter}


In the early 2000's Pitts and Gabbay's use of Fraenkl-Mostowski set
theory (FM set theory), a theory of sets with atoms, to model nominal phenomena,
sparked a vibrant line of research. \cite{DBLP:journals/fac/GabbayP02}
\cite{DBLP:journals/jcss/Clouston14}. As Chaitin points out, the
axioms and assumptions of a theory comprise its risk
\cite{chaitin1999unknowable}. This observation raises an interesting
point: is it possible to have a set theory with atoms without taking
on \emph{infinite} risk? It turns out it is.

We describe a theory of sets that enjoys both the cleanliness of
$\mathsf{ZF}$, that is arising only from the empty set, and yet with
an infinite supply of atoms.

\section{Intuitions}
We imagine two copies of FM set theory, say a \textcolor{red}{red one}
and a black one. The \textcolor{red}{red theory} has red
comprehensions,\textcolor{red}{$\{ \ldots \}$} and a red element-of
relation, \textcolor{red}{$\in$}, while the black theory,
correspondingly has black comprehensions, $\{ \ldots \}$ and a black
element-of relation, $\in$. The red element-of relation,
\textcolor{red}{$\in$}, can ``see into'' red sets, \textcolor{red}{$\{
  \ldots \}$}, but not black ones. Meanwhile, the black element-of
relation, $\in$, can ``see into'' black sets $\{ \ldots \}$, but not
red ones. Thus, the black sets can serve as atoms for the red sets,
while the red sets can serve as atoms for the black sets.

So, the two theories are mutually recursively defined. This mutual
recursion is closely linked to game semantics in the sense of
Abramksy, Hyland, and Ong, et al. Without loss of generality, we can
think of \textcolor{red}{red} as player and black as opponent. A
well-formed red set, \textcolor{red}{$S$}, is a winning strategy for
player, while a well-formed black set, $S$, is a winning strategy for
opponent.


\section{Formal theory}
The judgment \(\Sigma; \Gamma \vdash S\) is pronounced
``\(\Sigma\) thinks that \(S\) is well-formed, given
dependencies, \(\Gamma\) .'' Similarly,
\(\Sigma; \Gamma \vdash x\) is pronounced
``\(\Sigma\) thinks that \(x\) is an admissible atom, given
dependencies, \(\Gamma\).''

We think of $X$ as some infinite supply of ``atoms'' and
$\mathsf{S}[X]$ as a \emph{stream} of atoms drawn from $X$. We assume
basic stream operations on $\mathsf{S}[X]$, such as
$\mathsf{take}(\mathsf{S}[X],n)$ which returns the pair
$(\sigma,\Sigma)$ where $\sigma = x_{1}:\ldots:x_{n}$ and $\Sigma = \mathsf{drop}(\mathsf{S}[X],n)$. Given $\rho = \{j_{1}/i_{1}:\ldots:j_{n}/i_{n}\}$ the operation
$\mathsf{swap}(\mathsf{S}[X],\rho)$ denotes the application of the
permutation $\rho$ to $\mathsf{S}[X]$.

Thus, $\mathsf{S}[X]$ may be thought of as an ordering on $X$.

A rule of the form

\[\frac{ H_1, \ldots , H_n }{ C }R\]

is pronounced ``\(R\) concludes that \(C\) given \(H_1\), \(\ldots\),
\(H_n\)''.

We use $e,f,\ldots$ to range over sets and atoms.

The rules for judging when an atom is admissible or set is well
formed are as follows

\[\frac{ }{ \mathsf{S}[X]; () \vdash \emptyset}Foundation\]

\[\frac{ (x,\Sigma) = \mathsf{take}(\mathsf{S}[X],1)}{ \Sigma; x \vdash \{ x \}}Atomicity\]

\[\frac{ \Sigma; \Gamma \vdash S }{ \Sigma; \Gamma \vdash \{ S \}}Nesting\]

\[\frac{ \Sigma; \Gamma_1 \vdash S_1 \; \Sigma; \Gamma_2 \vdash S_2}{ \Sigma; \Gamma_1, \Gamma_2 \vdash S_1 \cup S_2}Union\]

\[\frac{ \Sigma; \Gamma_1 \vdash S_1 \; \Sigma; \Gamma_2 \vdash S_2}{ \Sigma; \Gamma_1, \Gamma_2 \vdash S_1 \cap S_2}Intersection\]

\[\frac{ \Sigma; \Gamma_1 \vdash S_1 \; \Sigma; \Gamma_2 \vdash S_2}{ \Sigma; \Gamma_1, \Gamma_2 \vdash S_1 \setminus S_2}Subtraction\]

\[\frac{ \Sigma; \Gamma \vdash S \; \Sigma; \Gamma_{1} \vdash S_{1} \;\ldots\; \Sigma; \Gamma_{n} \vdash S_{2}\; \Sigma; \Delta \vdash f : S_{1} \times \ldots \times S_{n} \to S }{ \Sigma; \Gamma,\Gamma_{1},\ldots, \Gamma_{n},\Delta \vdash \{ f(s_{1},\ldots,s_{n}) \; : \; s_1 \in S_{1},\; \ldots\;, s_{n} \in S_{n}\}}Comprehension\]

\subsection{Equations}\label{equations}

The syntactic theory is too fine grained. It makes syntactic
distinctions that do not correspond to distinct sets. We erase these
syntactic distinctions with a set of equations on set expressions.

\[\frac{ \Sigma; \Gamma \vdash e }{ \Sigma; \Gamma  \vdash e = e}Identity\]
\[\frac{\Sigma; \Gamma_{1} \vdash _{1}\; \Sigma; \Gamma_{2} \vdash e_{2}\; \Sigma; \Gamma_{3}  \vdash e_{1} = e_{2}}{ \Sigma; \Gamma_{3}  \vdash e_{1} = e_{2}}Symmetry\]
\[\frac{\Sigma; \Gamma_{1}  \vdash e_{1}\; \Sigma; \Gamma_{2}  \vdash e_{2}\; \Sigma; \Gamma_{3} \vdash e_{3}\; \Sigma; \Gamma_{4}  \vdash e_{1} = e_{2} \; \Sigma; \Gamma_{5}  \vdash e_{2} = e_{3}}{ \Sigma; \Gamma_{4},\Gamma_{5}  \vdash e_{1} = e_{3}}Transitivity\]

\[\frac{\Sigma; \Gamma  \vdash S}{\Sigma; \Gamma \vdash S \cup S =  S}Idempotence_{\cup}\]
\[\frac{\Sigma; \Gamma  \vdash S}{\Sigma; \Gamma \vdash S \cap S =  S}Idempotence_{\cap}\]




\documentclass[12pt]{llncs}

\usepackage[pdftex]{hyperref}                   
\usepackage {mathpartir}
\usepackage{bigpage}
\usepackage{bcprules}
\usepackage {listings}
\usepackage{amsmath}
\usepackage{amssymb}
\usepackage{amsfonts}
\usepackage{amstext}
\usepackage{latexsym}
\usepackage{longtable}
\usepackage{stmaryrd}
%\usepackage{fdsymbol}
\usepackage{graphicx} 
%\usepackage[margins=2.5cm,nohead,nofoot]{geometry}
%\usepackage{geometry}
\usepackage{color}
\usepackage{xcolor}


%\include{myPreamble}
\include{rset.local} 

%\ifpdf
%\usepackage[pdftex]{graphicx}
%\else
%\usepackage{graphicx}
%\fi

 % \ifpdf
%  \usepackage{pdfsync}
%  \if


%\title{Brief Article}
%\author{David F. Snyder}
%\author{L.G. Meredith}

%\address{Dept. of Math., Texas State University--San Marcos, San Marcos, TX 78666}
       
\pagestyle{empty}


\begin{document}

\lstset{language=[Objective]Caml,frame=shadowbox}


\input{rset.front}

In the early 2000's Pitts and Gabbay's use of Fraenkl-Mostowski set
theory (FM set theory), a theory of sets with atoms, to model nominal phenomena,
sparked a vibrant line of research. \cite{DBLP:journals/fac/GabbayP02}
\cite{DBLP:journals/jcss/Clouston14}. As Chaitin points out, the
axioms and assumptions of a theory comprise its risk
\cite{chaitin1999unknowable}. This observation raises an interesting
point: is it possible to have a set theory with atoms without taking
on \emph{infinite} risk? It turns out it is.

We describe a theory of sets that enjoys both the cleanliness of
$\mathsf{ZF}$, that is arising only from the empty set, and yet with
an infinite supply of atoms.

\input{rset.intuitions}

\input{rset.presentation}

\input{rset.stdfmset}

\input{rset.mainthm}

\input{rset.conc}

\input{rset.ack}

\bibliographystyle{plain}   
\bibliography{rset.bib}

\end{document}


\documentclass[12pt]{llncs}

\usepackage[pdftex]{hyperref}                   
\usepackage {mathpartir}
\usepackage{bigpage}
\usepackage{bcprules}
\usepackage {listings}
\usepackage{amsmath}
\usepackage{amssymb}
\usepackage{amsfonts}
\usepackage{amstext}
\usepackage{latexsym}
\usepackage{longtable}
\usepackage{stmaryrd}
%\usepackage{fdsymbol}
\usepackage{graphicx} 
%\usepackage[margins=2.5cm,nohead,nofoot]{geometry}
%\usepackage{geometry}
\usepackage{color}
\usepackage{xcolor}


%\include{myPreamble}
\include{rset.local} 

%\ifpdf
%\usepackage[pdftex]{graphicx}
%\else
%\usepackage{graphicx}
%\fi

 % \ifpdf
%  \usepackage{pdfsync}
%  \if


%\title{Brief Article}
%\author{David F. Snyder}
%\author{L.G. Meredith}

%\address{Dept. of Math., Texas State University--San Marcos, San Marcos, TX 78666}
       
\pagestyle{empty}


\begin{document}

\lstset{language=[Objective]Caml,frame=shadowbox}


\input{rset.front}

In the early 2000's Pitts and Gabbay's use of Fraenkl-Mostowski set
theory (FM set theory), a theory of sets with atoms, to model nominal phenomena,
sparked a vibrant line of research. \cite{DBLP:journals/fac/GabbayP02}
\cite{DBLP:journals/jcss/Clouston14}. As Chaitin points out, the
axioms and assumptions of a theory comprise its risk
\cite{chaitin1999unknowable}. This observation raises an interesting
point: is it possible to have a set theory with atoms without taking
on \emph{infinite} risk? It turns out it is.

We describe a theory of sets that enjoys both the cleanliness of
$\mathsf{ZF}$, that is arising only from the empty set, and yet with
an infinite supply of atoms.

\input{rset.intuitions}

\input{rset.presentation}

\input{rset.stdfmset}

\input{rset.mainthm}

\input{rset.conc}

\input{rset.ack}

\bibliographystyle{plain}   
\bibliography{rset.bib}

\end{document}


\section{Conclusions and future work}

We have presented a formal theory of sets that reconciles set theory with atoms with a set theory built only from the empty set.

% subsection other_calculi_other_bisimulations_and_geometry_as_behavior (end)




\paragraph{Acknowledgments.}
The author wishes to thank Jamie Gabbay for stimulating discussions
that led the author to devise this theory.


\bibliographystyle{plain}   
\bibliography{rset.bib}

\end{document}


\documentclass[12pt]{llncs}

\usepackage[pdftex]{hyperref}                   
\usepackage {mathpartir}
\usepackage{bigpage}
\usepackage{bcprules}
\usepackage {listings}
\usepackage{amsmath}
\usepackage{amssymb}
\usepackage{amsfonts}
\usepackage{amstext}
\usepackage{latexsym}
\usepackage{longtable}
\usepackage{stmaryrd}
%\usepackage{fdsymbol}
\usepackage{graphicx} 
%\usepackage[margins=2.5cm,nohead,nofoot]{geometry}
%\usepackage{geometry}
\usepackage{color}
\usepackage{xcolor}


%\include{myPreamble}
\documentclass[12pt]{llncs}

\usepackage[pdftex]{hyperref}                   
\usepackage {mathpartir}
\usepackage{bigpage}
\usepackage{bcprules}
\usepackage {listings}
\usepackage{amsmath}
\usepackage{amssymb}
\usepackage{amsfonts}
\usepackage{amstext}
\usepackage{latexsym}
\usepackage{longtable}
\usepackage{stmaryrd}
%\usepackage{fdsymbol}
\usepackage{graphicx} 
%\usepackage[margins=2.5cm,nohead,nofoot]{geometry}
%\usepackage{geometry}
\usepackage{color}
\usepackage{xcolor}


%\include{myPreamble}
\include{rset.local} 

%\ifpdf
%\usepackage[pdftex]{graphicx}
%\else
%\usepackage{graphicx}
%\fi

 % \ifpdf
%  \usepackage{pdfsync}
%  \if


%\title{Brief Article}
%\author{David F. Snyder}
%\author{L.G. Meredith}

%\address{Dept. of Math., Texas State University--San Marcos, San Marcos, TX 78666}
       
\pagestyle{empty}


\begin{document}

\lstset{language=[Objective]Caml,frame=shadowbox}


\input{rset.front}

In the early 2000's Pitts and Gabbay's use of Fraenkl-Mostowski set
theory (FM set theory), a theory of sets with atoms, to model nominal phenomena,
sparked a vibrant line of research. \cite{DBLP:journals/fac/GabbayP02}
\cite{DBLP:journals/jcss/Clouston14}. As Chaitin points out, the
axioms and assumptions of a theory comprise its risk
\cite{chaitin1999unknowable}. This observation raises an interesting
point: is it possible to have a set theory with atoms without taking
on \emph{infinite} risk? It turns out it is.

We describe a theory of sets that enjoys both the cleanliness of
$\mathsf{ZF}$, that is arising only from the empty set, and yet with
an infinite supply of atoms.

\input{rset.intuitions}

\input{rset.presentation}

\input{rset.stdfmset}

\input{rset.mainthm}

\input{rset.conc}

\input{rset.ack}

\bibliographystyle{plain}   
\bibliography{rset.bib}

\end{document}
 

%\ifpdf
%\usepackage[pdftex]{graphicx}
%\else
%\usepackage{graphicx}
%\fi

 % \ifpdf
%  \usepackage{pdfsync}
%  \if


%\title{Brief Article}
%\author{David F. Snyder}
%\author{L.G. Meredith}

%\address{Dept. of Math., Texas State University--San Marcos, San Marcos, TX 78666}
       
\pagestyle{empty}


\begin{document}

\lstset{language=[Objective]Caml,frame=shadowbox}


\def\lastname{Meredith}

\title{Notes on a reflective theory of sets}
\titlerunning{oslf}

\author{ Lucius Gregory Meredith\inst{1}
}
\institute{
        {Managing Partner, RTech Unchained 9336 California Ave SW, Seattle, WA 98103, USA} \\
  \email{ lgreg.meredith@gmail.com } 
}
 

\maketitle              % typeset the title of the contribution

%%% ----------------------------------------------------------------------

\begin{abstract}

  We describe a universe of set theories enjoying atoms without enduring infinite risk.

\end{abstract}

%\keywords{Process calculi, knots, invariants}

% \begin{keyword}
% concurrency, message-passing, process calculus, reflection, program logic
% \end{keyword}

%\end{frontmatter}


In the early 2000's Pitts and Gabbay's use of Fraenkl-Mostowski set
theory (FM set theory), a theory of sets with atoms, to model nominal phenomena,
sparked a vibrant line of research. \cite{DBLP:journals/fac/GabbayP02}
\cite{DBLP:journals/jcss/Clouston14}. As Chaitin points out, the
axioms and assumptions of a theory comprise its risk
\cite{chaitin1999unknowable}. This observation raises an interesting
point: is it possible to have a set theory with atoms without taking
on \emph{infinite} risk? It turns out it is.

We describe a theory of sets that enjoys both the cleanliness of
$\mathsf{ZF}$, that is arising only from the empty set, and yet with
an infinite supply of atoms.

\section{Intuitions}
We imagine two copies of FM set theory, say a \textcolor{red}{red one}
and a black one. The \textcolor{red}{red theory} has red
comprehensions,\textcolor{red}{$\{ \ldots \}$} and a red element-of
relation, \textcolor{red}{$\in$}, while the black theory,
correspondingly has black comprehensions, $\{ \ldots \}$ and a black
element-of relation, $\in$. The red element-of relation,
\textcolor{red}{$\in$}, can ``see into'' red sets, \textcolor{red}{$\{
  \ldots \}$}, but not black ones. Meanwhile, the black element-of
relation, $\in$, can ``see into'' black sets $\{ \ldots \}$, but not
red ones. Thus, the black sets can serve as atoms for the red sets,
while the red sets can serve as atoms for the black sets.

So, the two theories are mutually recursively defined. This mutual
recursion is closely linked to game semantics in the sense of
Abramksy, Hyland, and Ong, et al. Without loss of generality, we can
think of \textcolor{red}{red} as player and black as opponent. A
well-formed red set, \textcolor{red}{$S$}, is a winning strategy for
player, while a well-formed black set, $S$, is a winning strategy for
opponent.


\section{Formal theory}
The judgment \(\Sigma; \Gamma \vdash S\) is pronounced
``\(\Sigma\) thinks that \(S\) is well-formed, given
dependencies, \(\Gamma\) .'' Similarly,
\(\Sigma; \Gamma \vdash x\) is pronounced
``\(\Sigma\) thinks that \(x\) is an admissible atom, given
dependencies, \(\Gamma\).''

We think of $X$ as some infinite supply of ``atoms'' and
$\mathsf{S}[X]$ as a \emph{stream} of atoms drawn from $X$. We assume
basic stream operations on $\mathsf{S}[X]$, such as
$\mathsf{take}(\mathsf{S}[X],n)$ which returns the pair
$(\sigma,\Sigma)$ where $\sigma = x_{1}:\ldots:x_{n}$ and $\Sigma = \mathsf{drop}(\mathsf{S}[X],n)$. Given $\rho = \{j_{1}/i_{1}:\ldots:j_{n}/i_{n}\}$ the operation
$\mathsf{swap}(\mathsf{S}[X],\rho)$ denotes the application of the
permutation $\rho$ to $\mathsf{S}[X]$.

Thus, $\mathsf{S}[X]$ may be thought of as an ordering on $X$.

A rule of the form

\[\frac{ H_1, \ldots , H_n }{ C }R\]

is pronounced ``\(R\) concludes that \(C\) given \(H_1\), \(\ldots\),
\(H_n\)''.

We use $e,f,\ldots$ to range over sets and atoms.

The rules for judging when an atom is admissible or set is well
formed are as follows

\[\frac{ }{ \mathsf{S}[X]; () \vdash \emptyset}Foundation\]

\[\frac{ (x,\Sigma) = \mathsf{take}(\mathsf{S}[X],1)}{ \Sigma; x \vdash \{ x \}}Atomicity\]

\[\frac{ \Sigma; \Gamma \vdash S }{ \Sigma; \Gamma \vdash \{ S \}}Nesting\]

\[\frac{ \Sigma; \Gamma_1 \vdash S_1 \; \Sigma; \Gamma_2 \vdash S_2}{ \Sigma; \Gamma_1, \Gamma_2 \vdash S_1 \cup S_2}Union\]

\[\frac{ \Sigma; \Gamma_1 \vdash S_1 \; \Sigma; \Gamma_2 \vdash S_2}{ \Sigma; \Gamma_1, \Gamma_2 \vdash S_1 \cap S_2}Intersection\]

\[\frac{ \Sigma; \Gamma_1 \vdash S_1 \; \Sigma; \Gamma_2 \vdash S_2}{ \Sigma; \Gamma_1, \Gamma_2 \vdash S_1 \setminus S_2}Subtraction\]

\[\frac{ \Sigma; \Gamma \vdash S \; \Sigma; \Gamma_{1} \vdash S_{1} \;\ldots\; \Sigma; \Gamma_{n} \vdash S_{2}\; \Sigma; \Delta \vdash f : S_{1} \times \ldots \times S_{n} \to S }{ \Sigma; \Gamma,\Gamma_{1},\ldots, \Gamma_{n},\Delta \vdash \{ f(s_{1},\ldots,s_{n}) \; : \; s_1 \in S_{1},\; \ldots\;, s_{n} \in S_{n}\}}Comprehension\]

\subsection{Equations}\label{equations}

The syntactic theory is too fine grained. It makes syntactic
distinctions that do not correspond to distinct sets. We erase these
syntactic distinctions with a set of equations on set expressions.

\[\frac{ \Sigma; \Gamma \vdash e }{ \Sigma; \Gamma  \vdash e = e}Identity\]
\[\frac{\Sigma; \Gamma_{1} \vdash _{1}\; \Sigma; \Gamma_{2} \vdash e_{2}\; \Sigma; \Gamma_{3}  \vdash e_{1} = e_{2}}{ \Sigma; \Gamma_{3}  \vdash e_{1} = e_{2}}Symmetry\]
\[\frac{\Sigma; \Gamma_{1}  \vdash e_{1}\; \Sigma; \Gamma_{2}  \vdash e_{2}\; \Sigma; \Gamma_{3} \vdash e_{3}\; \Sigma; \Gamma_{4}  \vdash e_{1} = e_{2} \; \Sigma; \Gamma_{5}  \vdash e_{2} = e_{3}}{ \Sigma; \Gamma_{4},\Gamma_{5}  \vdash e_{1} = e_{3}}Transitivity\]

\[\frac{\Sigma; \Gamma  \vdash S}{\Sigma; \Gamma \vdash S \cup S =  S}Idempotence_{\cup}\]
\[\frac{\Sigma; \Gamma  \vdash S}{\Sigma; \Gamma \vdash S \cap S =  S}Idempotence_{\cap}\]




\documentclass[12pt]{llncs}

\usepackage[pdftex]{hyperref}                   
\usepackage {mathpartir}
\usepackage{bigpage}
\usepackage{bcprules}
\usepackage {listings}
\usepackage{amsmath}
\usepackage{amssymb}
\usepackage{amsfonts}
\usepackage{amstext}
\usepackage{latexsym}
\usepackage{longtable}
\usepackage{stmaryrd}
%\usepackage{fdsymbol}
\usepackage{graphicx} 
%\usepackage[margins=2.5cm,nohead,nofoot]{geometry}
%\usepackage{geometry}
\usepackage{color}
\usepackage{xcolor}


%\include{myPreamble}
\include{rset.local} 

%\ifpdf
%\usepackage[pdftex]{graphicx}
%\else
%\usepackage{graphicx}
%\fi

 % \ifpdf
%  \usepackage{pdfsync}
%  \if


%\title{Brief Article}
%\author{David F. Snyder}
%\author{L.G. Meredith}

%\address{Dept. of Math., Texas State University--San Marcos, San Marcos, TX 78666}
       
\pagestyle{empty}


\begin{document}

\lstset{language=[Objective]Caml,frame=shadowbox}


\input{rset.front}

In the early 2000's Pitts and Gabbay's use of Fraenkl-Mostowski set
theory (FM set theory), a theory of sets with atoms, to model nominal phenomena,
sparked a vibrant line of research. \cite{DBLP:journals/fac/GabbayP02}
\cite{DBLP:journals/jcss/Clouston14}. As Chaitin points out, the
axioms and assumptions of a theory comprise its risk
\cite{chaitin1999unknowable}. This observation raises an interesting
point: is it possible to have a set theory with atoms without taking
on \emph{infinite} risk? It turns out it is.

We describe a theory of sets that enjoys both the cleanliness of
$\mathsf{ZF}$, that is arising only from the empty set, and yet with
an infinite supply of atoms.

\input{rset.intuitions}

\input{rset.presentation}

\input{rset.stdfmset}

\input{rset.mainthm}

\input{rset.conc}

\input{rset.ack}

\bibliographystyle{plain}   
\bibliography{rset.bib}

\end{document}


\documentclass[12pt]{llncs}

\usepackage[pdftex]{hyperref}                   
\usepackage {mathpartir}
\usepackage{bigpage}
\usepackage{bcprules}
\usepackage {listings}
\usepackage{amsmath}
\usepackage{amssymb}
\usepackage{amsfonts}
\usepackage{amstext}
\usepackage{latexsym}
\usepackage{longtable}
\usepackage{stmaryrd}
%\usepackage{fdsymbol}
\usepackage{graphicx} 
%\usepackage[margins=2.5cm,nohead,nofoot]{geometry}
%\usepackage{geometry}
\usepackage{color}
\usepackage{xcolor}


%\include{myPreamble}
\include{rset.local} 

%\ifpdf
%\usepackage[pdftex]{graphicx}
%\else
%\usepackage{graphicx}
%\fi

 % \ifpdf
%  \usepackage{pdfsync}
%  \if


%\title{Brief Article}
%\author{David F. Snyder}
%\author{L.G. Meredith}

%\address{Dept. of Math., Texas State University--San Marcos, San Marcos, TX 78666}
       
\pagestyle{empty}


\begin{document}

\lstset{language=[Objective]Caml,frame=shadowbox}


\input{rset.front}

In the early 2000's Pitts and Gabbay's use of Fraenkl-Mostowski set
theory (FM set theory), a theory of sets with atoms, to model nominal phenomena,
sparked a vibrant line of research. \cite{DBLP:journals/fac/GabbayP02}
\cite{DBLP:journals/jcss/Clouston14}. As Chaitin points out, the
axioms and assumptions of a theory comprise its risk
\cite{chaitin1999unknowable}. This observation raises an interesting
point: is it possible to have a set theory with atoms without taking
on \emph{infinite} risk? It turns out it is.

We describe a theory of sets that enjoys both the cleanliness of
$\mathsf{ZF}$, that is arising only from the empty set, and yet with
an infinite supply of atoms.

\input{rset.intuitions}

\input{rset.presentation}

\input{rset.stdfmset}

\input{rset.mainthm}

\input{rset.conc}

\input{rset.ack}

\bibliographystyle{plain}   
\bibliography{rset.bib}

\end{document}


\section{Conclusions and future work}

We have presented a formal theory of sets that reconciles set theory with atoms with a set theory built only from the empty set.

% subsection other_calculi_other_bisimulations_and_geometry_as_behavior (end)




\paragraph{Acknowledgments.}
The author wishes to thank Jamie Gabbay for stimulating discussions
that led the author to devise this theory.


\bibliographystyle{plain}   
\bibliography{rset.bib}

\end{document}


\section{Conclusions and future work}

We have presented a formal theory of sets that reconciles set theory with atoms with a set theory built only from the empty set.

% subsection other_calculi_other_bisimulations_and_geometry_as_behavior (end)




\paragraph{Acknowledgments.}
The author wishes to thank Jamie Gabbay for stimulating discussions
that led the author to devise this theory.


\bibliographystyle{plain}   
\bibliography{rset.bib}

\end{document}
 

%\ifpdf
%\usepackage[pdftex]{graphicx}
%\else
%\usepackage{graphicx}
%\fi

 % \ifpdf
%  \usepackage{pdfsync}
%  \if


%\title{Brief Article}
%\author{David F. Snyder}
%\author{L.G. Meredith}

%\address{Dept. of Math., Texas State University--San Marcos, San Marcos, TX 78666}
       
\pagestyle{empty}


\begin{document}

\lstset{language=[Objective]Caml,frame=shadowbox}


\def\lastname{Meredith}

\title{Notes on a reflective theory of sets}
\titlerunning{oslf}

\author{ Lucius Gregory Meredith\inst{1}
}
\institute{
        {Managing Partner, RTech Unchained 9336 California Ave SW, Seattle, WA 98103, USA} \\
  \email{ lgreg.meredith@gmail.com } 
}
 

\maketitle              % typeset the title of the contribution

%%% ----------------------------------------------------------------------

\begin{abstract}

  We describe a universe of set theories enjoying atoms without enduring infinite risk.

\end{abstract}

%\keywords{Process calculi, knots, invariants}

% \begin{keyword}
% concurrency, message-passing, process calculus, reflection, program logic
% \end{keyword}

%\end{frontmatter}


In the early 2000's Pitts and Gabbay's use of Fraenkl-Mostowski set
theory (FM set theory), a theory of sets with atoms, to model nominal phenomena,
sparked a vibrant line of research. \cite{DBLP:journals/fac/GabbayP02}
\cite{DBLP:journals/jcss/Clouston14}. As Chaitin points out, the
axioms and assumptions of a theory comprise its risk
\cite{chaitin1999unknowable}. This observation raises an interesting
point: is it possible to have a set theory with atoms without taking
on \emph{infinite} risk? It turns out it is.

We describe a theory of sets that enjoys both the cleanliness of
$\mathsf{ZF}$, that is arising only from the empty set, and yet with
an infinite supply of atoms.

\section{Intuitions}
We imagine two copies of FM set theory, say a \textcolor{red}{red one}
and a black one. The \textcolor{red}{red theory} has red
comprehensions,\textcolor{red}{$\{ \ldots \}$} and a red element-of
relation, \textcolor{red}{$\in$}, while the black theory,
correspondingly has black comprehensions, $\{ \ldots \}$ and a black
element-of relation, $\in$. The red element-of relation,
\textcolor{red}{$\in$}, can ``see into'' red sets, \textcolor{red}{$\{
  \ldots \}$}, but not black ones. Meanwhile, the black element-of
relation, $\in$, can ``see into'' black sets $\{ \ldots \}$, but not
red ones. Thus, the black sets can serve as atoms for the red sets,
while the red sets can serve as atoms for the black sets.

So, the two theories are mutually recursively defined. This mutual
recursion is closely linked to game semantics in the sense of
Abramksy, Hyland, and Ong, et al. Without loss of generality, we can
think of \textcolor{red}{red} as player and black as opponent. A
well-formed red set, \textcolor{red}{$S$}, is a winning strategy for
player, while a well-formed black set, $S$, is a winning strategy for
opponent.


\section{Formal theory}
The judgment \(\Sigma; \Gamma \vdash S\) is pronounced
``\(\Sigma\) thinks that \(S\) is well-formed, given
dependencies, \(\Gamma\) .'' Similarly,
\(\Sigma; \Gamma \vdash x\) is pronounced
``\(\Sigma\) thinks that \(x\) is an admissible atom, given
dependencies, \(\Gamma\).''

We think of $X$ as some infinite supply of ``atoms'' and
$\mathsf{S}[X]$ as a \emph{stream} of atoms drawn from $X$. We assume
basic stream operations on $\mathsf{S}[X]$, such as
$\mathsf{take}(\mathsf{S}[X],n)$ which returns the pair
$(\sigma,\Sigma)$ where $\sigma = x_{1}:\ldots:x_{n}$ and $\Sigma = \mathsf{drop}(\mathsf{S}[X],n)$. Given $\rho = \{j_{1}/i_{1}:\ldots:j_{n}/i_{n}\}$ the operation
$\mathsf{swap}(\mathsf{S}[X],\rho)$ denotes the application of the
permutation $\rho$ to $\mathsf{S}[X]$.

Thus, $\mathsf{S}[X]$ may be thought of as an ordering on $X$.

A rule of the form

\[\frac{ H_1, \ldots , H_n }{ C }R\]

is pronounced ``\(R\) concludes that \(C\) given \(H_1\), \(\ldots\),
\(H_n\)''.

We use $e,f,\ldots$ to range over sets and atoms.

The rules for judging when an atom is admissible or set is well
formed are as follows

\[\frac{ }{ \mathsf{S}[X]; () \vdash \emptyset}Foundation\]

\[\frac{ (x,\Sigma) = \mathsf{take}(\mathsf{S}[X],1)}{ \Sigma; x \vdash \{ x \}}Atomicity\]

\[\frac{ \Sigma; \Gamma \vdash S }{ \Sigma; \Gamma \vdash \{ S \}}Nesting\]

\[\frac{ \Sigma; \Gamma_1 \vdash S_1 \; \Sigma; \Gamma_2 \vdash S_2}{ \Sigma; \Gamma_1, \Gamma_2 \vdash S_1 \cup S_2}Union\]

\[\frac{ \Sigma; \Gamma_1 \vdash S_1 \; \Sigma; \Gamma_2 \vdash S_2}{ \Sigma; \Gamma_1, \Gamma_2 \vdash S_1 \cap S_2}Intersection\]

\[\frac{ \Sigma; \Gamma_1 \vdash S_1 \; \Sigma; \Gamma_2 \vdash S_2}{ \Sigma; \Gamma_1, \Gamma_2 \vdash S_1 \setminus S_2}Subtraction\]

\[\frac{ \Sigma; \Gamma \vdash S \; \Sigma; \Gamma_{1} \vdash S_{1} \;\ldots\; \Sigma; \Gamma_{n} \vdash S_{2}\; \Sigma; \Delta \vdash f : S_{1} \times \ldots \times S_{n} \to S }{ \Sigma; \Gamma,\Gamma_{1},\ldots, \Gamma_{n},\Delta \vdash \{ f(s_{1},\ldots,s_{n}) \; : \; s_1 \in S_{1},\; \ldots\;, s_{n} \in S_{n}\}}Comprehension\]

\subsection{Equations}\label{equations}

The syntactic theory is too fine grained. It makes syntactic
distinctions that do not correspond to distinct sets. We erase these
syntactic distinctions with a set of equations on set expressions.

\[\frac{ \Sigma; \Gamma \vdash e }{ \Sigma; \Gamma  \vdash e = e}Identity\]
\[\frac{\Sigma; \Gamma_{1} \vdash _{1}\; \Sigma; \Gamma_{2} \vdash e_{2}\; \Sigma; \Gamma_{3}  \vdash e_{1} = e_{2}}{ \Sigma; \Gamma_{3}  \vdash e_{1} = e_{2}}Symmetry\]
\[\frac{\Sigma; \Gamma_{1}  \vdash e_{1}\; \Sigma; \Gamma_{2}  \vdash e_{2}\; \Sigma; \Gamma_{3} \vdash e_{3}\; \Sigma; \Gamma_{4}  \vdash e_{1} = e_{2} \; \Sigma; \Gamma_{5}  \vdash e_{2} = e_{3}}{ \Sigma; \Gamma_{4},\Gamma_{5}  \vdash e_{1} = e_{3}}Transitivity\]

\[\frac{\Sigma; \Gamma  \vdash S}{\Sigma; \Gamma \vdash S \cup S =  S}Idempotence_{\cup}\]
\[\frac{\Sigma; \Gamma  \vdash S}{\Sigma; \Gamma \vdash S \cap S =  S}Idempotence_{\cap}\]




\documentclass[12pt]{llncs}

\usepackage[pdftex]{hyperref}                   
\usepackage {mathpartir}
\usepackage{bigpage}
\usepackage{bcprules}
\usepackage {listings}
\usepackage{amsmath}
\usepackage{amssymb}
\usepackage{amsfonts}
\usepackage{amstext}
\usepackage{latexsym}
\usepackage{longtable}
\usepackage{stmaryrd}
%\usepackage{fdsymbol}
\usepackage{graphicx} 
%\usepackage[margins=2.5cm,nohead,nofoot]{geometry}
%\usepackage{geometry}
\usepackage{color}
\usepackage{xcolor}


%\include{myPreamble}
\documentclass[12pt]{llncs}

\usepackage[pdftex]{hyperref}                   
\usepackage {mathpartir}
\usepackage{bigpage}
\usepackage{bcprules}
\usepackage {listings}
\usepackage{amsmath}
\usepackage{amssymb}
\usepackage{amsfonts}
\usepackage{amstext}
\usepackage{latexsym}
\usepackage{longtable}
\usepackage{stmaryrd}
%\usepackage{fdsymbol}
\usepackage{graphicx} 
%\usepackage[margins=2.5cm,nohead,nofoot]{geometry}
%\usepackage{geometry}
\usepackage{color}
\usepackage{xcolor}


%\include{myPreamble}
\documentclass[12pt]{llncs}

\usepackage[pdftex]{hyperref}                   
\usepackage {mathpartir}
\usepackage{bigpage}
\usepackage{bcprules}
\usepackage {listings}
\usepackage{amsmath}
\usepackage{amssymb}
\usepackage{amsfonts}
\usepackage{amstext}
\usepackage{latexsym}
\usepackage{longtable}
\usepackage{stmaryrd}
%\usepackage{fdsymbol}
\usepackage{graphicx} 
%\usepackage[margins=2.5cm,nohead,nofoot]{geometry}
%\usepackage{geometry}
\usepackage{color}
\usepackage{xcolor}


%\include{myPreamble}
\include{rset.local} 

%\ifpdf
%\usepackage[pdftex]{graphicx}
%\else
%\usepackage{graphicx}
%\fi

 % \ifpdf
%  \usepackage{pdfsync}
%  \if


%\title{Brief Article}
%\author{David F. Snyder}
%\author{L.G. Meredith}

%\address{Dept. of Math., Texas State University--San Marcos, San Marcos, TX 78666}
       
\pagestyle{empty}


\begin{document}

\lstset{language=[Objective]Caml,frame=shadowbox}


\input{rset.front}

In the early 2000's Pitts and Gabbay's use of Fraenkl-Mostowski set
theory (FM set theory), a theory of sets with atoms, to model nominal phenomena,
sparked a vibrant line of research. \cite{DBLP:journals/fac/GabbayP02}
\cite{DBLP:journals/jcss/Clouston14}. As Chaitin points out, the
axioms and assumptions of a theory comprise its risk
\cite{chaitin1999unknowable}. This observation raises an interesting
point: is it possible to have a set theory with atoms without taking
on \emph{infinite} risk? It turns out it is.

We describe a theory of sets that enjoys both the cleanliness of
$\mathsf{ZF}$, that is arising only from the empty set, and yet with
an infinite supply of atoms.

\input{rset.intuitions}

\input{rset.presentation}

\input{rset.stdfmset}

\input{rset.mainthm}

\input{rset.conc}

\input{rset.ack}

\bibliographystyle{plain}   
\bibliography{rset.bib}

\end{document}
 

%\ifpdf
%\usepackage[pdftex]{graphicx}
%\else
%\usepackage{graphicx}
%\fi

 % \ifpdf
%  \usepackage{pdfsync}
%  \if


%\title{Brief Article}
%\author{David F. Snyder}
%\author{L.G. Meredith}

%\address{Dept. of Math., Texas State University--San Marcos, San Marcos, TX 78666}
       
\pagestyle{empty}


\begin{document}

\lstset{language=[Objective]Caml,frame=shadowbox}


\def\lastname{Meredith}

\title{Notes on a reflective theory of sets}
\titlerunning{oslf}

\author{ Lucius Gregory Meredith\inst{1}
}
\institute{
        {Managing Partner, RTech Unchained 9336 California Ave SW, Seattle, WA 98103, USA} \\
  \email{ lgreg.meredith@gmail.com } 
}
 

\maketitle              % typeset the title of the contribution

%%% ----------------------------------------------------------------------

\begin{abstract}

  We describe a universe of set theories enjoying atoms without enduring infinite risk.

\end{abstract}

%\keywords{Process calculi, knots, invariants}

% \begin{keyword}
% concurrency, message-passing, process calculus, reflection, program logic
% \end{keyword}

%\end{frontmatter}


In the early 2000's Pitts and Gabbay's use of Fraenkl-Mostowski set
theory (FM set theory), a theory of sets with atoms, to model nominal phenomena,
sparked a vibrant line of research. \cite{DBLP:journals/fac/GabbayP02}
\cite{DBLP:journals/jcss/Clouston14}. As Chaitin points out, the
axioms and assumptions of a theory comprise its risk
\cite{chaitin1999unknowable}. This observation raises an interesting
point: is it possible to have a set theory with atoms without taking
on \emph{infinite} risk? It turns out it is.

We describe a theory of sets that enjoys both the cleanliness of
$\mathsf{ZF}$, that is arising only from the empty set, and yet with
an infinite supply of atoms.

\section{Intuitions}
We imagine two copies of FM set theory, say a \textcolor{red}{red one}
and a black one. The \textcolor{red}{red theory} has red
comprehensions,\textcolor{red}{$\{ \ldots \}$} and a red element-of
relation, \textcolor{red}{$\in$}, while the black theory,
correspondingly has black comprehensions, $\{ \ldots \}$ and a black
element-of relation, $\in$. The red element-of relation,
\textcolor{red}{$\in$}, can ``see into'' red sets, \textcolor{red}{$\{
  \ldots \}$}, but not black ones. Meanwhile, the black element-of
relation, $\in$, can ``see into'' black sets $\{ \ldots \}$, but not
red ones. Thus, the black sets can serve as atoms for the red sets,
while the red sets can serve as atoms for the black sets.

So, the two theories are mutually recursively defined. This mutual
recursion is closely linked to game semantics in the sense of
Abramksy, Hyland, and Ong, et al. Without loss of generality, we can
think of \textcolor{red}{red} as player and black as opponent. A
well-formed red set, \textcolor{red}{$S$}, is a winning strategy for
player, while a well-formed black set, $S$, is a winning strategy for
opponent.


\section{Formal theory}
The judgment \(\Sigma; \Gamma \vdash S\) is pronounced
``\(\Sigma\) thinks that \(S\) is well-formed, given
dependencies, \(\Gamma\) .'' Similarly,
\(\Sigma; \Gamma \vdash x\) is pronounced
``\(\Sigma\) thinks that \(x\) is an admissible atom, given
dependencies, \(\Gamma\).''

We think of $X$ as some infinite supply of ``atoms'' and
$\mathsf{S}[X]$ as a \emph{stream} of atoms drawn from $X$. We assume
basic stream operations on $\mathsf{S}[X]$, such as
$\mathsf{take}(\mathsf{S}[X],n)$ which returns the pair
$(\sigma,\Sigma)$ where $\sigma = x_{1}:\ldots:x_{n}$ and $\Sigma = \mathsf{drop}(\mathsf{S}[X],n)$. Given $\rho = \{j_{1}/i_{1}:\ldots:j_{n}/i_{n}\}$ the operation
$\mathsf{swap}(\mathsf{S}[X],\rho)$ denotes the application of the
permutation $\rho$ to $\mathsf{S}[X]$.

Thus, $\mathsf{S}[X]$ may be thought of as an ordering on $X$.

A rule of the form

\[\frac{ H_1, \ldots , H_n }{ C }R\]

is pronounced ``\(R\) concludes that \(C\) given \(H_1\), \(\ldots\),
\(H_n\)''.

We use $e,f,\ldots$ to range over sets and atoms.

The rules for judging when an atom is admissible or set is well
formed are as follows

\[\frac{ }{ \mathsf{S}[X]; () \vdash \emptyset}Foundation\]

\[\frac{ (x,\Sigma) = \mathsf{take}(\mathsf{S}[X],1)}{ \Sigma; x \vdash \{ x \}}Atomicity\]

\[\frac{ \Sigma; \Gamma \vdash S }{ \Sigma; \Gamma \vdash \{ S \}}Nesting\]

\[\frac{ \Sigma; \Gamma_1 \vdash S_1 \; \Sigma; \Gamma_2 \vdash S_2}{ \Sigma; \Gamma_1, \Gamma_2 \vdash S_1 \cup S_2}Union\]

\[\frac{ \Sigma; \Gamma_1 \vdash S_1 \; \Sigma; \Gamma_2 \vdash S_2}{ \Sigma; \Gamma_1, \Gamma_2 \vdash S_1 \cap S_2}Intersection\]

\[\frac{ \Sigma; \Gamma_1 \vdash S_1 \; \Sigma; \Gamma_2 \vdash S_2}{ \Sigma; \Gamma_1, \Gamma_2 \vdash S_1 \setminus S_2}Subtraction\]

\[\frac{ \Sigma; \Gamma \vdash S \; \Sigma; \Gamma_{1} \vdash S_{1} \;\ldots\; \Sigma; \Gamma_{n} \vdash S_{2}\; \Sigma; \Delta \vdash f : S_{1} \times \ldots \times S_{n} \to S }{ \Sigma; \Gamma,\Gamma_{1},\ldots, \Gamma_{n},\Delta \vdash \{ f(s_{1},\ldots,s_{n}) \; : \; s_1 \in S_{1},\; \ldots\;, s_{n} \in S_{n}\}}Comprehension\]

\subsection{Equations}\label{equations}

The syntactic theory is too fine grained. It makes syntactic
distinctions that do not correspond to distinct sets. We erase these
syntactic distinctions with a set of equations on set expressions.

\[\frac{ \Sigma; \Gamma \vdash e }{ \Sigma; \Gamma  \vdash e = e}Identity\]
\[\frac{\Sigma; \Gamma_{1} \vdash _{1}\; \Sigma; \Gamma_{2} \vdash e_{2}\; \Sigma; \Gamma_{3}  \vdash e_{1} = e_{2}}{ \Sigma; \Gamma_{3}  \vdash e_{1} = e_{2}}Symmetry\]
\[\frac{\Sigma; \Gamma_{1}  \vdash e_{1}\; \Sigma; \Gamma_{2}  \vdash e_{2}\; \Sigma; \Gamma_{3} \vdash e_{3}\; \Sigma; \Gamma_{4}  \vdash e_{1} = e_{2} \; \Sigma; \Gamma_{5}  \vdash e_{2} = e_{3}}{ \Sigma; \Gamma_{4},\Gamma_{5}  \vdash e_{1} = e_{3}}Transitivity\]

\[\frac{\Sigma; \Gamma  \vdash S}{\Sigma; \Gamma \vdash S \cup S =  S}Idempotence_{\cup}\]
\[\frac{\Sigma; \Gamma  \vdash S}{\Sigma; \Gamma \vdash S \cap S =  S}Idempotence_{\cap}\]




\documentclass[12pt]{llncs}

\usepackage[pdftex]{hyperref}                   
\usepackage {mathpartir}
\usepackage{bigpage}
\usepackage{bcprules}
\usepackage {listings}
\usepackage{amsmath}
\usepackage{amssymb}
\usepackage{amsfonts}
\usepackage{amstext}
\usepackage{latexsym}
\usepackage{longtable}
\usepackage{stmaryrd}
%\usepackage{fdsymbol}
\usepackage{graphicx} 
%\usepackage[margins=2.5cm,nohead,nofoot]{geometry}
%\usepackage{geometry}
\usepackage{color}
\usepackage{xcolor}


%\include{myPreamble}
\include{rset.local} 

%\ifpdf
%\usepackage[pdftex]{graphicx}
%\else
%\usepackage{graphicx}
%\fi

 % \ifpdf
%  \usepackage{pdfsync}
%  \if


%\title{Brief Article}
%\author{David F. Snyder}
%\author{L.G. Meredith}

%\address{Dept. of Math., Texas State University--San Marcos, San Marcos, TX 78666}
       
\pagestyle{empty}


\begin{document}

\lstset{language=[Objective]Caml,frame=shadowbox}


\input{rset.front}

In the early 2000's Pitts and Gabbay's use of Fraenkl-Mostowski set
theory (FM set theory), a theory of sets with atoms, to model nominal phenomena,
sparked a vibrant line of research. \cite{DBLP:journals/fac/GabbayP02}
\cite{DBLP:journals/jcss/Clouston14}. As Chaitin points out, the
axioms and assumptions of a theory comprise its risk
\cite{chaitin1999unknowable}. This observation raises an interesting
point: is it possible to have a set theory with atoms without taking
on \emph{infinite} risk? It turns out it is.

We describe a theory of sets that enjoys both the cleanliness of
$\mathsf{ZF}$, that is arising only from the empty set, and yet with
an infinite supply of atoms.

\input{rset.intuitions}

\input{rset.presentation}

\input{rset.stdfmset}

\input{rset.mainthm}

\input{rset.conc}

\input{rset.ack}

\bibliographystyle{plain}   
\bibliography{rset.bib}

\end{document}


\documentclass[12pt]{llncs}

\usepackage[pdftex]{hyperref}                   
\usepackage {mathpartir}
\usepackage{bigpage}
\usepackage{bcprules}
\usepackage {listings}
\usepackage{amsmath}
\usepackage{amssymb}
\usepackage{amsfonts}
\usepackage{amstext}
\usepackage{latexsym}
\usepackage{longtable}
\usepackage{stmaryrd}
%\usepackage{fdsymbol}
\usepackage{graphicx} 
%\usepackage[margins=2.5cm,nohead,nofoot]{geometry}
%\usepackage{geometry}
\usepackage{color}
\usepackage{xcolor}


%\include{myPreamble}
\include{rset.local} 

%\ifpdf
%\usepackage[pdftex]{graphicx}
%\else
%\usepackage{graphicx}
%\fi

 % \ifpdf
%  \usepackage{pdfsync}
%  \if


%\title{Brief Article}
%\author{David F. Snyder}
%\author{L.G. Meredith}

%\address{Dept. of Math., Texas State University--San Marcos, San Marcos, TX 78666}
       
\pagestyle{empty}


\begin{document}

\lstset{language=[Objective]Caml,frame=shadowbox}


\input{rset.front}

In the early 2000's Pitts and Gabbay's use of Fraenkl-Mostowski set
theory (FM set theory), a theory of sets with atoms, to model nominal phenomena,
sparked a vibrant line of research. \cite{DBLP:journals/fac/GabbayP02}
\cite{DBLP:journals/jcss/Clouston14}. As Chaitin points out, the
axioms and assumptions of a theory comprise its risk
\cite{chaitin1999unknowable}. This observation raises an interesting
point: is it possible to have a set theory with atoms without taking
on \emph{infinite} risk? It turns out it is.

We describe a theory of sets that enjoys both the cleanliness of
$\mathsf{ZF}$, that is arising only from the empty set, and yet with
an infinite supply of atoms.

\input{rset.intuitions}

\input{rset.presentation}

\input{rset.stdfmset}

\input{rset.mainthm}

\input{rset.conc}

\input{rset.ack}

\bibliographystyle{plain}   
\bibliography{rset.bib}

\end{document}


\section{Conclusions and future work}

We have presented a formal theory of sets that reconciles set theory with atoms with a set theory built only from the empty set.

% subsection other_calculi_other_bisimulations_and_geometry_as_behavior (end)




\paragraph{Acknowledgments.}
The author wishes to thank Jamie Gabbay for stimulating discussions
that led the author to devise this theory.


\bibliographystyle{plain}   
\bibliography{rset.bib}

\end{document}
 

%\ifpdf
%\usepackage[pdftex]{graphicx}
%\else
%\usepackage{graphicx}
%\fi

 % \ifpdf
%  \usepackage{pdfsync}
%  \if


%\title{Brief Article}
%\author{David F. Snyder}
%\author{L.G. Meredith}

%\address{Dept. of Math., Texas State University--San Marcos, San Marcos, TX 78666}
       
\pagestyle{empty}


\begin{document}

\lstset{language=[Objective]Caml,frame=shadowbox}


\def\lastname{Meredith}

\title{Notes on a reflective theory of sets}
\titlerunning{oslf}

\author{ Lucius Gregory Meredith\inst{1}
}
\institute{
        {Managing Partner, RTech Unchained 9336 California Ave SW, Seattle, WA 98103, USA} \\
  \email{ lgreg.meredith@gmail.com } 
}
 

\maketitle              % typeset the title of the contribution

%%% ----------------------------------------------------------------------

\begin{abstract}

  We describe a universe of set theories enjoying atoms without enduring infinite risk.

\end{abstract}

%\keywords{Process calculi, knots, invariants}

% \begin{keyword}
% concurrency, message-passing, process calculus, reflection, program logic
% \end{keyword}

%\end{frontmatter}


In the early 2000's Pitts and Gabbay's use of Fraenkl-Mostowski set
theory (FM set theory), a theory of sets with atoms, to model nominal phenomena,
sparked a vibrant line of research. \cite{DBLP:journals/fac/GabbayP02}
\cite{DBLP:journals/jcss/Clouston14}. As Chaitin points out, the
axioms and assumptions of a theory comprise its risk
\cite{chaitin1999unknowable}. This observation raises an interesting
point: is it possible to have a set theory with atoms without taking
on \emph{infinite} risk? It turns out it is.

We describe a theory of sets that enjoys both the cleanliness of
$\mathsf{ZF}$, that is arising only from the empty set, and yet with
an infinite supply of atoms.

\section{Intuitions}
We imagine two copies of FM set theory, say a \textcolor{red}{red one}
and a black one. The \textcolor{red}{red theory} has red
comprehensions,\textcolor{red}{$\{ \ldots \}$} and a red element-of
relation, \textcolor{red}{$\in$}, while the black theory,
correspondingly has black comprehensions, $\{ \ldots \}$ and a black
element-of relation, $\in$. The red element-of relation,
\textcolor{red}{$\in$}, can ``see into'' red sets, \textcolor{red}{$\{
  \ldots \}$}, but not black ones. Meanwhile, the black element-of
relation, $\in$, can ``see into'' black sets $\{ \ldots \}$, but not
red ones. Thus, the black sets can serve as atoms for the red sets,
while the red sets can serve as atoms for the black sets.

So, the two theories are mutually recursively defined. This mutual
recursion is closely linked to game semantics in the sense of
Abramksy, Hyland, and Ong, et al. Without loss of generality, we can
think of \textcolor{red}{red} as player and black as opponent. A
well-formed red set, \textcolor{red}{$S$}, is a winning strategy for
player, while a well-formed black set, $S$, is a winning strategy for
opponent.


\section{Formal theory}
The judgment \(\Sigma; \Gamma \vdash S\) is pronounced
``\(\Sigma\) thinks that \(S\) is well-formed, given
dependencies, \(\Gamma\) .'' Similarly,
\(\Sigma; \Gamma \vdash x\) is pronounced
``\(\Sigma\) thinks that \(x\) is an admissible atom, given
dependencies, \(\Gamma\).''

We think of $X$ as some infinite supply of ``atoms'' and
$\mathsf{S}[X]$ as a \emph{stream} of atoms drawn from $X$. We assume
basic stream operations on $\mathsf{S}[X]$, such as
$\mathsf{take}(\mathsf{S}[X],n)$ which returns the pair
$(\sigma,\Sigma)$ where $\sigma = x_{1}:\ldots:x_{n}$ and $\Sigma = \mathsf{drop}(\mathsf{S}[X],n)$. Given $\rho = \{j_{1}/i_{1}:\ldots:j_{n}/i_{n}\}$ the operation
$\mathsf{swap}(\mathsf{S}[X],\rho)$ denotes the application of the
permutation $\rho$ to $\mathsf{S}[X]$.

Thus, $\mathsf{S}[X]$ may be thought of as an ordering on $X$.

A rule of the form

\[\frac{ H_1, \ldots , H_n }{ C }R\]

is pronounced ``\(R\) concludes that \(C\) given \(H_1\), \(\ldots\),
\(H_n\)''.

We use $e,f,\ldots$ to range over sets and atoms.

The rules for judging when an atom is admissible or set is well
formed are as follows

\[\frac{ }{ \mathsf{S}[X]; () \vdash \emptyset}Foundation\]

\[\frac{ (x,\Sigma) = \mathsf{take}(\mathsf{S}[X],1)}{ \Sigma; x \vdash \{ x \}}Atomicity\]

\[\frac{ \Sigma; \Gamma \vdash S }{ \Sigma; \Gamma \vdash \{ S \}}Nesting\]

\[\frac{ \Sigma; \Gamma_1 \vdash S_1 \; \Sigma; \Gamma_2 \vdash S_2}{ \Sigma; \Gamma_1, \Gamma_2 \vdash S_1 \cup S_2}Union\]

\[\frac{ \Sigma; \Gamma_1 \vdash S_1 \; \Sigma; \Gamma_2 \vdash S_2}{ \Sigma; \Gamma_1, \Gamma_2 \vdash S_1 \cap S_2}Intersection\]

\[\frac{ \Sigma; \Gamma_1 \vdash S_1 \; \Sigma; \Gamma_2 \vdash S_2}{ \Sigma; \Gamma_1, \Gamma_2 \vdash S_1 \setminus S_2}Subtraction\]

\[\frac{ \Sigma; \Gamma \vdash S \; \Sigma; \Gamma_{1} \vdash S_{1} \;\ldots\; \Sigma; \Gamma_{n} \vdash S_{2}\; \Sigma; \Delta \vdash f : S_{1} \times \ldots \times S_{n} \to S }{ \Sigma; \Gamma,\Gamma_{1},\ldots, \Gamma_{n},\Delta \vdash \{ f(s_{1},\ldots,s_{n}) \; : \; s_1 \in S_{1},\; \ldots\;, s_{n} \in S_{n}\}}Comprehension\]

\subsection{Equations}\label{equations}

The syntactic theory is too fine grained. It makes syntactic
distinctions that do not correspond to distinct sets. We erase these
syntactic distinctions with a set of equations on set expressions.

\[\frac{ \Sigma; \Gamma \vdash e }{ \Sigma; \Gamma  \vdash e = e}Identity\]
\[\frac{\Sigma; \Gamma_{1} \vdash _{1}\; \Sigma; \Gamma_{2} \vdash e_{2}\; \Sigma; \Gamma_{3}  \vdash e_{1} = e_{2}}{ \Sigma; \Gamma_{3}  \vdash e_{1} = e_{2}}Symmetry\]
\[\frac{\Sigma; \Gamma_{1}  \vdash e_{1}\; \Sigma; \Gamma_{2}  \vdash e_{2}\; \Sigma; \Gamma_{3} \vdash e_{3}\; \Sigma; \Gamma_{4}  \vdash e_{1} = e_{2} \; \Sigma; \Gamma_{5}  \vdash e_{2} = e_{3}}{ \Sigma; \Gamma_{4},\Gamma_{5}  \vdash e_{1} = e_{3}}Transitivity\]

\[\frac{\Sigma; \Gamma  \vdash S}{\Sigma; \Gamma \vdash S \cup S =  S}Idempotence_{\cup}\]
\[\frac{\Sigma; \Gamma  \vdash S}{\Sigma; \Gamma \vdash S \cap S =  S}Idempotence_{\cap}\]




\documentclass[12pt]{llncs}

\usepackage[pdftex]{hyperref}                   
\usepackage {mathpartir}
\usepackage{bigpage}
\usepackage{bcprules}
\usepackage {listings}
\usepackage{amsmath}
\usepackage{amssymb}
\usepackage{amsfonts}
\usepackage{amstext}
\usepackage{latexsym}
\usepackage{longtable}
\usepackage{stmaryrd}
%\usepackage{fdsymbol}
\usepackage{graphicx} 
%\usepackage[margins=2.5cm,nohead,nofoot]{geometry}
%\usepackage{geometry}
\usepackage{color}
\usepackage{xcolor}


%\include{myPreamble}
\documentclass[12pt]{llncs}

\usepackage[pdftex]{hyperref}                   
\usepackage {mathpartir}
\usepackage{bigpage}
\usepackage{bcprules}
\usepackage {listings}
\usepackage{amsmath}
\usepackage{amssymb}
\usepackage{amsfonts}
\usepackage{amstext}
\usepackage{latexsym}
\usepackage{longtable}
\usepackage{stmaryrd}
%\usepackage{fdsymbol}
\usepackage{graphicx} 
%\usepackage[margins=2.5cm,nohead,nofoot]{geometry}
%\usepackage{geometry}
\usepackage{color}
\usepackage{xcolor}


%\include{myPreamble}
\include{rset.local} 

%\ifpdf
%\usepackage[pdftex]{graphicx}
%\else
%\usepackage{graphicx}
%\fi

 % \ifpdf
%  \usepackage{pdfsync}
%  \if


%\title{Brief Article}
%\author{David F. Snyder}
%\author{L.G. Meredith}

%\address{Dept. of Math., Texas State University--San Marcos, San Marcos, TX 78666}
       
\pagestyle{empty}


\begin{document}

\lstset{language=[Objective]Caml,frame=shadowbox}


\input{rset.front}

In the early 2000's Pitts and Gabbay's use of Fraenkl-Mostowski set
theory (FM set theory), a theory of sets with atoms, to model nominal phenomena,
sparked a vibrant line of research. \cite{DBLP:journals/fac/GabbayP02}
\cite{DBLP:journals/jcss/Clouston14}. As Chaitin points out, the
axioms and assumptions of a theory comprise its risk
\cite{chaitin1999unknowable}. This observation raises an interesting
point: is it possible to have a set theory with atoms without taking
on \emph{infinite} risk? It turns out it is.

We describe a theory of sets that enjoys both the cleanliness of
$\mathsf{ZF}$, that is arising only from the empty set, and yet with
an infinite supply of atoms.

\input{rset.intuitions}

\input{rset.presentation}

\input{rset.stdfmset}

\input{rset.mainthm}

\input{rset.conc}

\input{rset.ack}

\bibliographystyle{plain}   
\bibliography{rset.bib}

\end{document}
 

%\ifpdf
%\usepackage[pdftex]{graphicx}
%\else
%\usepackage{graphicx}
%\fi

 % \ifpdf
%  \usepackage{pdfsync}
%  \if


%\title{Brief Article}
%\author{David F. Snyder}
%\author{L.G. Meredith}

%\address{Dept. of Math., Texas State University--San Marcos, San Marcos, TX 78666}
       
\pagestyle{empty}


\begin{document}

\lstset{language=[Objective]Caml,frame=shadowbox}


\def\lastname{Meredith}

\title{Notes on a reflective theory of sets}
\titlerunning{oslf}

\author{ Lucius Gregory Meredith\inst{1}
}
\institute{
        {Managing Partner, RTech Unchained 9336 California Ave SW, Seattle, WA 98103, USA} \\
  \email{ lgreg.meredith@gmail.com } 
}
 

\maketitle              % typeset the title of the contribution

%%% ----------------------------------------------------------------------

\begin{abstract}

  We describe a universe of set theories enjoying atoms without enduring infinite risk.

\end{abstract}

%\keywords{Process calculi, knots, invariants}

% \begin{keyword}
% concurrency, message-passing, process calculus, reflection, program logic
% \end{keyword}

%\end{frontmatter}


In the early 2000's Pitts and Gabbay's use of Fraenkl-Mostowski set
theory (FM set theory), a theory of sets with atoms, to model nominal phenomena,
sparked a vibrant line of research. \cite{DBLP:journals/fac/GabbayP02}
\cite{DBLP:journals/jcss/Clouston14}. As Chaitin points out, the
axioms and assumptions of a theory comprise its risk
\cite{chaitin1999unknowable}. This observation raises an interesting
point: is it possible to have a set theory with atoms without taking
on \emph{infinite} risk? It turns out it is.

We describe a theory of sets that enjoys both the cleanliness of
$\mathsf{ZF}$, that is arising only from the empty set, and yet with
an infinite supply of atoms.

\section{Intuitions}
We imagine two copies of FM set theory, say a \textcolor{red}{red one}
and a black one. The \textcolor{red}{red theory} has red
comprehensions,\textcolor{red}{$\{ \ldots \}$} and a red element-of
relation, \textcolor{red}{$\in$}, while the black theory,
correspondingly has black comprehensions, $\{ \ldots \}$ and a black
element-of relation, $\in$. The red element-of relation,
\textcolor{red}{$\in$}, can ``see into'' red sets, \textcolor{red}{$\{
  \ldots \}$}, but not black ones. Meanwhile, the black element-of
relation, $\in$, can ``see into'' black sets $\{ \ldots \}$, but not
red ones. Thus, the black sets can serve as atoms for the red sets,
while the red sets can serve as atoms for the black sets.

So, the two theories are mutually recursively defined. This mutual
recursion is closely linked to game semantics in the sense of
Abramksy, Hyland, and Ong, et al. Without loss of generality, we can
think of \textcolor{red}{red} as player and black as opponent. A
well-formed red set, \textcolor{red}{$S$}, is a winning strategy for
player, while a well-formed black set, $S$, is a winning strategy for
opponent.


\section{Formal theory}
The judgment \(\Sigma; \Gamma \vdash S\) is pronounced
``\(\Sigma\) thinks that \(S\) is well-formed, given
dependencies, \(\Gamma\) .'' Similarly,
\(\Sigma; \Gamma \vdash x\) is pronounced
``\(\Sigma\) thinks that \(x\) is an admissible atom, given
dependencies, \(\Gamma\).''

We think of $X$ as some infinite supply of ``atoms'' and
$\mathsf{S}[X]$ as a \emph{stream} of atoms drawn from $X$. We assume
basic stream operations on $\mathsf{S}[X]$, such as
$\mathsf{take}(\mathsf{S}[X],n)$ which returns the pair
$(\sigma,\Sigma)$ where $\sigma = x_{1}:\ldots:x_{n}$ and $\Sigma = \mathsf{drop}(\mathsf{S}[X],n)$. Given $\rho = \{j_{1}/i_{1}:\ldots:j_{n}/i_{n}\}$ the operation
$\mathsf{swap}(\mathsf{S}[X],\rho)$ denotes the application of the
permutation $\rho$ to $\mathsf{S}[X]$.

Thus, $\mathsf{S}[X]$ may be thought of as an ordering on $X$.

A rule of the form

\[\frac{ H_1, \ldots , H_n }{ C }R\]

is pronounced ``\(R\) concludes that \(C\) given \(H_1\), \(\ldots\),
\(H_n\)''.

We use $e,f,\ldots$ to range over sets and atoms.

The rules for judging when an atom is admissible or set is well
formed are as follows

\[\frac{ }{ \mathsf{S}[X]; () \vdash \emptyset}Foundation\]

\[\frac{ (x,\Sigma) = \mathsf{take}(\mathsf{S}[X],1)}{ \Sigma; x \vdash \{ x \}}Atomicity\]

\[\frac{ \Sigma; \Gamma \vdash S }{ \Sigma; \Gamma \vdash \{ S \}}Nesting\]

\[\frac{ \Sigma; \Gamma_1 \vdash S_1 \; \Sigma; \Gamma_2 \vdash S_2}{ \Sigma; \Gamma_1, \Gamma_2 \vdash S_1 \cup S_2}Union\]

\[\frac{ \Sigma; \Gamma_1 \vdash S_1 \; \Sigma; \Gamma_2 \vdash S_2}{ \Sigma; \Gamma_1, \Gamma_2 \vdash S_1 \cap S_2}Intersection\]

\[\frac{ \Sigma; \Gamma_1 \vdash S_1 \; \Sigma; \Gamma_2 \vdash S_2}{ \Sigma; \Gamma_1, \Gamma_2 \vdash S_1 \setminus S_2}Subtraction\]

\[\frac{ \Sigma; \Gamma \vdash S \; \Sigma; \Gamma_{1} \vdash S_{1} \;\ldots\; \Sigma; \Gamma_{n} \vdash S_{2}\; \Sigma; \Delta \vdash f : S_{1} \times \ldots \times S_{n} \to S }{ \Sigma; \Gamma,\Gamma_{1},\ldots, \Gamma_{n},\Delta \vdash \{ f(s_{1},\ldots,s_{n}) \; : \; s_1 \in S_{1},\; \ldots\;, s_{n} \in S_{n}\}}Comprehension\]

\subsection{Equations}\label{equations}

The syntactic theory is too fine grained. It makes syntactic
distinctions that do not correspond to distinct sets. We erase these
syntactic distinctions with a set of equations on set expressions.

\[\frac{ \Sigma; \Gamma \vdash e }{ \Sigma; \Gamma  \vdash e = e}Identity\]
\[\frac{\Sigma; \Gamma_{1} \vdash _{1}\; \Sigma; \Gamma_{2} \vdash e_{2}\; \Sigma; \Gamma_{3}  \vdash e_{1} = e_{2}}{ \Sigma; \Gamma_{3}  \vdash e_{1} = e_{2}}Symmetry\]
\[\frac{\Sigma; \Gamma_{1}  \vdash e_{1}\; \Sigma; \Gamma_{2}  \vdash e_{2}\; \Sigma; \Gamma_{3} \vdash e_{3}\; \Sigma; \Gamma_{4}  \vdash e_{1} = e_{2} \; \Sigma; \Gamma_{5}  \vdash e_{2} = e_{3}}{ \Sigma; \Gamma_{4},\Gamma_{5}  \vdash e_{1} = e_{3}}Transitivity\]

\[\frac{\Sigma; \Gamma  \vdash S}{\Sigma; \Gamma \vdash S \cup S =  S}Idempotence_{\cup}\]
\[\frac{\Sigma; \Gamma  \vdash S}{\Sigma; \Gamma \vdash S \cap S =  S}Idempotence_{\cap}\]




\documentclass[12pt]{llncs}

\usepackage[pdftex]{hyperref}                   
\usepackage {mathpartir}
\usepackage{bigpage}
\usepackage{bcprules}
\usepackage {listings}
\usepackage{amsmath}
\usepackage{amssymb}
\usepackage{amsfonts}
\usepackage{amstext}
\usepackage{latexsym}
\usepackage{longtable}
\usepackage{stmaryrd}
%\usepackage{fdsymbol}
\usepackage{graphicx} 
%\usepackage[margins=2.5cm,nohead,nofoot]{geometry}
%\usepackage{geometry}
\usepackage{color}
\usepackage{xcolor}


%\include{myPreamble}
\include{rset.local} 

%\ifpdf
%\usepackage[pdftex]{graphicx}
%\else
%\usepackage{graphicx}
%\fi

 % \ifpdf
%  \usepackage{pdfsync}
%  \if


%\title{Brief Article}
%\author{David F. Snyder}
%\author{L.G. Meredith}

%\address{Dept. of Math., Texas State University--San Marcos, San Marcos, TX 78666}
       
\pagestyle{empty}


\begin{document}

\lstset{language=[Objective]Caml,frame=shadowbox}


\input{rset.front}

In the early 2000's Pitts and Gabbay's use of Fraenkl-Mostowski set
theory (FM set theory), a theory of sets with atoms, to model nominal phenomena,
sparked a vibrant line of research. \cite{DBLP:journals/fac/GabbayP02}
\cite{DBLP:journals/jcss/Clouston14}. As Chaitin points out, the
axioms and assumptions of a theory comprise its risk
\cite{chaitin1999unknowable}. This observation raises an interesting
point: is it possible to have a set theory with atoms without taking
on \emph{infinite} risk? It turns out it is.

We describe a theory of sets that enjoys both the cleanliness of
$\mathsf{ZF}$, that is arising only from the empty set, and yet with
an infinite supply of atoms.

\input{rset.intuitions}

\input{rset.presentation}

\input{rset.stdfmset}

\input{rset.mainthm}

\input{rset.conc}

\input{rset.ack}

\bibliographystyle{plain}   
\bibliography{rset.bib}

\end{document}


\documentclass[12pt]{llncs}

\usepackage[pdftex]{hyperref}                   
\usepackage {mathpartir}
\usepackage{bigpage}
\usepackage{bcprules}
\usepackage {listings}
\usepackage{amsmath}
\usepackage{amssymb}
\usepackage{amsfonts}
\usepackage{amstext}
\usepackage{latexsym}
\usepackage{longtable}
\usepackage{stmaryrd}
%\usepackage{fdsymbol}
\usepackage{graphicx} 
%\usepackage[margins=2.5cm,nohead,nofoot]{geometry}
%\usepackage{geometry}
\usepackage{color}
\usepackage{xcolor}


%\include{myPreamble}
\include{rset.local} 

%\ifpdf
%\usepackage[pdftex]{graphicx}
%\else
%\usepackage{graphicx}
%\fi

 % \ifpdf
%  \usepackage{pdfsync}
%  \if


%\title{Brief Article}
%\author{David F. Snyder}
%\author{L.G. Meredith}

%\address{Dept. of Math., Texas State University--San Marcos, San Marcos, TX 78666}
       
\pagestyle{empty}


\begin{document}

\lstset{language=[Objective]Caml,frame=shadowbox}


\input{rset.front}

In the early 2000's Pitts and Gabbay's use of Fraenkl-Mostowski set
theory (FM set theory), a theory of sets with atoms, to model nominal phenomena,
sparked a vibrant line of research. \cite{DBLP:journals/fac/GabbayP02}
\cite{DBLP:journals/jcss/Clouston14}. As Chaitin points out, the
axioms and assumptions of a theory comprise its risk
\cite{chaitin1999unknowable}. This observation raises an interesting
point: is it possible to have a set theory with atoms without taking
on \emph{infinite} risk? It turns out it is.

We describe a theory of sets that enjoys both the cleanliness of
$\mathsf{ZF}$, that is arising only from the empty set, and yet with
an infinite supply of atoms.

\input{rset.intuitions}

\input{rset.presentation}

\input{rset.stdfmset}

\input{rset.mainthm}

\input{rset.conc}

\input{rset.ack}

\bibliographystyle{plain}   
\bibliography{rset.bib}

\end{document}


\section{Conclusions and future work}

We have presented a formal theory of sets that reconciles set theory with atoms with a set theory built only from the empty set.

% subsection other_calculi_other_bisimulations_and_geometry_as_behavior (end)




\paragraph{Acknowledgments.}
The author wishes to thank Jamie Gabbay for stimulating discussions
that led the author to devise this theory.


\bibliographystyle{plain}   
\bibliography{rset.bib}

\end{document}


\documentclass[12pt]{llncs}

\usepackage[pdftex]{hyperref}                   
\usepackage {mathpartir}
\usepackage{bigpage}
\usepackage{bcprules}
\usepackage {listings}
\usepackage{amsmath}
\usepackage{amssymb}
\usepackage{amsfonts}
\usepackage{amstext}
\usepackage{latexsym}
\usepackage{longtable}
\usepackage{stmaryrd}
%\usepackage{fdsymbol}
\usepackage{graphicx} 
%\usepackage[margins=2.5cm,nohead,nofoot]{geometry}
%\usepackage{geometry}
\usepackage{color}
\usepackage{xcolor}


%\include{myPreamble}
\documentclass[12pt]{llncs}

\usepackage[pdftex]{hyperref}                   
\usepackage {mathpartir}
\usepackage{bigpage}
\usepackage{bcprules}
\usepackage {listings}
\usepackage{amsmath}
\usepackage{amssymb}
\usepackage{amsfonts}
\usepackage{amstext}
\usepackage{latexsym}
\usepackage{longtable}
\usepackage{stmaryrd}
%\usepackage{fdsymbol}
\usepackage{graphicx} 
%\usepackage[margins=2.5cm,nohead,nofoot]{geometry}
%\usepackage{geometry}
\usepackage{color}
\usepackage{xcolor}


%\include{myPreamble}
\include{rset.local} 

%\ifpdf
%\usepackage[pdftex]{graphicx}
%\else
%\usepackage{graphicx}
%\fi

 % \ifpdf
%  \usepackage{pdfsync}
%  \if


%\title{Brief Article}
%\author{David F. Snyder}
%\author{L.G. Meredith}

%\address{Dept. of Math., Texas State University--San Marcos, San Marcos, TX 78666}
       
\pagestyle{empty}


\begin{document}

\lstset{language=[Objective]Caml,frame=shadowbox}


\input{rset.front}

In the early 2000's Pitts and Gabbay's use of Fraenkl-Mostowski set
theory (FM set theory), a theory of sets with atoms, to model nominal phenomena,
sparked a vibrant line of research. \cite{DBLP:journals/fac/GabbayP02}
\cite{DBLP:journals/jcss/Clouston14}. As Chaitin points out, the
axioms and assumptions of a theory comprise its risk
\cite{chaitin1999unknowable}. This observation raises an interesting
point: is it possible to have a set theory with atoms without taking
on \emph{infinite} risk? It turns out it is.

We describe a theory of sets that enjoys both the cleanliness of
$\mathsf{ZF}$, that is arising only from the empty set, and yet with
an infinite supply of atoms.

\input{rset.intuitions}

\input{rset.presentation}

\input{rset.stdfmset}

\input{rset.mainthm}

\input{rset.conc}

\input{rset.ack}

\bibliographystyle{plain}   
\bibliography{rset.bib}

\end{document}
 

%\ifpdf
%\usepackage[pdftex]{graphicx}
%\else
%\usepackage{graphicx}
%\fi

 % \ifpdf
%  \usepackage{pdfsync}
%  \if


%\title{Brief Article}
%\author{David F. Snyder}
%\author{L.G. Meredith}

%\address{Dept. of Math., Texas State University--San Marcos, San Marcos, TX 78666}
       
\pagestyle{empty}


\begin{document}

\lstset{language=[Objective]Caml,frame=shadowbox}


\def\lastname{Meredith}

\title{Notes on a reflective theory of sets}
\titlerunning{oslf}

\author{ Lucius Gregory Meredith\inst{1}
}
\institute{
        {Managing Partner, RTech Unchained 9336 California Ave SW, Seattle, WA 98103, USA} \\
  \email{ lgreg.meredith@gmail.com } 
}
 

\maketitle              % typeset the title of the contribution

%%% ----------------------------------------------------------------------

\begin{abstract}

  We describe a universe of set theories enjoying atoms without enduring infinite risk.

\end{abstract}

%\keywords{Process calculi, knots, invariants}

% \begin{keyword}
% concurrency, message-passing, process calculus, reflection, program logic
% \end{keyword}

%\end{frontmatter}


In the early 2000's Pitts and Gabbay's use of Fraenkl-Mostowski set
theory (FM set theory), a theory of sets with atoms, to model nominal phenomena,
sparked a vibrant line of research. \cite{DBLP:journals/fac/GabbayP02}
\cite{DBLP:journals/jcss/Clouston14}. As Chaitin points out, the
axioms and assumptions of a theory comprise its risk
\cite{chaitin1999unknowable}. This observation raises an interesting
point: is it possible to have a set theory with atoms without taking
on \emph{infinite} risk? It turns out it is.

We describe a theory of sets that enjoys both the cleanliness of
$\mathsf{ZF}$, that is arising only from the empty set, and yet with
an infinite supply of atoms.

\section{Intuitions}
We imagine two copies of FM set theory, say a \textcolor{red}{red one}
and a black one. The \textcolor{red}{red theory} has red
comprehensions,\textcolor{red}{$\{ \ldots \}$} and a red element-of
relation, \textcolor{red}{$\in$}, while the black theory,
correspondingly has black comprehensions, $\{ \ldots \}$ and a black
element-of relation, $\in$. The red element-of relation,
\textcolor{red}{$\in$}, can ``see into'' red sets, \textcolor{red}{$\{
  \ldots \}$}, but not black ones. Meanwhile, the black element-of
relation, $\in$, can ``see into'' black sets $\{ \ldots \}$, but not
red ones. Thus, the black sets can serve as atoms for the red sets,
while the red sets can serve as atoms for the black sets.

So, the two theories are mutually recursively defined. This mutual
recursion is closely linked to game semantics in the sense of
Abramksy, Hyland, and Ong, et al. Without loss of generality, we can
think of \textcolor{red}{red} as player and black as opponent. A
well-formed red set, \textcolor{red}{$S$}, is a winning strategy for
player, while a well-formed black set, $S$, is a winning strategy for
opponent.


\section{Formal theory}
The judgment \(\Sigma; \Gamma \vdash S\) is pronounced
``\(\Sigma\) thinks that \(S\) is well-formed, given
dependencies, \(\Gamma\) .'' Similarly,
\(\Sigma; \Gamma \vdash x\) is pronounced
``\(\Sigma\) thinks that \(x\) is an admissible atom, given
dependencies, \(\Gamma\).''

We think of $X$ as some infinite supply of ``atoms'' and
$\mathsf{S}[X]$ as a \emph{stream} of atoms drawn from $X$. We assume
basic stream operations on $\mathsf{S}[X]$, such as
$\mathsf{take}(\mathsf{S}[X],n)$ which returns the pair
$(\sigma,\Sigma)$ where $\sigma = x_{1}:\ldots:x_{n}$ and $\Sigma = \mathsf{drop}(\mathsf{S}[X],n)$. Given $\rho = \{j_{1}/i_{1}:\ldots:j_{n}/i_{n}\}$ the operation
$\mathsf{swap}(\mathsf{S}[X],\rho)$ denotes the application of the
permutation $\rho$ to $\mathsf{S}[X]$.

Thus, $\mathsf{S}[X]$ may be thought of as an ordering on $X$.

A rule of the form

\[\frac{ H_1, \ldots , H_n }{ C }R\]

is pronounced ``\(R\) concludes that \(C\) given \(H_1\), \(\ldots\),
\(H_n\)''.

We use $e,f,\ldots$ to range over sets and atoms.

The rules for judging when an atom is admissible or set is well
formed are as follows

\[\frac{ }{ \mathsf{S}[X]; () \vdash \emptyset}Foundation\]

\[\frac{ (x,\Sigma) = \mathsf{take}(\mathsf{S}[X],1)}{ \Sigma; x \vdash \{ x \}}Atomicity\]

\[\frac{ \Sigma; \Gamma \vdash S }{ \Sigma; \Gamma \vdash \{ S \}}Nesting\]

\[\frac{ \Sigma; \Gamma_1 \vdash S_1 \; \Sigma; \Gamma_2 \vdash S_2}{ \Sigma; \Gamma_1, \Gamma_2 \vdash S_1 \cup S_2}Union\]

\[\frac{ \Sigma; \Gamma_1 \vdash S_1 \; \Sigma; \Gamma_2 \vdash S_2}{ \Sigma; \Gamma_1, \Gamma_2 \vdash S_1 \cap S_2}Intersection\]

\[\frac{ \Sigma; \Gamma_1 \vdash S_1 \; \Sigma; \Gamma_2 \vdash S_2}{ \Sigma; \Gamma_1, \Gamma_2 \vdash S_1 \setminus S_2}Subtraction\]

\[\frac{ \Sigma; \Gamma \vdash S \; \Sigma; \Gamma_{1} \vdash S_{1} \;\ldots\; \Sigma; \Gamma_{n} \vdash S_{2}\; \Sigma; \Delta \vdash f : S_{1} \times \ldots \times S_{n} \to S }{ \Sigma; \Gamma,\Gamma_{1},\ldots, \Gamma_{n},\Delta \vdash \{ f(s_{1},\ldots,s_{n}) \; : \; s_1 \in S_{1},\; \ldots\;, s_{n} \in S_{n}\}}Comprehension\]

\subsection{Equations}\label{equations}

The syntactic theory is too fine grained. It makes syntactic
distinctions that do not correspond to distinct sets. We erase these
syntactic distinctions with a set of equations on set expressions.

\[\frac{ \Sigma; \Gamma \vdash e }{ \Sigma; \Gamma  \vdash e = e}Identity\]
\[\frac{\Sigma; \Gamma_{1} \vdash _{1}\; \Sigma; \Gamma_{2} \vdash e_{2}\; \Sigma; \Gamma_{3}  \vdash e_{1} = e_{2}}{ \Sigma; \Gamma_{3}  \vdash e_{1} = e_{2}}Symmetry\]
\[\frac{\Sigma; \Gamma_{1}  \vdash e_{1}\; \Sigma; \Gamma_{2}  \vdash e_{2}\; \Sigma; \Gamma_{3} \vdash e_{3}\; \Sigma; \Gamma_{4}  \vdash e_{1} = e_{2} \; \Sigma; \Gamma_{5}  \vdash e_{2} = e_{3}}{ \Sigma; \Gamma_{4},\Gamma_{5}  \vdash e_{1} = e_{3}}Transitivity\]

\[\frac{\Sigma; \Gamma  \vdash S}{\Sigma; \Gamma \vdash S \cup S =  S}Idempotence_{\cup}\]
\[\frac{\Sigma; \Gamma  \vdash S}{\Sigma; \Gamma \vdash S \cap S =  S}Idempotence_{\cap}\]




\documentclass[12pt]{llncs}

\usepackage[pdftex]{hyperref}                   
\usepackage {mathpartir}
\usepackage{bigpage}
\usepackage{bcprules}
\usepackage {listings}
\usepackage{amsmath}
\usepackage{amssymb}
\usepackage{amsfonts}
\usepackage{amstext}
\usepackage{latexsym}
\usepackage{longtable}
\usepackage{stmaryrd}
%\usepackage{fdsymbol}
\usepackage{graphicx} 
%\usepackage[margins=2.5cm,nohead,nofoot]{geometry}
%\usepackage{geometry}
\usepackage{color}
\usepackage{xcolor}


%\include{myPreamble}
\include{rset.local} 

%\ifpdf
%\usepackage[pdftex]{graphicx}
%\else
%\usepackage{graphicx}
%\fi

 % \ifpdf
%  \usepackage{pdfsync}
%  \if


%\title{Brief Article}
%\author{David F. Snyder}
%\author{L.G. Meredith}

%\address{Dept. of Math., Texas State University--San Marcos, San Marcos, TX 78666}
       
\pagestyle{empty}


\begin{document}

\lstset{language=[Objective]Caml,frame=shadowbox}


\input{rset.front}

In the early 2000's Pitts and Gabbay's use of Fraenkl-Mostowski set
theory (FM set theory), a theory of sets with atoms, to model nominal phenomena,
sparked a vibrant line of research. \cite{DBLP:journals/fac/GabbayP02}
\cite{DBLP:journals/jcss/Clouston14}. As Chaitin points out, the
axioms and assumptions of a theory comprise its risk
\cite{chaitin1999unknowable}. This observation raises an interesting
point: is it possible to have a set theory with atoms without taking
on \emph{infinite} risk? It turns out it is.

We describe a theory of sets that enjoys both the cleanliness of
$\mathsf{ZF}$, that is arising only from the empty set, and yet with
an infinite supply of atoms.

\input{rset.intuitions}

\input{rset.presentation}

\input{rset.stdfmset}

\input{rset.mainthm}

\input{rset.conc}

\input{rset.ack}

\bibliographystyle{plain}   
\bibliography{rset.bib}

\end{document}


\documentclass[12pt]{llncs}

\usepackage[pdftex]{hyperref}                   
\usepackage {mathpartir}
\usepackage{bigpage}
\usepackage{bcprules}
\usepackage {listings}
\usepackage{amsmath}
\usepackage{amssymb}
\usepackage{amsfonts}
\usepackage{amstext}
\usepackage{latexsym}
\usepackage{longtable}
\usepackage{stmaryrd}
%\usepackage{fdsymbol}
\usepackage{graphicx} 
%\usepackage[margins=2.5cm,nohead,nofoot]{geometry}
%\usepackage{geometry}
\usepackage{color}
\usepackage{xcolor}


%\include{myPreamble}
\include{rset.local} 

%\ifpdf
%\usepackage[pdftex]{graphicx}
%\else
%\usepackage{graphicx}
%\fi

 % \ifpdf
%  \usepackage{pdfsync}
%  \if


%\title{Brief Article}
%\author{David F. Snyder}
%\author{L.G. Meredith}

%\address{Dept. of Math., Texas State University--San Marcos, San Marcos, TX 78666}
       
\pagestyle{empty}


\begin{document}

\lstset{language=[Objective]Caml,frame=shadowbox}


\input{rset.front}

In the early 2000's Pitts and Gabbay's use of Fraenkl-Mostowski set
theory (FM set theory), a theory of sets with atoms, to model nominal phenomena,
sparked a vibrant line of research. \cite{DBLP:journals/fac/GabbayP02}
\cite{DBLP:journals/jcss/Clouston14}. As Chaitin points out, the
axioms and assumptions of a theory comprise its risk
\cite{chaitin1999unknowable}. This observation raises an interesting
point: is it possible to have a set theory with atoms without taking
on \emph{infinite} risk? It turns out it is.

We describe a theory of sets that enjoys both the cleanliness of
$\mathsf{ZF}$, that is arising only from the empty set, and yet with
an infinite supply of atoms.

\input{rset.intuitions}

\input{rset.presentation}

\input{rset.stdfmset}

\input{rset.mainthm}

\input{rset.conc}

\input{rset.ack}

\bibliographystyle{plain}   
\bibliography{rset.bib}

\end{document}


\section{Conclusions and future work}

We have presented a formal theory of sets that reconciles set theory with atoms with a set theory built only from the empty set.

% subsection other_calculi_other_bisimulations_and_geometry_as_behavior (end)




\paragraph{Acknowledgments.}
The author wishes to thank Jamie Gabbay for stimulating discussions
that led the author to devise this theory.


\bibliographystyle{plain}   
\bibliography{rset.bib}

\end{document}


\section{Conclusions and future work}

We have presented a formal theory of sets that reconciles set theory with atoms with a set theory built only from the empty set.

% subsection other_calculi_other_bisimulations_and_geometry_as_behavior (end)




\paragraph{Acknowledgments.}
The author wishes to thank Jamie Gabbay for stimulating discussions
that led the author to devise this theory.


\bibliographystyle{plain}   
\bibliography{rset.bib}

\end{document}


\documentclass[12pt]{llncs}

\usepackage[pdftex]{hyperref}                   
\usepackage {mathpartir}
\usepackage{bigpage}
\usepackage{bcprules}
\usepackage {listings}
\usepackage{amsmath}
\usepackage{amssymb}
\usepackage{amsfonts}
\usepackage{amstext}
\usepackage{latexsym}
\usepackage{longtable}
\usepackage{stmaryrd}
%\usepackage{fdsymbol}
\usepackage{graphicx} 
%\usepackage[margins=2.5cm,nohead,nofoot]{geometry}
%\usepackage{geometry}
\usepackage{color}
\usepackage{xcolor}


%\include{myPreamble}
\documentclass[12pt]{llncs}

\usepackage[pdftex]{hyperref}                   
\usepackage {mathpartir}
\usepackage{bigpage}
\usepackage{bcprules}
\usepackage {listings}
\usepackage{amsmath}
\usepackage{amssymb}
\usepackage{amsfonts}
\usepackage{amstext}
\usepackage{latexsym}
\usepackage{longtable}
\usepackage{stmaryrd}
%\usepackage{fdsymbol}
\usepackage{graphicx} 
%\usepackage[margins=2.5cm,nohead,nofoot]{geometry}
%\usepackage{geometry}
\usepackage{color}
\usepackage{xcolor}


%\include{myPreamble}
\documentclass[12pt]{llncs}

\usepackage[pdftex]{hyperref}                   
\usepackage {mathpartir}
\usepackage{bigpage}
\usepackage{bcprules}
\usepackage {listings}
\usepackage{amsmath}
\usepackage{amssymb}
\usepackage{amsfonts}
\usepackage{amstext}
\usepackage{latexsym}
\usepackage{longtable}
\usepackage{stmaryrd}
%\usepackage{fdsymbol}
\usepackage{graphicx} 
%\usepackage[margins=2.5cm,nohead,nofoot]{geometry}
%\usepackage{geometry}
\usepackage{color}
\usepackage{xcolor}


%\include{myPreamble}
\include{rset.local} 

%\ifpdf
%\usepackage[pdftex]{graphicx}
%\else
%\usepackage{graphicx}
%\fi

 % \ifpdf
%  \usepackage{pdfsync}
%  \if


%\title{Brief Article}
%\author{David F. Snyder}
%\author{L.G. Meredith}

%\address{Dept. of Math., Texas State University--San Marcos, San Marcos, TX 78666}
       
\pagestyle{empty}


\begin{document}

\lstset{language=[Objective]Caml,frame=shadowbox}


\input{rset.front}

In the early 2000's Pitts and Gabbay's use of Fraenkl-Mostowski set
theory (FM set theory), a theory of sets with atoms, to model nominal phenomena,
sparked a vibrant line of research. \cite{DBLP:journals/fac/GabbayP02}
\cite{DBLP:journals/jcss/Clouston14}. As Chaitin points out, the
axioms and assumptions of a theory comprise its risk
\cite{chaitin1999unknowable}. This observation raises an interesting
point: is it possible to have a set theory with atoms without taking
on \emph{infinite} risk? It turns out it is.

We describe a theory of sets that enjoys both the cleanliness of
$\mathsf{ZF}$, that is arising only from the empty set, and yet with
an infinite supply of atoms.

\input{rset.intuitions}

\input{rset.presentation}

\input{rset.stdfmset}

\input{rset.mainthm}

\input{rset.conc}

\input{rset.ack}

\bibliographystyle{plain}   
\bibliography{rset.bib}

\end{document}
 

%\ifpdf
%\usepackage[pdftex]{graphicx}
%\else
%\usepackage{graphicx}
%\fi

 % \ifpdf
%  \usepackage{pdfsync}
%  \if


%\title{Brief Article}
%\author{David F. Snyder}
%\author{L.G. Meredith}

%\address{Dept. of Math., Texas State University--San Marcos, San Marcos, TX 78666}
       
\pagestyle{empty}


\begin{document}

\lstset{language=[Objective]Caml,frame=shadowbox}


\def\lastname{Meredith}

\title{Notes on a reflective theory of sets}
\titlerunning{oslf}

\author{ Lucius Gregory Meredith\inst{1}
}
\institute{
        {Managing Partner, RTech Unchained 9336 California Ave SW, Seattle, WA 98103, USA} \\
  \email{ lgreg.meredith@gmail.com } 
}
 

\maketitle              % typeset the title of the contribution

%%% ----------------------------------------------------------------------

\begin{abstract}

  We describe a universe of set theories enjoying atoms without enduring infinite risk.

\end{abstract}

%\keywords{Process calculi, knots, invariants}

% \begin{keyword}
% concurrency, message-passing, process calculus, reflection, program logic
% \end{keyword}

%\end{frontmatter}


In the early 2000's Pitts and Gabbay's use of Fraenkl-Mostowski set
theory (FM set theory), a theory of sets with atoms, to model nominal phenomena,
sparked a vibrant line of research. \cite{DBLP:journals/fac/GabbayP02}
\cite{DBLP:journals/jcss/Clouston14}. As Chaitin points out, the
axioms and assumptions of a theory comprise its risk
\cite{chaitin1999unknowable}. This observation raises an interesting
point: is it possible to have a set theory with atoms without taking
on \emph{infinite} risk? It turns out it is.

We describe a theory of sets that enjoys both the cleanliness of
$\mathsf{ZF}$, that is arising only from the empty set, and yet with
an infinite supply of atoms.

\section{Intuitions}
We imagine two copies of FM set theory, say a \textcolor{red}{red one}
and a black one. The \textcolor{red}{red theory} has red
comprehensions,\textcolor{red}{$\{ \ldots \}$} and a red element-of
relation, \textcolor{red}{$\in$}, while the black theory,
correspondingly has black comprehensions, $\{ \ldots \}$ and a black
element-of relation, $\in$. The red element-of relation,
\textcolor{red}{$\in$}, can ``see into'' red sets, \textcolor{red}{$\{
  \ldots \}$}, but not black ones. Meanwhile, the black element-of
relation, $\in$, can ``see into'' black sets $\{ \ldots \}$, but not
red ones. Thus, the black sets can serve as atoms for the red sets,
while the red sets can serve as atoms for the black sets.

So, the two theories are mutually recursively defined. This mutual
recursion is closely linked to game semantics in the sense of
Abramksy, Hyland, and Ong, et al. Without loss of generality, we can
think of \textcolor{red}{red} as player and black as opponent. A
well-formed red set, \textcolor{red}{$S$}, is a winning strategy for
player, while a well-formed black set, $S$, is a winning strategy for
opponent.


\section{Formal theory}
The judgment \(\Sigma; \Gamma \vdash S\) is pronounced
``\(\Sigma\) thinks that \(S\) is well-formed, given
dependencies, \(\Gamma\) .'' Similarly,
\(\Sigma; \Gamma \vdash x\) is pronounced
``\(\Sigma\) thinks that \(x\) is an admissible atom, given
dependencies, \(\Gamma\).''

We think of $X$ as some infinite supply of ``atoms'' and
$\mathsf{S}[X]$ as a \emph{stream} of atoms drawn from $X$. We assume
basic stream operations on $\mathsf{S}[X]$, such as
$\mathsf{take}(\mathsf{S}[X],n)$ which returns the pair
$(\sigma,\Sigma)$ where $\sigma = x_{1}:\ldots:x_{n}$ and $\Sigma = \mathsf{drop}(\mathsf{S}[X],n)$. Given $\rho = \{j_{1}/i_{1}:\ldots:j_{n}/i_{n}\}$ the operation
$\mathsf{swap}(\mathsf{S}[X],\rho)$ denotes the application of the
permutation $\rho$ to $\mathsf{S}[X]$.

Thus, $\mathsf{S}[X]$ may be thought of as an ordering on $X$.

A rule of the form

\[\frac{ H_1, \ldots , H_n }{ C }R\]

is pronounced ``\(R\) concludes that \(C\) given \(H_1\), \(\ldots\),
\(H_n\)''.

We use $e,f,\ldots$ to range over sets and atoms.

The rules for judging when an atom is admissible or set is well
formed are as follows

\[\frac{ }{ \mathsf{S}[X]; () \vdash \emptyset}Foundation\]

\[\frac{ (x,\Sigma) = \mathsf{take}(\mathsf{S}[X],1)}{ \Sigma; x \vdash \{ x \}}Atomicity\]

\[\frac{ \Sigma; \Gamma \vdash S }{ \Sigma; \Gamma \vdash \{ S \}}Nesting\]

\[\frac{ \Sigma; \Gamma_1 \vdash S_1 \; \Sigma; \Gamma_2 \vdash S_2}{ \Sigma; \Gamma_1, \Gamma_2 \vdash S_1 \cup S_2}Union\]

\[\frac{ \Sigma; \Gamma_1 \vdash S_1 \; \Sigma; \Gamma_2 \vdash S_2}{ \Sigma; \Gamma_1, \Gamma_2 \vdash S_1 \cap S_2}Intersection\]

\[\frac{ \Sigma; \Gamma_1 \vdash S_1 \; \Sigma; \Gamma_2 \vdash S_2}{ \Sigma; \Gamma_1, \Gamma_2 \vdash S_1 \setminus S_2}Subtraction\]

\[\frac{ \Sigma; \Gamma \vdash S \; \Sigma; \Gamma_{1} \vdash S_{1} \;\ldots\; \Sigma; \Gamma_{n} \vdash S_{2}\; \Sigma; \Delta \vdash f : S_{1} \times \ldots \times S_{n} \to S }{ \Sigma; \Gamma,\Gamma_{1},\ldots, \Gamma_{n},\Delta \vdash \{ f(s_{1},\ldots,s_{n}) \; : \; s_1 \in S_{1},\; \ldots\;, s_{n} \in S_{n}\}}Comprehension\]

\subsection{Equations}\label{equations}

The syntactic theory is too fine grained. It makes syntactic
distinctions that do not correspond to distinct sets. We erase these
syntactic distinctions with a set of equations on set expressions.

\[\frac{ \Sigma; \Gamma \vdash e }{ \Sigma; \Gamma  \vdash e = e}Identity\]
\[\frac{\Sigma; \Gamma_{1} \vdash _{1}\; \Sigma; \Gamma_{2} \vdash e_{2}\; \Sigma; \Gamma_{3}  \vdash e_{1} = e_{2}}{ \Sigma; \Gamma_{3}  \vdash e_{1} = e_{2}}Symmetry\]
\[\frac{\Sigma; \Gamma_{1}  \vdash e_{1}\; \Sigma; \Gamma_{2}  \vdash e_{2}\; \Sigma; \Gamma_{3} \vdash e_{3}\; \Sigma; \Gamma_{4}  \vdash e_{1} = e_{2} \; \Sigma; \Gamma_{5}  \vdash e_{2} = e_{3}}{ \Sigma; \Gamma_{4},\Gamma_{5}  \vdash e_{1} = e_{3}}Transitivity\]

\[\frac{\Sigma; \Gamma  \vdash S}{\Sigma; \Gamma \vdash S \cup S =  S}Idempotence_{\cup}\]
\[\frac{\Sigma; \Gamma  \vdash S}{\Sigma; \Gamma \vdash S \cap S =  S}Idempotence_{\cap}\]




\documentclass[12pt]{llncs}

\usepackage[pdftex]{hyperref}                   
\usepackage {mathpartir}
\usepackage{bigpage}
\usepackage{bcprules}
\usepackage {listings}
\usepackage{amsmath}
\usepackage{amssymb}
\usepackage{amsfonts}
\usepackage{amstext}
\usepackage{latexsym}
\usepackage{longtable}
\usepackage{stmaryrd}
%\usepackage{fdsymbol}
\usepackage{graphicx} 
%\usepackage[margins=2.5cm,nohead,nofoot]{geometry}
%\usepackage{geometry}
\usepackage{color}
\usepackage{xcolor}


%\include{myPreamble}
\include{rset.local} 

%\ifpdf
%\usepackage[pdftex]{graphicx}
%\else
%\usepackage{graphicx}
%\fi

 % \ifpdf
%  \usepackage{pdfsync}
%  \if


%\title{Brief Article}
%\author{David F. Snyder}
%\author{L.G. Meredith}

%\address{Dept. of Math., Texas State University--San Marcos, San Marcos, TX 78666}
       
\pagestyle{empty}


\begin{document}

\lstset{language=[Objective]Caml,frame=shadowbox}


\input{rset.front}

In the early 2000's Pitts and Gabbay's use of Fraenkl-Mostowski set
theory (FM set theory), a theory of sets with atoms, to model nominal phenomena,
sparked a vibrant line of research. \cite{DBLP:journals/fac/GabbayP02}
\cite{DBLP:journals/jcss/Clouston14}. As Chaitin points out, the
axioms and assumptions of a theory comprise its risk
\cite{chaitin1999unknowable}. This observation raises an interesting
point: is it possible to have a set theory with atoms without taking
on \emph{infinite} risk? It turns out it is.

We describe a theory of sets that enjoys both the cleanliness of
$\mathsf{ZF}$, that is arising only from the empty set, and yet with
an infinite supply of atoms.

\input{rset.intuitions}

\input{rset.presentation}

\input{rset.stdfmset}

\input{rset.mainthm}

\input{rset.conc}

\input{rset.ack}

\bibliographystyle{plain}   
\bibliography{rset.bib}

\end{document}


\documentclass[12pt]{llncs}

\usepackage[pdftex]{hyperref}                   
\usepackage {mathpartir}
\usepackage{bigpage}
\usepackage{bcprules}
\usepackage {listings}
\usepackage{amsmath}
\usepackage{amssymb}
\usepackage{amsfonts}
\usepackage{amstext}
\usepackage{latexsym}
\usepackage{longtable}
\usepackage{stmaryrd}
%\usepackage{fdsymbol}
\usepackage{graphicx} 
%\usepackage[margins=2.5cm,nohead,nofoot]{geometry}
%\usepackage{geometry}
\usepackage{color}
\usepackage{xcolor}


%\include{myPreamble}
\include{rset.local} 

%\ifpdf
%\usepackage[pdftex]{graphicx}
%\else
%\usepackage{graphicx}
%\fi

 % \ifpdf
%  \usepackage{pdfsync}
%  \if


%\title{Brief Article}
%\author{David F. Snyder}
%\author{L.G. Meredith}

%\address{Dept. of Math., Texas State University--San Marcos, San Marcos, TX 78666}
       
\pagestyle{empty}


\begin{document}

\lstset{language=[Objective]Caml,frame=shadowbox}


\input{rset.front}

In the early 2000's Pitts and Gabbay's use of Fraenkl-Mostowski set
theory (FM set theory), a theory of sets with atoms, to model nominal phenomena,
sparked a vibrant line of research. \cite{DBLP:journals/fac/GabbayP02}
\cite{DBLP:journals/jcss/Clouston14}. As Chaitin points out, the
axioms and assumptions of a theory comprise its risk
\cite{chaitin1999unknowable}. This observation raises an interesting
point: is it possible to have a set theory with atoms without taking
on \emph{infinite} risk? It turns out it is.

We describe a theory of sets that enjoys both the cleanliness of
$\mathsf{ZF}$, that is arising only from the empty set, and yet with
an infinite supply of atoms.

\input{rset.intuitions}

\input{rset.presentation}

\input{rset.stdfmset}

\input{rset.mainthm}

\input{rset.conc}

\input{rset.ack}

\bibliographystyle{plain}   
\bibliography{rset.bib}

\end{document}


\section{Conclusions and future work}

We have presented a formal theory of sets that reconciles set theory with atoms with a set theory built only from the empty set.

% subsection other_calculi_other_bisimulations_and_geometry_as_behavior (end)




\paragraph{Acknowledgments.}
The author wishes to thank Jamie Gabbay for stimulating discussions
that led the author to devise this theory.


\bibliographystyle{plain}   
\bibliography{rset.bib}

\end{document}
 

%\ifpdf
%\usepackage[pdftex]{graphicx}
%\else
%\usepackage{graphicx}
%\fi

 % \ifpdf
%  \usepackage{pdfsync}
%  \if


%\title{Brief Article}
%\author{David F. Snyder}
%\author{L.G. Meredith}

%\address{Dept. of Math., Texas State University--San Marcos, San Marcos, TX 78666}
       
\pagestyle{empty}


\begin{document}

\lstset{language=[Objective]Caml,frame=shadowbox}


\def\lastname{Meredith}

\title{Notes on a reflective theory of sets}
\titlerunning{oslf}

\author{ Lucius Gregory Meredith\inst{1}
}
\institute{
        {Managing Partner, RTech Unchained 9336 California Ave SW, Seattle, WA 98103, USA} \\
  \email{ lgreg.meredith@gmail.com } 
}
 

\maketitle              % typeset the title of the contribution

%%% ----------------------------------------------------------------------

\begin{abstract}

  We describe a universe of set theories enjoying atoms without enduring infinite risk.

\end{abstract}

%\keywords{Process calculi, knots, invariants}

% \begin{keyword}
% concurrency, message-passing, process calculus, reflection, program logic
% \end{keyword}

%\end{frontmatter}


In the early 2000's Pitts and Gabbay's use of Fraenkl-Mostowski set
theory (FM set theory), a theory of sets with atoms, to model nominal phenomena,
sparked a vibrant line of research. \cite{DBLP:journals/fac/GabbayP02}
\cite{DBLP:journals/jcss/Clouston14}. As Chaitin points out, the
axioms and assumptions of a theory comprise its risk
\cite{chaitin1999unknowable}. This observation raises an interesting
point: is it possible to have a set theory with atoms without taking
on \emph{infinite} risk? It turns out it is.

We describe a theory of sets that enjoys both the cleanliness of
$\mathsf{ZF}$, that is arising only from the empty set, and yet with
an infinite supply of atoms.

\section{Intuitions}
We imagine two copies of FM set theory, say a \textcolor{red}{red one}
and a black one. The \textcolor{red}{red theory} has red
comprehensions,\textcolor{red}{$\{ \ldots \}$} and a red element-of
relation, \textcolor{red}{$\in$}, while the black theory,
correspondingly has black comprehensions, $\{ \ldots \}$ and a black
element-of relation, $\in$. The red element-of relation,
\textcolor{red}{$\in$}, can ``see into'' red sets, \textcolor{red}{$\{
  \ldots \}$}, but not black ones. Meanwhile, the black element-of
relation, $\in$, can ``see into'' black sets $\{ \ldots \}$, but not
red ones. Thus, the black sets can serve as atoms for the red sets,
while the red sets can serve as atoms for the black sets.

So, the two theories are mutually recursively defined. This mutual
recursion is closely linked to game semantics in the sense of
Abramksy, Hyland, and Ong, et al. Without loss of generality, we can
think of \textcolor{red}{red} as player and black as opponent. A
well-formed red set, \textcolor{red}{$S$}, is a winning strategy for
player, while a well-formed black set, $S$, is a winning strategy for
opponent.


\section{Formal theory}
The judgment \(\Sigma; \Gamma \vdash S\) is pronounced
``\(\Sigma\) thinks that \(S\) is well-formed, given
dependencies, \(\Gamma\) .'' Similarly,
\(\Sigma; \Gamma \vdash x\) is pronounced
``\(\Sigma\) thinks that \(x\) is an admissible atom, given
dependencies, \(\Gamma\).''

We think of $X$ as some infinite supply of ``atoms'' and
$\mathsf{S}[X]$ as a \emph{stream} of atoms drawn from $X$. We assume
basic stream operations on $\mathsf{S}[X]$, such as
$\mathsf{take}(\mathsf{S}[X],n)$ which returns the pair
$(\sigma,\Sigma)$ where $\sigma = x_{1}:\ldots:x_{n}$ and $\Sigma = \mathsf{drop}(\mathsf{S}[X],n)$. Given $\rho = \{j_{1}/i_{1}:\ldots:j_{n}/i_{n}\}$ the operation
$\mathsf{swap}(\mathsf{S}[X],\rho)$ denotes the application of the
permutation $\rho$ to $\mathsf{S}[X]$.

Thus, $\mathsf{S}[X]$ may be thought of as an ordering on $X$.

A rule of the form

\[\frac{ H_1, \ldots , H_n }{ C }R\]

is pronounced ``\(R\) concludes that \(C\) given \(H_1\), \(\ldots\),
\(H_n\)''.

We use $e,f,\ldots$ to range over sets and atoms.

The rules for judging when an atom is admissible or set is well
formed are as follows

\[\frac{ }{ \mathsf{S}[X]; () \vdash \emptyset}Foundation\]

\[\frac{ (x,\Sigma) = \mathsf{take}(\mathsf{S}[X],1)}{ \Sigma; x \vdash \{ x \}}Atomicity\]

\[\frac{ \Sigma; \Gamma \vdash S }{ \Sigma; \Gamma \vdash \{ S \}}Nesting\]

\[\frac{ \Sigma; \Gamma_1 \vdash S_1 \; \Sigma; \Gamma_2 \vdash S_2}{ \Sigma; \Gamma_1, \Gamma_2 \vdash S_1 \cup S_2}Union\]

\[\frac{ \Sigma; \Gamma_1 \vdash S_1 \; \Sigma; \Gamma_2 \vdash S_2}{ \Sigma; \Gamma_1, \Gamma_2 \vdash S_1 \cap S_2}Intersection\]

\[\frac{ \Sigma; \Gamma_1 \vdash S_1 \; \Sigma; \Gamma_2 \vdash S_2}{ \Sigma; \Gamma_1, \Gamma_2 \vdash S_1 \setminus S_2}Subtraction\]

\[\frac{ \Sigma; \Gamma \vdash S \; \Sigma; \Gamma_{1} \vdash S_{1} \;\ldots\; \Sigma; \Gamma_{n} \vdash S_{2}\; \Sigma; \Delta \vdash f : S_{1} \times \ldots \times S_{n} \to S }{ \Sigma; \Gamma,\Gamma_{1},\ldots, \Gamma_{n},\Delta \vdash \{ f(s_{1},\ldots,s_{n}) \; : \; s_1 \in S_{1},\; \ldots\;, s_{n} \in S_{n}\}}Comprehension\]

\subsection{Equations}\label{equations}

The syntactic theory is too fine grained. It makes syntactic
distinctions that do not correspond to distinct sets. We erase these
syntactic distinctions with a set of equations on set expressions.

\[\frac{ \Sigma; \Gamma \vdash e }{ \Sigma; \Gamma  \vdash e = e}Identity\]
\[\frac{\Sigma; \Gamma_{1} \vdash _{1}\; \Sigma; \Gamma_{2} \vdash e_{2}\; \Sigma; \Gamma_{3}  \vdash e_{1} = e_{2}}{ \Sigma; \Gamma_{3}  \vdash e_{1} = e_{2}}Symmetry\]
\[\frac{\Sigma; \Gamma_{1}  \vdash e_{1}\; \Sigma; \Gamma_{2}  \vdash e_{2}\; \Sigma; \Gamma_{3} \vdash e_{3}\; \Sigma; \Gamma_{4}  \vdash e_{1} = e_{2} \; \Sigma; \Gamma_{5}  \vdash e_{2} = e_{3}}{ \Sigma; \Gamma_{4},\Gamma_{5}  \vdash e_{1} = e_{3}}Transitivity\]

\[\frac{\Sigma; \Gamma  \vdash S}{\Sigma; \Gamma \vdash S \cup S =  S}Idempotence_{\cup}\]
\[\frac{\Sigma; \Gamma  \vdash S}{\Sigma; \Gamma \vdash S \cap S =  S}Idempotence_{\cap}\]




\documentclass[12pt]{llncs}

\usepackage[pdftex]{hyperref}                   
\usepackage {mathpartir}
\usepackage{bigpage}
\usepackage{bcprules}
\usepackage {listings}
\usepackage{amsmath}
\usepackage{amssymb}
\usepackage{amsfonts}
\usepackage{amstext}
\usepackage{latexsym}
\usepackage{longtable}
\usepackage{stmaryrd}
%\usepackage{fdsymbol}
\usepackage{graphicx} 
%\usepackage[margins=2.5cm,nohead,nofoot]{geometry}
%\usepackage{geometry}
\usepackage{color}
\usepackage{xcolor}


%\include{myPreamble}
\documentclass[12pt]{llncs}

\usepackage[pdftex]{hyperref}                   
\usepackage {mathpartir}
\usepackage{bigpage}
\usepackage{bcprules}
\usepackage {listings}
\usepackage{amsmath}
\usepackage{amssymb}
\usepackage{amsfonts}
\usepackage{amstext}
\usepackage{latexsym}
\usepackage{longtable}
\usepackage{stmaryrd}
%\usepackage{fdsymbol}
\usepackage{graphicx} 
%\usepackage[margins=2.5cm,nohead,nofoot]{geometry}
%\usepackage{geometry}
\usepackage{color}
\usepackage{xcolor}


%\include{myPreamble}
\include{rset.local} 

%\ifpdf
%\usepackage[pdftex]{graphicx}
%\else
%\usepackage{graphicx}
%\fi

 % \ifpdf
%  \usepackage{pdfsync}
%  \if


%\title{Brief Article}
%\author{David F. Snyder}
%\author{L.G. Meredith}

%\address{Dept. of Math., Texas State University--San Marcos, San Marcos, TX 78666}
       
\pagestyle{empty}


\begin{document}

\lstset{language=[Objective]Caml,frame=shadowbox}


\input{rset.front}

In the early 2000's Pitts and Gabbay's use of Fraenkl-Mostowski set
theory (FM set theory), a theory of sets with atoms, to model nominal phenomena,
sparked a vibrant line of research. \cite{DBLP:journals/fac/GabbayP02}
\cite{DBLP:journals/jcss/Clouston14}. As Chaitin points out, the
axioms and assumptions of a theory comprise its risk
\cite{chaitin1999unknowable}. This observation raises an interesting
point: is it possible to have a set theory with atoms without taking
on \emph{infinite} risk? It turns out it is.

We describe a theory of sets that enjoys both the cleanliness of
$\mathsf{ZF}$, that is arising only from the empty set, and yet with
an infinite supply of atoms.

\input{rset.intuitions}

\input{rset.presentation}

\input{rset.stdfmset}

\input{rset.mainthm}

\input{rset.conc}

\input{rset.ack}

\bibliographystyle{plain}   
\bibliography{rset.bib}

\end{document}
 

%\ifpdf
%\usepackage[pdftex]{graphicx}
%\else
%\usepackage{graphicx}
%\fi

 % \ifpdf
%  \usepackage{pdfsync}
%  \if


%\title{Brief Article}
%\author{David F. Snyder}
%\author{L.G. Meredith}

%\address{Dept. of Math., Texas State University--San Marcos, San Marcos, TX 78666}
       
\pagestyle{empty}


\begin{document}

\lstset{language=[Objective]Caml,frame=shadowbox}


\def\lastname{Meredith}

\title{Notes on a reflective theory of sets}
\titlerunning{oslf}

\author{ Lucius Gregory Meredith\inst{1}
}
\institute{
        {Managing Partner, RTech Unchained 9336 California Ave SW, Seattle, WA 98103, USA} \\
  \email{ lgreg.meredith@gmail.com } 
}
 

\maketitle              % typeset the title of the contribution

%%% ----------------------------------------------------------------------

\begin{abstract}

  We describe a universe of set theories enjoying atoms without enduring infinite risk.

\end{abstract}

%\keywords{Process calculi, knots, invariants}

% \begin{keyword}
% concurrency, message-passing, process calculus, reflection, program logic
% \end{keyword}

%\end{frontmatter}


In the early 2000's Pitts and Gabbay's use of Fraenkl-Mostowski set
theory (FM set theory), a theory of sets with atoms, to model nominal phenomena,
sparked a vibrant line of research. \cite{DBLP:journals/fac/GabbayP02}
\cite{DBLP:journals/jcss/Clouston14}. As Chaitin points out, the
axioms and assumptions of a theory comprise its risk
\cite{chaitin1999unknowable}. This observation raises an interesting
point: is it possible to have a set theory with atoms without taking
on \emph{infinite} risk? It turns out it is.

We describe a theory of sets that enjoys both the cleanliness of
$\mathsf{ZF}$, that is arising only from the empty set, and yet with
an infinite supply of atoms.

\section{Intuitions}
We imagine two copies of FM set theory, say a \textcolor{red}{red one}
and a black one. The \textcolor{red}{red theory} has red
comprehensions,\textcolor{red}{$\{ \ldots \}$} and a red element-of
relation, \textcolor{red}{$\in$}, while the black theory,
correspondingly has black comprehensions, $\{ \ldots \}$ and a black
element-of relation, $\in$. The red element-of relation,
\textcolor{red}{$\in$}, can ``see into'' red sets, \textcolor{red}{$\{
  \ldots \}$}, but not black ones. Meanwhile, the black element-of
relation, $\in$, can ``see into'' black sets $\{ \ldots \}$, but not
red ones. Thus, the black sets can serve as atoms for the red sets,
while the red sets can serve as atoms for the black sets.

So, the two theories are mutually recursively defined. This mutual
recursion is closely linked to game semantics in the sense of
Abramksy, Hyland, and Ong, et al. Without loss of generality, we can
think of \textcolor{red}{red} as player and black as opponent. A
well-formed red set, \textcolor{red}{$S$}, is a winning strategy for
player, while a well-formed black set, $S$, is a winning strategy for
opponent.


\section{Formal theory}
The judgment \(\Sigma; \Gamma \vdash S\) is pronounced
``\(\Sigma\) thinks that \(S\) is well-formed, given
dependencies, \(\Gamma\) .'' Similarly,
\(\Sigma; \Gamma \vdash x\) is pronounced
``\(\Sigma\) thinks that \(x\) is an admissible atom, given
dependencies, \(\Gamma\).''

We think of $X$ as some infinite supply of ``atoms'' and
$\mathsf{S}[X]$ as a \emph{stream} of atoms drawn from $X$. We assume
basic stream operations on $\mathsf{S}[X]$, such as
$\mathsf{take}(\mathsf{S}[X],n)$ which returns the pair
$(\sigma,\Sigma)$ where $\sigma = x_{1}:\ldots:x_{n}$ and $\Sigma = \mathsf{drop}(\mathsf{S}[X],n)$. Given $\rho = \{j_{1}/i_{1}:\ldots:j_{n}/i_{n}\}$ the operation
$\mathsf{swap}(\mathsf{S}[X],\rho)$ denotes the application of the
permutation $\rho$ to $\mathsf{S}[X]$.

Thus, $\mathsf{S}[X]$ may be thought of as an ordering on $X$.

A rule of the form

\[\frac{ H_1, \ldots , H_n }{ C }R\]

is pronounced ``\(R\) concludes that \(C\) given \(H_1\), \(\ldots\),
\(H_n\)''.

We use $e,f,\ldots$ to range over sets and atoms.

The rules for judging when an atom is admissible or set is well
formed are as follows

\[\frac{ }{ \mathsf{S}[X]; () \vdash \emptyset}Foundation\]

\[\frac{ (x,\Sigma) = \mathsf{take}(\mathsf{S}[X],1)}{ \Sigma; x \vdash \{ x \}}Atomicity\]

\[\frac{ \Sigma; \Gamma \vdash S }{ \Sigma; \Gamma \vdash \{ S \}}Nesting\]

\[\frac{ \Sigma; \Gamma_1 \vdash S_1 \; \Sigma; \Gamma_2 \vdash S_2}{ \Sigma; \Gamma_1, \Gamma_2 \vdash S_1 \cup S_2}Union\]

\[\frac{ \Sigma; \Gamma_1 \vdash S_1 \; \Sigma; \Gamma_2 \vdash S_2}{ \Sigma; \Gamma_1, \Gamma_2 \vdash S_1 \cap S_2}Intersection\]

\[\frac{ \Sigma; \Gamma_1 \vdash S_1 \; \Sigma; \Gamma_2 \vdash S_2}{ \Sigma; \Gamma_1, \Gamma_2 \vdash S_1 \setminus S_2}Subtraction\]

\[\frac{ \Sigma; \Gamma \vdash S \; \Sigma; \Gamma_{1} \vdash S_{1} \;\ldots\; \Sigma; \Gamma_{n} \vdash S_{2}\; \Sigma; \Delta \vdash f : S_{1} \times \ldots \times S_{n} \to S }{ \Sigma; \Gamma,\Gamma_{1},\ldots, \Gamma_{n},\Delta \vdash \{ f(s_{1},\ldots,s_{n}) \; : \; s_1 \in S_{1},\; \ldots\;, s_{n} \in S_{n}\}}Comprehension\]

\subsection{Equations}\label{equations}

The syntactic theory is too fine grained. It makes syntactic
distinctions that do not correspond to distinct sets. We erase these
syntactic distinctions with a set of equations on set expressions.

\[\frac{ \Sigma; \Gamma \vdash e }{ \Sigma; \Gamma  \vdash e = e}Identity\]
\[\frac{\Sigma; \Gamma_{1} \vdash _{1}\; \Sigma; \Gamma_{2} \vdash e_{2}\; \Sigma; \Gamma_{3}  \vdash e_{1} = e_{2}}{ \Sigma; \Gamma_{3}  \vdash e_{1} = e_{2}}Symmetry\]
\[\frac{\Sigma; \Gamma_{1}  \vdash e_{1}\; \Sigma; \Gamma_{2}  \vdash e_{2}\; \Sigma; \Gamma_{3} \vdash e_{3}\; \Sigma; \Gamma_{4}  \vdash e_{1} = e_{2} \; \Sigma; \Gamma_{5}  \vdash e_{2} = e_{3}}{ \Sigma; \Gamma_{4},\Gamma_{5}  \vdash e_{1} = e_{3}}Transitivity\]

\[\frac{\Sigma; \Gamma  \vdash S}{\Sigma; \Gamma \vdash S \cup S =  S}Idempotence_{\cup}\]
\[\frac{\Sigma; \Gamma  \vdash S}{\Sigma; \Gamma \vdash S \cap S =  S}Idempotence_{\cap}\]




\documentclass[12pt]{llncs}

\usepackage[pdftex]{hyperref}                   
\usepackage {mathpartir}
\usepackage{bigpage}
\usepackage{bcprules}
\usepackage {listings}
\usepackage{amsmath}
\usepackage{amssymb}
\usepackage{amsfonts}
\usepackage{amstext}
\usepackage{latexsym}
\usepackage{longtable}
\usepackage{stmaryrd}
%\usepackage{fdsymbol}
\usepackage{graphicx} 
%\usepackage[margins=2.5cm,nohead,nofoot]{geometry}
%\usepackage{geometry}
\usepackage{color}
\usepackage{xcolor}


%\include{myPreamble}
\include{rset.local} 

%\ifpdf
%\usepackage[pdftex]{graphicx}
%\else
%\usepackage{graphicx}
%\fi

 % \ifpdf
%  \usepackage{pdfsync}
%  \if


%\title{Brief Article}
%\author{David F. Snyder}
%\author{L.G. Meredith}

%\address{Dept. of Math., Texas State University--San Marcos, San Marcos, TX 78666}
       
\pagestyle{empty}


\begin{document}

\lstset{language=[Objective]Caml,frame=shadowbox}


\input{rset.front}

In the early 2000's Pitts and Gabbay's use of Fraenkl-Mostowski set
theory (FM set theory), a theory of sets with atoms, to model nominal phenomena,
sparked a vibrant line of research. \cite{DBLP:journals/fac/GabbayP02}
\cite{DBLP:journals/jcss/Clouston14}. As Chaitin points out, the
axioms and assumptions of a theory comprise its risk
\cite{chaitin1999unknowable}. This observation raises an interesting
point: is it possible to have a set theory with atoms without taking
on \emph{infinite} risk? It turns out it is.

We describe a theory of sets that enjoys both the cleanliness of
$\mathsf{ZF}$, that is arising only from the empty set, and yet with
an infinite supply of atoms.

\input{rset.intuitions}

\input{rset.presentation}

\input{rset.stdfmset}

\input{rset.mainthm}

\input{rset.conc}

\input{rset.ack}

\bibliographystyle{plain}   
\bibliography{rset.bib}

\end{document}


\documentclass[12pt]{llncs}

\usepackage[pdftex]{hyperref}                   
\usepackage {mathpartir}
\usepackage{bigpage}
\usepackage{bcprules}
\usepackage {listings}
\usepackage{amsmath}
\usepackage{amssymb}
\usepackage{amsfonts}
\usepackage{amstext}
\usepackage{latexsym}
\usepackage{longtable}
\usepackage{stmaryrd}
%\usepackage{fdsymbol}
\usepackage{graphicx} 
%\usepackage[margins=2.5cm,nohead,nofoot]{geometry}
%\usepackage{geometry}
\usepackage{color}
\usepackage{xcolor}


%\include{myPreamble}
\include{rset.local} 

%\ifpdf
%\usepackage[pdftex]{graphicx}
%\else
%\usepackage{graphicx}
%\fi

 % \ifpdf
%  \usepackage{pdfsync}
%  \if


%\title{Brief Article}
%\author{David F. Snyder}
%\author{L.G. Meredith}

%\address{Dept. of Math., Texas State University--San Marcos, San Marcos, TX 78666}
       
\pagestyle{empty}


\begin{document}

\lstset{language=[Objective]Caml,frame=shadowbox}


\input{rset.front}

In the early 2000's Pitts and Gabbay's use of Fraenkl-Mostowski set
theory (FM set theory), a theory of sets with atoms, to model nominal phenomena,
sparked a vibrant line of research. \cite{DBLP:journals/fac/GabbayP02}
\cite{DBLP:journals/jcss/Clouston14}. As Chaitin points out, the
axioms and assumptions of a theory comprise its risk
\cite{chaitin1999unknowable}. This observation raises an interesting
point: is it possible to have a set theory with atoms without taking
on \emph{infinite} risk? It turns out it is.

We describe a theory of sets that enjoys both the cleanliness of
$\mathsf{ZF}$, that is arising only from the empty set, and yet with
an infinite supply of atoms.

\input{rset.intuitions}

\input{rset.presentation}

\input{rset.stdfmset}

\input{rset.mainthm}

\input{rset.conc}

\input{rset.ack}

\bibliographystyle{plain}   
\bibliography{rset.bib}

\end{document}


\section{Conclusions and future work}

We have presented a formal theory of sets that reconciles set theory with atoms with a set theory built only from the empty set.

% subsection other_calculi_other_bisimulations_and_geometry_as_behavior (end)




\paragraph{Acknowledgments.}
The author wishes to thank Jamie Gabbay for stimulating discussions
that led the author to devise this theory.


\bibliographystyle{plain}   
\bibliography{rset.bib}

\end{document}


\documentclass[12pt]{llncs}

\usepackage[pdftex]{hyperref}                   
\usepackage {mathpartir}
\usepackage{bigpage}
\usepackage{bcprules}
\usepackage {listings}
\usepackage{amsmath}
\usepackage{amssymb}
\usepackage{amsfonts}
\usepackage{amstext}
\usepackage{latexsym}
\usepackage{longtable}
\usepackage{stmaryrd}
%\usepackage{fdsymbol}
\usepackage{graphicx} 
%\usepackage[margins=2.5cm,nohead,nofoot]{geometry}
%\usepackage{geometry}
\usepackage{color}
\usepackage{xcolor}


%\include{myPreamble}
\documentclass[12pt]{llncs}

\usepackage[pdftex]{hyperref}                   
\usepackage {mathpartir}
\usepackage{bigpage}
\usepackage{bcprules}
\usepackage {listings}
\usepackage{amsmath}
\usepackage{amssymb}
\usepackage{amsfonts}
\usepackage{amstext}
\usepackage{latexsym}
\usepackage{longtable}
\usepackage{stmaryrd}
%\usepackage{fdsymbol}
\usepackage{graphicx} 
%\usepackage[margins=2.5cm,nohead,nofoot]{geometry}
%\usepackage{geometry}
\usepackage{color}
\usepackage{xcolor}


%\include{myPreamble}
\include{rset.local} 

%\ifpdf
%\usepackage[pdftex]{graphicx}
%\else
%\usepackage{graphicx}
%\fi

 % \ifpdf
%  \usepackage{pdfsync}
%  \if


%\title{Brief Article}
%\author{David F. Snyder}
%\author{L.G. Meredith}

%\address{Dept. of Math., Texas State University--San Marcos, San Marcos, TX 78666}
       
\pagestyle{empty}


\begin{document}

\lstset{language=[Objective]Caml,frame=shadowbox}


\input{rset.front}

In the early 2000's Pitts and Gabbay's use of Fraenkl-Mostowski set
theory (FM set theory), a theory of sets with atoms, to model nominal phenomena,
sparked a vibrant line of research. \cite{DBLP:journals/fac/GabbayP02}
\cite{DBLP:journals/jcss/Clouston14}. As Chaitin points out, the
axioms and assumptions of a theory comprise its risk
\cite{chaitin1999unknowable}. This observation raises an interesting
point: is it possible to have a set theory with atoms without taking
on \emph{infinite} risk? It turns out it is.

We describe a theory of sets that enjoys both the cleanliness of
$\mathsf{ZF}$, that is arising only from the empty set, and yet with
an infinite supply of atoms.

\input{rset.intuitions}

\input{rset.presentation}

\input{rset.stdfmset}

\input{rset.mainthm}

\input{rset.conc}

\input{rset.ack}

\bibliographystyle{plain}   
\bibliography{rset.bib}

\end{document}
 

%\ifpdf
%\usepackage[pdftex]{graphicx}
%\else
%\usepackage{graphicx}
%\fi

 % \ifpdf
%  \usepackage{pdfsync}
%  \if


%\title{Brief Article}
%\author{David F. Snyder}
%\author{L.G. Meredith}

%\address{Dept. of Math., Texas State University--San Marcos, San Marcos, TX 78666}
       
\pagestyle{empty}


\begin{document}

\lstset{language=[Objective]Caml,frame=shadowbox}


\def\lastname{Meredith}

\title{Notes on a reflective theory of sets}
\titlerunning{oslf}

\author{ Lucius Gregory Meredith\inst{1}
}
\institute{
        {Managing Partner, RTech Unchained 9336 California Ave SW, Seattle, WA 98103, USA} \\
  \email{ lgreg.meredith@gmail.com } 
}
 

\maketitle              % typeset the title of the contribution

%%% ----------------------------------------------------------------------

\begin{abstract}

  We describe a universe of set theories enjoying atoms without enduring infinite risk.

\end{abstract}

%\keywords{Process calculi, knots, invariants}

% \begin{keyword}
% concurrency, message-passing, process calculus, reflection, program logic
% \end{keyword}

%\end{frontmatter}


In the early 2000's Pitts and Gabbay's use of Fraenkl-Mostowski set
theory (FM set theory), a theory of sets with atoms, to model nominal phenomena,
sparked a vibrant line of research. \cite{DBLP:journals/fac/GabbayP02}
\cite{DBLP:journals/jcss/Clouston14}. As Chaitin points out, the
axioms and assumptions of a theory comprise its risk
\cite{chaitin1999unknowable}. This observation raises an interesting
point: is it possible to have a set theory with atoms without taking
on \emph{infinite} risk? It turns out it is.

We describe a theory of sets that enjoys both the cleanliness of
$\mathsf{ZF}$, that is arising only from the empty set, and yet with
an infinite supply of atoms.

\section{Intuitions}
We imagine two copies of FM set theory, say a \textcolor{red}{red one}
and a black one. The \textcolor{red}{red theory} has red
comprehensions,\textcolor{red}{$\{ \ldots \}$} and a red element-of
relation, \textcolor{red}{$\in$}, while the black theory,
correspondingly has black comprehensions, $\{ \ldots \}$ and a black
element-of relation, $\in$. The red element-of relation,
\textcolor{red}{$\in$}, can ``see into'' red sets, \textcolor{red}{$\{
  \ldots \}$}, but not black ones. Meanwhile, the black element-of
relation, $\in$, can ``see into'' black sets $\{ \ldots \}$, but not
red ones. Thus, the black sets can serve as atoms for the red sets,
while the red sets can serve as atoms for the black sets.

So, the two theories are mutually recursively defined. This mutual
recursion is closely linked to game semantics in the sense of
Abramksy, Hyland, and Ong, et al. Without loss of generality, we can
think of \textcolor{red}{red} as player and black as opponent. A
well-formed red set, \textcolor{red}{$S$}, is a winning strategy for
player, while a well-formed black set, $S$, is a winning strategy for
opponent.


\section{Formal theory}
The judgment \(\Sigma; \Gamma \vdash S\) is pronounced
``\(\Sigma\) thinks that \(S\) is well-formed, given
dependencies, \(\Gamma\) .'' Similarly,
\(\Sigma; \Gamma \vdash x\) is pronounced
``\(\Sigma\) thinks that \(x\) is an admissible atom, given
dependencies, \(\Gamma\).''

We think of $X$ as some infinite supply of ``atoms'' and
$\mathsf{S}[X]$ as a \emph{stream} of atoms drawn from $X$. We assume
basic stream operations on $\mathsf{S}[X]$, such as
$\mathsf{take}(\mathsf{S}[X],n)$ which returns the pair
$(\sigma,\Sigma)$ where $\sigma = x_{1}:\ldots:x_{n}$ and $\Sigma = \mathsf{drop}(\mathsf{S}[X],n)$. Given $\rho = \{j_{1}/i_{1}:\ldots:j_{n}/i_{n}\}$ the operation
$\mathsf{swap}(\mathsf{S}[X],\rho)$ denotes the application of the
permutation $\rho$ to $\mathsf{S}[X]$.

Thus, $\mathsf{S}[X]$ may be thought of as an ordering on $X$.

A rule of the form

\[\frac{ H_1, \ldots , H_n }{ C }R\]

is pronounced ``\(R\) concludes that \(C\) given \(H_1\), \(\ldots\),
\(H_n\)''.

We use $e,f,\ldots$ to range over sets and atoms.

The rules for judging when an atom is admissible or set is well
formed are as follows

\[\frac{ }{ \mathsf{S}[X]; () \vdash \emptyset}Foundation\]

\[\frac{ (x,\Sigma) = \mathsf{take}(\mathsf{S}[X],1)}{ \Sigma; x \vdash \{ x \}}Atomicity\]

\[\frac{ \Sigma; \Gamma \vdash S }{ \Sigma; \Gamma \vdash \{ S \}}Nesting\]

\[\frac{ \Sigma; \Gamma_1 \vdash S_1 \; \Sigma; \Gamma_2 \vdash S_2}{ \Sigma; \Gamma_1, \Gamma_2 \vdash S_1 \cup S_2}Union\]

\[\frac{ \Sigma; \Gamma_1 \vdash S_1 \; \Sigma; \Gamma_2 \vdash S_2}{ \Sigma; \Gamma_1, \Gamma_2 \vdash S_1 \cap S_2}Intersection\]

\[\frac{ \Sigma; \Gamma_1 \vdash S_1 \; \Sigma; \Gamma_2 \vdash S_2}{ \Sigma; \Gamma_1, \Gamma_2 \vdash S_1 \setminus S_2}Subtraction\]

\[\frac{ \Sigma; \Gamma \vdash S \; \Sigma; \Gamma_{1} \vdash S_{1} \;\ldots\; \Sigma; \Gamma_{n} \vdash S_{2}\; \Sigma; \Delta \vdash f : S_{1} \times \ldots \times S_{n} \to S }{ \Sigma; \Gamma,\Gamma_{1},\ldots, \Gamma_{n},\Delta \vdash \{ f(s_{1},\ldots,s_{n}) \; : \; s_1 \in S_{1},\; \ldots\;, s_{n} \in S_{n}\}}Comprehension\]

\subsection{Equations}\label{equations}

The syntactic theory is too fine grained. It makes syntactic
distinctions that do not correspond to distinct sets. We erase these
syntactic distinctions with a set of equations on set expressions.

\[\frac{ \Sigma; \Gamma \vdash e }{ \Sigma; \Gamma  \vdash e = e}Identity\]
\[\frac{\Sigma; \Gamma_{1} \vdash _{1}\; \Sigma; \Gamma_{2} \vdash e_{2}\; \Sigma; \Gamma_{3}  \vdash e_{1} = e_{2}}{ \Sigma; \Gamma_{3}  \vdash e_{1} = e_{2}}Symmetry\]
\[\frac{\Sigma; \Gamma_{1}  \vdash e_{1}\; \Sigma; \Gamma_{2}  \vdash e_{2}\; \Sigma; \Gamma_{3} \vdash e_{3}\; \Sigma; \Gamma_{4}  \vdash e_{1} = e_{2} \; \Sigma; \Gamma_{5}  \vdash e_{2} = e_{3}}{ \Sigma; \Gamma_{4},\Gamma_{5}  \vdash e_{1} = e_{3}}Transitivity\]

\[\frac{\Sigma; \Gamma  \vdash S}{\Sigma; \Gamma \vdash S \cup S =  S}Idempotence_{\cup}\]
\[\frac{\Sigma; \Gamma  \vdash S}{\Sigma; \Gamma \vdash S \cap S =  S}Idempotence_{\cap}\]




\documentclass[12pt]{llncs}

\usepackage[pdftex]{hyperref}                   
\usepackage {mathpartir}
\usepackage{bigpage}
\usepackage{bcprules}
\usepackage {listings}
\usepackage{amsmath}
\usepackage{amssymb}
\usepackage{amsfonts}
\usepackage{amstext}
\usepackage{latexsym}
\usepackage{longtable}
\usepackage{stmaryrd}
%\usepackage{fdsymbol}
\usepackage{graphicx} 
%\usepackage[margins=2.5cm,nohead,nofoot]{geometry}
%\usepackage{geometry}
\usepackage{color}
\usepackage{xcolor}


%\include{myPreamble}
\include{rset.local} 

%\ifpdf
%\usepackage[pdftex]{graphicx}
%\else
%\usepackage{graphicx}
%\fi

 % \ifpdf
%  \usepackage{pdfsync}
%  \if


%\title{Brief Article}
%\author{David F. Snyder}
%\author{L.G. Meredith}

%\address{Dept. of Math., Texas State University--San Marcos, San Marcos, TX 78666}
       
\pagestyle{empty}


\begin{document}

\lstset{language=[Objective]Caml,frame=shadowbox}


\input{rset.front}

In the early 2000's Pitts and Gabbay's use of Fraenkl-Mostowski set
theory (FM set theory), a theory of sets with atoms, to model nominal phenomena,
sparked a vibrant line of research. \cite{DBLP:journals/fac/GabbayP02}
\cite{DBLP:journals/jcss/Clouston14}. As Chaitin points out, the
axioms and assumptions of a theory comprise its risk
\cite{chaitin1999unknowable}. This observation raises an interesting
point: is it possible to have a set theory with atoms without taking
on \emph{infinite} risk? It turns out it is.

We describe a theory of sets that enjoys both the cleanliness of
$\mathsf{ZF}$, that is arising only from the empty set, and yet with
an infinite supply of atoms.

\input{rset.intuitions}

\input{rset.presentation}

\input{rset.stdfmset}

\input{rset.mainthm}

\input{rset.conc}

\input{rset.ack}

\bibliographystyle{plain}   
\bibliography{rset.bib}

\end{document}


\documentclass[12pt]{llncs}

\usepackage[pdftex]{hyperref}                   
\usepackage {mathpartir}
\usepackage{bigpage}
\usepackage{bcprules}
\usepackage {listings}
\usepackage{amsmath}
\usepackage{amssymb}
\usepackage{amsfonts}
\usepackage{amstext}
\usepackage{latexsym}
\usepackage{longtable}
\usepackage{stmaryrd}
%\usepackage{fdsymbol}
\usepackage{graphicx} 
%\usepackage[margins=2.5cm,nohead,nofoot]{geometry}
%\usepackage{geometry}
\usepackage{color}
\usepackage{xcolor}


%\include{myPreamble}
\include{rset.local} 

%\ifpdf
%\usepackage[pdftex]{graphicx}
%\else
%\usepackage{graphicx}
%\fi

 % \ifpdf
%  \usepackage{pdfsync}
%  \if


%\title{Brief Article}
%\author{David F. Snyder}
%\author{L.G. Meredith}

%\address{Dept. of Math., Texas State University--San Marcos, San Marcos, TX 78666}
       
\pagestyle{empty}


\begin{document}

\lstset{language=[Objective]Caml,frame=shadowbox}


\input{rset.front}

In the early 2000's Pitts and Gabbay's use of Fraenkl-Mostowski set
theory (FM set theory), a theory of sets with atoms, to model nominal phenomena,
sparked a vibrant line of research. \cite{DBLP:journals/fac/GabbayP02}
\cite{DBLP:journals/jcss/Clouston14}. As Chaitin points out, the
axioms and assumptions of a theory comprise its risk
\cite{chaitin1999unknowable}. This observation raises an interesting
point: is it possible to have a set theory with atoms without taking
on \emph{infinite} risk? It turns out it is.

We describe a theory of sets that enjoys both the cleanliness of
$\mathsf{ZF}$, that is arising only from the empty set, and yet with
an infinite supply of atoms.

\input{rset.intuitions}

\input{rset.presentation}

\input{rset.stdfmset}

\input{rset.mainthm}

\input{rset.conc}

\input{rset.ack}

\bibliographystyle{plain}   
\bibliography{rset.bib}

\end{document}


\section{Conclusions and future work}

We have presented a formal theory of sets that reconciles set theory with atoms with a set theory built only from the empty set.

% subsection other_calculi_other_bisimulations_and_geometry_as_behavior (end)




\paragraph{Acknowledgments.}
The author wishes to thank Jamie Gabbay for stimulating discussions
that led the author to devise this theory.


\bibliographystyle{plain}   
\bibliography{rset.bib}

\end{document}


\section{Conclusions and future work}

We have presented a formal theory of sets that reconciles set theory with atoms with a set theory built only from the empty set.

% subsection other_calculi_other_bisimulations_and_geometry_as_behavior (end)




\paragraph{Acknowledgments.}
The author wishes to thank Jamie Gabbay for stimulating discussions
that led the author to devise this theory.


\bibliographystyle{plain}   
\bibliography{rset.bib}

\end{document}


\section{Conclusions and future work}

We have presented a formal theory of sets that reconciles set theory with atoms with a set theory built only from the empty set.

% subsection other_calculi_other_bisimulations_and_geometry_as_behavior (end)




\paragraph{Acknowledgments.}
The author wishes to thank Jamie Gabbay for stimulating discussions
that led the author to devise this theory.


\bibliographystyle{plain}   
\bibliography{rset.bib}

\end{document}
