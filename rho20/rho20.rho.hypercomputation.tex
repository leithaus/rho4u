\section{Hypercomputation and rho}
In \cite{WIKI:TuringHyperComputation} Turing already recognized a
tower of computation beyond the Turing machine. It is widely thought
that these more powerful systems are not physically realizable. The
rho calculus, and the mobile process calculi more generally, offer a
different perspective on this question.

The story begins from a bit of trolling that Carl Hewitt used to
indulge in. Hewitt was aware of the result that fair communication is
strictly stronger than Turing machines, and because Hewitt's actor
model presumes fair communication, to get a rise out of his audience
Hewitt would regularly say that actors are more powerful than Turing
machines.

Yet, there is an insight here that should not go unremarked. In more
recent years the field of computing has recognized randomness,
entropy, and non-determinism as key resources in computing. In
particular, introducing this kind of resource can dramatically speed
up certain computations. In the context of Turing's work and Hewitt's
trolling the question arises why could we not arrange for ordinal-oracles to
be the source of the ``non-determinism'' in computations?

More specifically, in the case of the mobile process calculi there is
effectively only one place where non-determinism arises: race
conditions. Why then could we not arrange for oracles to supply
information about the winners of races? Even more specifically, using
the annihilation version of the $\mathsf{COMM}$ rule we can arrange
for the complexity of a computation to determine the winners of
races. That is, we can arrange for complexity-based racing in which
computations with lower complexity win races on average. Oracles which
can compute in finite time limits that would require infinite steps or
infinite storage computation will often win these complexity-based
races. Moreover, this approach is compositional. At each higher
ordinal we get new oracles which deliver new answers about the winners
of races at the current ``level'' of this tower of hypercomputation.

Even further, the rho calculus can track the flow of causal
information. Specifically, we can track the tree of $\mathsf{COMM}$
events. In some sense, this is the rho calculus equivalent of a light
cone. Yet, in this tower of hypercomputation we can track when oracles
are causing events at lower levels of the tower. Indeed, oracles at
any level of the tower may influence events at any of the lower levels.

Under the hypothesis that physicality \emph{is} causality we arrive at
a startling conclusion that the ``physically'' computable universe is
open, rather than closed. There is a place for limitless novelty and
this novelty can influence ``physical'' computations, changing the
course of what might otherwise be a completely deterministic chain of
events.
