\section{Stochastic and quantum rho}

If we follow the route followed by the development of the stochastic
$\pi$-calculus, then the core idea is that we can associate to each
channel (type) a rate. The rate determines the relative frequency of
interaction and thus, just as with standard practice in chemistry, we
can use a form of the Gillespie algorithm to turn process expressions
decorated with channel rates into simulations.

Somewhat surprisingly, this technique extends to quantum
simulations. We use complex numbers for rates and there is a
corresponding quantum version of Gillespie that turns process
expressions decorated with complex valued channel rates into
simulations. Under this interpretation the $\mathsf{COMM}$ rule becomes
a version of the Born rule.

However, there are different approaches to consider which we will
return to after providing the details of the more standard approaches.

\subsection{Stochastic rho}

TBD

\subsection{Quantum rho}

TBD

\subsection{Stochasticity and annihilation}

TBD

\subsection{Quantum mechanics directly encoded}

TBD
