\section{Stochastic and quantum rho}

If we follow the route followed by the development of the stochastic
$\pi$-calculus, then the core idea is that we can associate to each
channel (type) a rate. The rate determines the relative frequency of
interaction and thus, just as with standard practice in chemistry, we
can use a form of the Gillespie algorithm to turn process expressions
decorated with channel rates into simulations.

Somewhat surprisingly, this technique extends to quantum
simulations. We use complex numbers for rates and there is a
corresponding quantum version of Gillespie that turns process
expressions decorated with complex valued channel rates into
simulations. Under this interpretation the $\mathsf{COMM}$ rule becomes
a version of the Born rule.

However, there are different approaches to consider which we will
return to after providing the details of the more standard approaches.

\subsection{Stochastic rho}

In the stochastic version of rho, we assume we have a function $r$ which assigns a positive rate to a given channel.  Hence, $r: @\text{Proc} \rightarrow \mathbb{R}^+$, where $@\text{Proc}$ is the name-space, and $\mathbb{R}^+$ is the set of non-negative reals.  For convenience, we write $(c,r')$ for the name of channel $c$ with rate $r'=r(c)$, although we note that $r'$ is fully determined by $c$.  The sets of names and processes for stochastic rho are thus identical to those of rho, while the name equivalence relation is modified such that for names $c$ and $d$, $c\equiv_N d$ iff $r(c)=r(d)$.

We consider the stochastic reduction of a rho process $P$.  In rho, $P$ may have multiple possible reductions, $P'_i$ for $i=1...N$ (where $N$ is finite), corresponding to different possible applications of the $\mathsf{COMM}$-rule.  Each of these will be associated with a channel, and we may write these channels and their associated rates as $(c_1,r_1),...,(c_N,r_N)$.  The rate coefficients may be regarded as reaction propensities in a chemical system.  Hence, following the Gillespie algorithm, to reduce $P$, assuming $\sum_n r_n > 0$ we first sample a time $\tau$ according to the exponential distribution:

\begin{eqnarray}
\tau \sim \left(\sum_n r_n\right) \exp\left(-\tau\sum_n r_n\right),
\end{eqnarray}

\noindent and then choose a reduction $Q$ according to the distribution:

\begin{eqnarray}\label{eq:prob_update}
p(Q^*\equiv P^*_n) \propto \sum_{\{m|P^*_m\equiv P^*_n\}} r_m.
\end{eqnarray}

\noindent For the case $\sum_n r_n = 0$, $P$ reduces in an infinite time, and hence may be regarded as a terminal state.

\subsection{Quantum rho}

In quantum rho, we follow the same approach as the stochastic case, but instead of using real channel rates, instead we allow complex rates, hence we have $r:@\text{Proc} \rightarrow \mathbb{C}$.  Now, in addition to the sort of rho processes, we introduce an additional sort of superimposed rho processes, which we will write as $\text{Proc}^*$:

\begin{eqnarray}
\phi, \psi ::= \alpha_1 \ket{P_1} \; + \; ... \; + \; \alpha_i \ket{P_i} \; + \; ...  + \alpha_M \ket{P_M},
\end{eqnarray}

\noindent where $M>0$, $\alpha_1 ... \alpha_M \in \mathbb{C}$, $\sum_i |\alpha_1|^2 = 1$ and we require that, for all $i\neq j$, $P_i \not\equiv P_j$.  Since each channel has a complex rate, in certain cases a rho process $P$ may be associated with a coherent superimposed successor state over its possible rho reductions, $P'_1 \; ... \; P'_I$.  We thus define a further function $s:\text{Proc} \times \mathbb{N} \rightarrow \text{Proc}^*\cup\{\text{null}\}$:

\begin{eqnarray}\label{eq:succ_state}
s(P,M) = 
\begin{cases}
\sum_i r_i \ket{P'_i}/\sqrt{r_0} & \text{if} \; r_0 = \sum_i |r_i|^2 > 0 \; \wedge\forall i\neq j. P'_i \equiv P'_j \Rightarrow r^2_i r^2_j = 0 \\
\ket{P'_{i^*}}
 & \text{if} \; r_0 = \sum_i |r_i|^2 > 0 \; \wedge M=1, \; p(i^*=i)=|r_i|^2/r_0 \\
\text{null} & \text{otherwise,}
\end{cases}\nonumber\\
\end{eqnarray}

\noindent We note that $s(P,M)$ is deterministic except when $M=1$ and $\exists i\neq j. P'_i \equiv P'_j \wedge r^2_i>0 \wedge r^2_j > 0$.

In quantum rho, we wish to define an operational semantics over the enhanced processes, $\phi,\psi\in \text{Proc}^*$.  To do so, we consider the matrix $\tilde{U}_{\phi}$, letting $\phi=\sum_{i=1...M} \alpha_i \ket{P_i}$, defined as:

\begin{eqnarray}
\tilde{U}_{\phi} = \left[\sqrt{r_0^{(1)}}\text{vec}(s(P_1,M)), ... , \sqrt{r_0^{(M)}}\text{vec}(s(P_M),M)\right],
\end{eqnarray}

\noindent where $\text{vec}(s(P_i),M)$ denotes a vertical vectorization of the state $s(P_i,M)$ when $s(P_i,M) \in \text{Proc}^*$ and the vector $\mathbf{0}$, if $s(P_i,M)=\text{null}$, while $[,]$ denotes horizontal concatenation of column vectors.  Here, we note that $\text{vec}:\text{Proc}^*\cup\{\text{null}\} \rightarrow \mathbb{C}^{|S|}$, where $S=\{Q|\exists i,j, \; Q\equiv P'_{i,j} \vee \exists i, \; Q\equiv P_i\}$, where $P'_{i,j}$ is the $j$'th reduction in rho of $P_i$.  Since all rho terms have finitely many reductions, $|S|<\infty$, and the state $s(P_i,M)$ is vectorized by imposing an ordering on $S$ (for instance, lexicographic) so that $S=\{Q_1,...,Q_{|S|}\}$), and letting:

\begin{eqnarray}
\text{vec}_k(s(P_i,M)) = 
\begin{cases}
\braket{Q_k}{s(P_i,M)} & \text{if} \; s(P_i,M) \neq \text{null} \\
0 & \text{otherwise,}
\end{cases}
\end{eqnarray}

\noindent where $\text{vec}_k(.)$ is the $k$'th entry of $\text{vec}(.)$.

We now consider the matrix $\tilde{U}_\phi$.  If we can find a unitary matrix $U_\phi$ over the basis $\ket{Q_1}...\ket{Q_{|S|}}$ and real value $\alpha_0\in \mathbb{R}^+$ such that:

\begin{eqnarray}\label{eq:unitary}
\tilde{U}_\phi = \alpha_0 U_\phi[.,\pi(1...M)],
\end{eqnarray}

\noindent where $\pi$ is a permutation over columns, then we may update $\phi$ according to unitary Schr\"odinger dynamics.  We note that this implies $\alpha^2_0=r_0^{(1)}=...=r_0^{(M)}$.  We call this a `unitary reduction', and in this case, we sample a time according to:

\begin{eqnarray}
\tau \sim \alpha^2_0 \exp\left(-\tau\alpha^2_0\right),
\end{eqnarray}

\noindent and update $\phi$ to $\phi'=\sum_j \beta_j \ket{Q_j}$, where $\mathbf{\beta}=U_{\phi} [\alpha_{\pi^{-1}(1)} ; ... ; \alpha_{\pi^{-1}(|S|)}]$, writing $[.;.]$ for vertical concatenation, and letting $\alpha_i = 0$ for $i>M$.

On the other hand, if $\tilde{U}$ cannot be expressed as in Eq. \ref{eq:unitary}, we may apply the Born rule to reduce $\phi$, assuming $\phi$ is a superposition over at least two rho expressions (hence $M>1$), which we call `measurement-based reduction'.  In this case, we let $\tau=0$ (i.e. the reduction is instantaneous), and sample $\phi'$ according to:

\begin{eqnarray}
p(\phi'=\ket{P_i}) = |\alpha_i|^2.
\end{eqnarray}

We note that, if $M=1$, so that $\phi=\ket{P}$, and $\phi'=s(P)\in \text{Proc}^*$, but $\tilde{U}_{\phi'}$ cannot be expressed as in Eq. \ref{eq:unitary}, the quantum rho reduction semantics above simplifies to the probabilistic reduction semantics of stochastic rho, replacing the complex channel rates $r_i$ with $|r_i|^2$.  Hence, a reduction time $\tau$ is sampled according to $\tau \sim r_0\exp(-\tau r_0)$, where $r_0=\sum_i |r_i|^2$, and $\phi'=\sum_i r_i/\sqrt{r_0} \ket{P'_i}$, which reduces instantaneously to $\phi''$ using the Born rule, meaning that $p(\phi''=\ket{P_i'})=\alpha_i^2=|r_i|^2/r_0$, as in Eq. \ref{eq:prob_update}. Similarly, if $M=1$ and there exists two possible reductions of $P$ in rho such that $P'_i\equiv P'_j$ with non-zero complex channel weights, the the quantum rho reduction semantics above simplifies to the probabilistic reduction semantics by the probabilistic selection of $i^*$ in Eq. \ref{eq:succ_state}.  Finally, we note that all terminal states in quantum rho are of the form $\phi=\ket{P}$ (since if $M>1$, the Born rule may be applied), and for which $P$ either has no reductions in rho, or has reductions which are all associated with channels having zero complex weight. 

\vspace{0.5cm}
\noindent \textbf{A universal quantum circuit model in quantum rho.}. We briefly show that quantum rho can be used to express arbitrary quantum computations, by exhibiting a universal quantum gate set.  For this purpose, we use the rho encoding $\mathsf{T} ::= t\mathsf{!}(0)$ and $\mathsf{F} ::= f\mathsf{!}(0)$, for the Booleans True and False respectively, where $t$ and $f$ are two distinguished channels.  We can then encode conditional statements in rho as:

\begin{eqnarray}
(\mathsf{if} \; y \; P \; Q) ::= \mathsf{*}y \; \mathsf{|} \; (\mathsf{for}(z\leftarrow t)P) + (\mathsf{for}(z\leftarrow f)Q).
\end{eqnarray}

\noindent For a classical circuit of size $N$ bits, we assume that we can encode process and name tuples in rho, and hence we have encodings for the $2^N$ processes $(\mathsf{F},...,\mathsf{F}),...,(\mathsf{F},...,\mathsf{T}),...,(\mathsf{T},...,\mathsf{T})$, each a tuple of processes of length $N$.  Then, we can write a classical circuit of length $L$ in rho with input provided on channel $x_0$ and output generated on channel $x_L$, as the term:

\begin{eqnarray}
\mathsf{for}(y\leftarrow x_0)(x_1 \mathsf{!}(G_1)) \mathsf{|} \mathsf{for}(y\leftarrow x_1)(x_2 \mathsf{!}(G_2)) \mathsf{|} \; ... \; \mathsf{|} \mathsf{for}(y\leftarrow x_{L-1})(x_L \mathsf{!}(G_L)),
\end{eqnarray}

\noindent where $G_l$ is the rho expression for the $l$’th gate in the circuit.  Below, we provide rho expressions for classical $\mathsf{NOT}(n)$, $\mathsf{AND}(n,m)$ and $\mathsf{OR}(n,m)$ gates, where $n,m\in\{1…N\}$ denote the indices of the bits to which the gates are applied:

\begin{eqnarray}
\mathsf{NOT}(n) &::=& (\mathsf{if} \; y_n \; \mathsf{Replace}(y,n,\mathsf{F}) \; \mathsf{Replace}(y,n,\mathsf{T})) \nonumber \\
\mathsf{AND}(m,n) &::=& (\mathsf{if} \; y_m \; (\mathsf{if} \; y_n \; y \; \mathsf{Replace}(y,m,\mathsf{F})) \; \mathsf{Replace}(y,n,\mathsf{F})) \nonumber \\
\mathsf{OR}(m,n) &::=& (\mathsf{if} \; y_m \; \mathsf{Replace}(y,n,\mathsf{T}) \; (\mathsf{if} \; y_n \; \mathsf{Replace}(y,m,\mathsf{T})) \; y),
\end{eqnarray}

\noindent where $y_n$ denotes the projection onto the $n$’th entry of the tuple $y$, and $\mathsf{Replace}(y,n,P)$ denotes the process formed by replacing the $n$’th entry of the tuple $y$ with $@P$.

To define a universal quantum gate model, we use the same circuit model, where the terms above are now interpreted as terms in quantum rho, where all channels are assumed to have rate $1$, unless explicitly stated.  One possible set of universal quantum gates consists of the Clifford gates ($\mathsf{HADAMARD}$, $\mathsf{PHASE}$ and $\mathsf{CNOT}$) along with the $\mathsf{TOFFOLI}$ gate. Below, we give quantum rho expressions for this gate set:

\begin{eqnarray}
\mathsf{HADAMARD}(n) &::=& (\mathsf{if} \; y_n \; \mathsf{Replace}(y,n,\mathsf{P}^+) \; \mathsf{Replace}(y,n,\mathsf{P}^-)) \nonumber \\
\mathsf{PHASE}(n) &::=& (\mathsf{if} \; y_n \; y \; \mathsf{Replace}(y,n,\mathsf{P}^i(\mathsf{F}))) \nonumber \\
\mathsf{CNOT}(m,n) &::=& (\mathsf{if} \; y_m \; \mathsf{NOT}(n) \; y) \nonumber \\
\mathsf{TOFFOLI}(m_1,m_2,n) &::=& (\mathsf{if} \; y_{m_1} \; (\mathsf{if} \; y_{m_2} \; \mathsf{NOT}(n) \; y) \; y),
\end{eqnarray}

\noindent where the processes $\mathsf{P}^+$, $\mathsf{P}^-$ and $\mathsf{P}^i$ are defined as:

\begin{eqnarray}
\mathsf{P}^+ &::=& ((g_3 \mathsf{!} (\mathsf{T})) \mathsf{|} (\mathsf{for}(y'\leftarrow g_3)\mathsf{T} + \mathsf{for}(y'\leftarrow g_3)\mathsf{F}) \nonumber \\
\mathsf{P}^- &::=& ((g_3 \mathsf{!} (\mathsf{T}) + g_4 \mathsf{!} (\mathsf{T})) \mathsf{|} (\mathsf{for}(y'\leftarrow g_3)\mathsf{T} + \mathsf{for}(y'\leftarrow g_4)\mathsf{F}) \nonumber \\
\mathsf{P}^i(z) &::=& ((g_5 \mathsf{!} (z)) \mathsf{|} (\mathsf{for}(z'\leftarrow g_5)z'),
\end{eqnarray}

\noindent and the channels $g_3$, $g_4$, and $g_5$ have the rates $(1/\sqrt(2))$, $(-1/\sqrt(2))$, and $i$ respectively. 

To allow a quantum state to be provided as input to a quantum circuit, the circuit may always be extended by prefixing a quantum circuit which starts from the $\ket{\mathsf{F},...,\mathsf{F})}$  state and prepares the required input state.  Hence, we may run a circuit in quantum rho by initializing the system to the state $\ket{(x_0 \mathsf{!}(\mathsf{F},…,\mathsf{F})\mathsf{|}C_0 \mathsf{|}C)}$, where  $C_0$ is the quantum rho encoding of the prefix circuit, and $C$ is the desired core circuit encoding.  Reduction of the circuit will result in an expression of the form $\ket{(x_L \mathsf{!}(P))}$ where $P$ is a tuple encoding $N$ Boolean values, representing a measurement of the circuit’s output in the computational basis.  To prevent a measurement occurring in order to apply further processing to the circuit’s final quantum state, for instance using an additional circuit $D$ (which we assume to share no common channel names with $C$), circuits may simply be concatenated compositionally, producing the term $\ket{(x_0 \mathsf{!}(\mathsf{F},…,\mathsf{F})|C_0 |C|D)}$.

\subsection{Stochasticity and annihilation}

TBD

\subsection{Quantum mechanics directly encoded}

TBD
