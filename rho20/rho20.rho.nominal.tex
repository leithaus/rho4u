\section{Name play}
One of the interesting observations about names in the
${\pi}$-calculus is that it is only through the use of the
$\mathsf{new}$ operator \emph{under replication},
i.e. $\mathsf{!}(\mathsf{new}\; x)P$, that we get what amounts to
infinite storage, i.e. the ability to model phenomena like the tape in
a Turing machine. This leads us to the hypothesis that controlling
which kinds of processes may be quoted to produce names in the rho
calculus allows us to control the computational expressive power of
the calculus.

\subsection{Regular languages and rho}
TBD

\subsection{Context-free languages and rho}
TBD

\subsection{Infinite names}
TBD
