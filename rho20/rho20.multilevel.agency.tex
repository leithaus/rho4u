\section{Multilevel Agency}
The rho calculus provides a number of important perspectives on
multilevel agency. In this section we focus on three of them that can
be reduced to calculi that can be coded up and executed on modern
computer hardware.

\subsection{Annihilation}
Another important variation has to do with the rewrite rules. There is
a nested recursion version of the $\mathsf{COMM}$-rule that aligns
with intuitions about the recursive nature of behavior in
compositionally defined agents. While for your author the maths make
it clearer than the English it is worth setting out some of the
motivations for this variation of the dynamics of the rho
calculus.

The reductionist view of science is that phenomena like medicinal
cures rest on the phenomena of biology. That is, we find explanation
of the mechanism underlying the medical procedure in the biological
phenomena the procedure interacts with in some fashion. Meanwhile, the
phenomena of biology rests, in turn, on the phenomena of chemistry;
and chemistry rests on (nuclear) physics. By way of an example, in the
reductionist narrative the interaction of two blood cells in the veins
of the body of a person is ultimately explained in terms of the
interactions of various chemical compounds, which are in turn
explained in terms of the interaction of various atoms and their
subatomic components, such as electrons, etc.

While this is accepted philosophy and pedagogy, it is quite difficult
to reduce this account to actual computations, except in very limited
or schematic and abstract fashion. For example, getting differential
equations to work effectively across such scales (from cells to
quarks) is simply not practical. The reductionist picture is
comforting and to some extent supported by evidence, but not reducible
to practice except in extremely limited cases.

The variation of the rho calculus dynamics presented here was
developed to explore the power of compositionality to capture this
kind of tower of reduction. Essentially, we develop of notion of
process interaction that in turn relies on the interaction of ``lower
level'' processes, and so on all the way down to a real bottom. The
notion gives a precise definition of what it means to be ``lower
level'' and how that relates to computational dynamics. In some sense,
this is the crudest of pictures of multilevel agency, and yet given
the efficacy of the process calculi to faithfully represent chemical,
biochemical, and biological phenomena, there is some hope that it
might not just be a pleasant abstraction.

First, we define what it means for two processes to
annihilate each other.

\begin{definition}
  Annihilation: Processes $P$ and $Q$ are said to annihilate one another, written $P \bot Q$, just when $\forall R. P \mathsf{|} Q \rightarrow^{*} R \Rightarrow R \rightarrow^{*} \pzero$.
\end{definition}

Thus, when $P \bot Q$, all rewrites out of $P \mathsf{|} Q$ eventually
lead to $\pzero$. Evidently, $P \bot Q \iff Q \bot P$, and $\pzero
\bot \pzero$. Naturally, we can extend annihilation to names: $x \bot
y \iff \procn{x} \bot \procn{y}$.

Annihilation affords a new version of the $\mathsf{COMM}$-rule:

\begin{mathpar}
  \inferrule* [lab=COMM] {x_{t} \;\bot x_{s}, \;\;\; |\arrvec{y}| = |\arrvec{Q}|} {(R_1 + \mathsf{for}( \arrvec{y} \leftarrow x_{t} )P) \;\mathsf{|}\; (x_{s}!(\arrvec{Q}).P' + R_2)
  \red P\substn{\arrvec{\quotep{Q}}}{\arrvec{y}}\mathsf{|}P'}
\end{mathpar}

All annihilation-based reduction happens in terms of reductions that
happen at a lower degree of quotation, and grounds out in the fact
that $\pzero \bot \pzero$ and thus $\quotep{\pzero} \bot \quotep{\pzero}$.

\begin{example}
  For example, let $P_{1} := \prefix{\quotep{\pzero}}{\quotep{\pzero}}{\pzero} \mathsf{|} \outputp{\quotep{\pzero}}{\pzero}$. Then $P_{1} \red \pzero$ because $\pzero \bot
\pzero$. Suppose now that we set $x_{0}^{-}, x_{0}^{+} := \quotep{\pzero}$. Then, we can write $P_{1}$ as
$\prefix{x_{0}^{-}}{x_{0}^{-}}{\pzero} \mathsf{|} \outputp{x_{0}^{+}}{\pzero}$. Now, set

\begin{mathpar}
  x_{1}^{-}:= \quotep{(\prefix{x_{0}^{-}}{x_{0}^{-}}{\pzero})}
  \and
  x_{1}^{+} := \quotep{(\outputp{x_{0}^{+}}{\pzero})}
\end{mathpar}

Then define 
$P_{2} := \prefix{x_{1}^{-}}{x_{1}^{-}}{\pzero} \mathsf{|} \outputp{x_{1}^{+}}{\pzero}$. Then $P_{2} \red \pzero$ because $x_{1}^{-} \bot x_{1}^{+}$,
and hence $\prefix{x_{1}^{-}}{x_{1}^{-}}{\pzero} \bot \outputp{x_{1}^{+}}{\pzero}$.

More generally, set

\begin{mathpar}
  x_{i}^{-} := \quotep{(\prefix{x_{i-1}^{-}}{x_{i-1}^{-}}{\pzero})}
  \and
  x_{i}^{+} := \quotep{\outputp{x_{i-1}^{+}}{\pzero}} \\
  \and P_{i} := \prefix{\quotep{x_{i-1}^{-}}}{x_{i-1}^{-}}{\pzero} \mathsf{|} \outputp{\quotep{x_{i-1}^{+}}}{\pzero}
\end{mathpar}

Then $P_{i} \red \pzero$ and hence $x_{i}^{-} \bot x_{i}^{+}$.
\end{example}
This allows
for a measure of reduction complexity.

\subsection{Procedural reflection}

The perspective on multilevel agency embodied in this conservative
extension to the calculus is focused on a process being able to ``see
into the future of another process''. That is, for a given process $P$
to probe how another process, say $Q$, might evolve. This ability to
``imagine'' the behavior of another becomes introspection when a
process turns this imaginative capacity onto itself. This
introspection creates a kind of reflective tower, ala
$\mathsf{3-Lisp}$ \cite{DBLP:conf/popl/Smith84} or $\mathsf{Brown}$
\cite{DBLP:journals/lisp/WandF88}. Different floors or levels in this
tower correspond to different levels of introspection. That is, level
$n$ is one level of introspection deeper than level $n-1$. It is in
this sense that we view procedural reflection as providing a kind of
multilevel agency.

Note that the fact that the $\pi$-calculus can be encoded into the rho
calculus means that rho is Turing complete, or a model of universal
computation. The reason we say that the procedurally reflective rho
calculus is a conservative extension is because it doesn't add any
fundamentally new expressive power in the way that adding an oracle
that answers certain halting questions would. As such, we can write a
meta-circular interpreter for the rho calculus in the rho
calculus. Having done so, we could make reflection on the evolution of
a process available to the process undergoing evolution. In some
sense, this is the rho calculus equivalent of functional programming
phenomena like $\mathsf{call/cc}$, or more generally, delimited
continuations \cite{DBLP:journals/jfp/DybvigJS07}.

To achieve this ability for a process, say $P$ to look into its future
we introduce an additional syntactic category, $x\mathsf{?}P$. Then, if
$P$ evolves in a single step to $P'$, then $x\mathsf{?}P$ evolves to
$x\mathsf{!}\mathsf{(}P'\mathsf{)}$. That is, a next step in P's
evolution is made available at $x$. Rendering this idea in symbols, is
simply adding the new syntactic category, extending the definition of
free names in the obvious way, and one additional rewrite rule.

\begin{mathpar}
\inferrule* [lab=process] {} {P, Q \bc \pzero \;\bm\; \mathsf{for}(
  y \leftarrow x )P \;\bm\; x\mathsf{!}(Q) \;\bm\;
  \mathsf{*}x \;\bm\; x\mathsf{?}P \;\bm\; P\mathsf{|}Q } \and \inferrule* [lab=name] {}
            {x, y \bc \mathsf{@}P }
\end{mathpar}

\begin{mathpar}
  \inferrule* [lab=equiv] {} {P\mathsf{|}\pzero \equiv P \;\; P\mathsf{|}Q \equiv Q\mathsf{|}P \;\; P\mathsf{|}(Q\mathsf{|}R) \equiv (P\mathsf{|}Q)\mathsf{R}} \\
  \and
  \inferrule* [lab=alpha] {} {\mathsf{for}(y \leftarrow x )P \equiv \mathsf{for}(z \leftarrow x )(P\{z/y\} \; \mathsf{if} z \notin \mathsf{FN}(P)}
\end{mathpar}

\begin{mathpar}
  \inferrule* [lab=COMM] {} {\mathsf{for}( y \leftarrow x )P \;\mathsf{|}\; x!(\arrvec{Q})
    \red P\substn{\quotep{Q}}{y}} \\
  \and
  \inferrule* [lab=PAR]{P \red P'}{P\mathsf{|}Q \red P'\mathsf{|}Q}
  \and
  \inferrule* [lab=EQUIV]{{P \;\scong\; P'} \andalso {P' \red Q'} \andalso {Q' \;\scong\; Q}}{P \red Q} \\
  \and
  \inferrule* [lab=REFL] {P \red P'} {x\mathsf{?}P \red x\mathsf{!}\mathsf{(}P'\mathsf{)}} \\
\end{mathpar}

What makes this version of procedural reflection distinct and
interesting is that it fits well within a concurrency setting. It is
common folklore that $\mathsf{call/cc}$-like continutations and
concurrency are not the best bedfellows. Essentially,
$\mathsf{call/cc}$ provides the ability to revert \emph{global}
state. But, in some very real sense there is no global state, or the
state shared amongst a collection of agents is partitioned in such a
way that not every agent has access and there is no shared way to roll
back the clock for all agents. Once one agent has fired a missile,
there is no time machine that rolls everyone's state back and puts the
missile back in the silo. This fact about the partitioning of state is
closely related to the famous arrow of time problem in physics, but we
won't delve into it here \cite{enwiki:1234510214}. Instead, we note
that the form procedural reflection takes in the rho calculus is
perfectly amenable to multi-agent settings in which state is
partitioned and time cannot be globally rolled back.

\subsection{Programmable contexts}
\subsubsection{Process grammar}\label{subsub:process_grammar}

\begin{mathpar}
  \inferrule* [lab=process] {} {P, Q \bc \pzero \;\bm\; \mathsf{U}(x) \;\bm\; \mathsf{for}(y \leftarrow x )P \;\bm\; x\mathsf{!}(Q) \;\bm\;
  P\mathsf{|}Q \;\bm\; \mathsf{*}x \;\bm\; \mathsf{COMM}(K) }
  \and
  \inferrule* [lab=name] {} {x, y \bc \mathsf{@}\langle K, P\rangle }
  \and
  \inferrule* [lab=context] {} {K \bc \bigbox \;\bm\;  \mathsf{for}(y \leftarrow x )K \;\bm\; x\mathsf{!}(K) \;\bm\; P\mathsf{|}K}
\end{mathpar}

\begin{definition}
  \emph{Free and bound names} The calculation of the free names of a
  process, $P$, denoted $\freenames{P}$ is given recursively by
  
  \begin{mathpar}
    \freenames{\pzero} = \emptyset
    \and
    \freenames{\mathsf{U}(x)} = \{ x \}
    \and
    \freenames{\mathsf{for}(y \leftarrow x)P} = \{ x \} \cup \freenames{P}\setminus\{y\}
    \and
    \freenames{x!(P)} = \{ x \} \cup \freenames{P}
    \and
    \freenames{P|Q} = \freenames{P} \cup \freenames{Q}
    \and
    \freenames{\mathsf{*}{x}} = \{ x \}
    \and
    \freenames{\mathsf{COMM}(K)} = \freenames{K}
    \and 
    \freenames{\bigbox} = \emptyset
    \and
    \freenames{\mathsf{for}(y \leftarrow x)K} = \{ x \} \cup \freenames{K}\setminus\{y\}
    \and
    \freenames{x!(K)} = \{ x \} \cup \freenames{K}
    \and
    \freenames{P|K} = \freenames{P} \cup \freenames{K}
  \end{mathpar}
  
  An occurrence of $x$ in a process $P$ is \textit{bound} if it is not
  free. The set of names occurring in a process (bound or free) is
  denoted by $\names{P}$.
\end{definition}

\subsection{Operational semantics}

Finally, we introduce the computational dynamics. What marks these
algebras as distinct from other more traditionally studied algebraic
structures, e.g. vector spaces or polynomial rings, is the manner in
which dynamics is captured. In traditional structures, dynamics is typically
expressed through morphisms between such structures, as in linear maps
between vector spaces or morphisms between rings. In algebras
associated with the semantics of computation, the dynamics is
expressed as part of the algebraic structure itself, through a
reduction reduction relation typically denoted by $\red$. Below, we
give a recursive presentation of this relation for the calculus used
in the encoding.

\begin{mathpar}
  \inferrule* [lab=Catalyze] {} {\mathsf{U}(x) \;\mathsf{|}\; \dropn{\quotep{\langle K,Q \rangle}} \red \mathsf{COMM}(K) \;\mathsf{|}\; x\mathsf{!}(Q)} \\
  \and
  \inferrule* [lab=Comm] {x_{t} \;\nameeq\; x_{s}} {\mathsf{COMM}(K) \;\mathsf{|}\; \mathsf{for}( y \leftarrow x_{t} )P \;\mathsf{|}\; x_{s}!(Q)
    \red P\substn{\quotep{\langle K,Q \rangle}}{y}} \\
  \and
  \inferrule* [lab=Par]{P \red P'}{P\mathsf{|}Q \red P'\mathsf{|}Q} \\
  \and
  \inferrule* [lab=Equiv]{{P \;\scong\; P'} \andalso {P' \red Q'} \andalso {Q' \;\scong\; Q}}{P \red Q}
\end{mathpar}

We write $P\red$ if $\exists Q $ such that $ P \red Q$ and $P\not\red$, otherwise.

\subsection{ Movement in space: an example }
In the following example let $x = \spacen{\bigbox}{\pzero}$ and $y = \spacen{K}{Q}$ for some $K$ and $Q$.
\begin{eqnarray*}
  & \binpar{\binpar{\prefix{x}{y}{\dropn{y}}}{\outputp{x}{P}}}{\mathsf{COMM}(K_{1})} & \\
  \red & & \\
  & \dropn{y}\substn{\spacen{K_{1}}{P}}{y} & \\
  = & & \\
  & K_{1}[P] &
\end{eqnarray*}

Now, if $K_{1}$ is also of the form $\binpar{\binpar{\binpar{\prefix{x'}{y'}{\dropn{y'}}}{\outputp{x'}{\bigbox}}}{\mathsf{COMM}(K_{2})}}{R}$, then $P$ will move to the location $\binpar{K_{2}[P]}{R}$. That is, $K_{1}[P] \red \binpar{K_{2}[P]}{R}$. And if $K_{2}$ is likewise of the form $\binpar{\binpar{\binpar{\prefix{x''}{y''}{\dropn{y''}}}{\outputp{x''}{\bigbox}}}{\mathsf{COMM}(K_{3})}}{R'}$, then $P$ will move to the location $\binpar{K_{3}[P]}{\binpar{R}{R'}}$.

Thus, we have a means to describe movement of a process from location to location.
