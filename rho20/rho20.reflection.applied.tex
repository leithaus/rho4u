\section{Applied reflection: Reflective set theory and reflective graphs}
In this section we illustrate, albeit briefly, that reflection has a
much wider utility than just computation. We describe reflective
versions of set theory and graph theory and illustrate via some core
examples how reflection accounts for an interesting range of phenomena.

\subsection{Reflective set theory}

This presentation will be just a sketch. A more rigorous formal
account is most likely built using Awodey's algebraic set theory.

\begin{mathpar}
  \inferrule* [lab=red-set] {} {\textcolor{red}{\mathsf{S}}[A] \bc \textcolor{red}{\bold{\mathsf{\{}}} (A \bm \textcolor{red}{\mathsf{S}}[A])^{*} \textcolor{red}{\bold{\mathsf{\}}}}}
  \and
  \inferrule* [lab=black-set] {} {\mathsf{S}[A] \bc \bold{\mathsf{\{}} (A \bm \mathsf{S}[A])^{*} \bold{\mathsf{\}}}} \\
  \and
  \inferrule* [lab=rset-player] {} {\mathsf{RSet}_{player} = \textcolor{red}{\mathsf{S}}[\mathsf{S}[\mathsf{RSet}_{player}]]}
  \and
  \inferrule* [lab=rset-opponent] {} {\mathsf{RSet}_{opponent} = \mathsf{S}[\textcolor{red}{\mathsf{S}}[\mathsf{RSet}_{opponent}]]}
  \and
  \inferrule* [lab=rset] {} {\mathsf{RSet} = \mathsf{RSet}_{player} \oplus \mathsf{RSet}_{opponent}}
\end{mathpar}

The intuition is that we have two distinct \emph{copies} of set
theory, which we will denote for purposes of discussion, red set
theory and black set theory. Both of these theories are considered
parametric in a theory of atoms. Thus, these theories are instances of
a Fraenkl-Mostowski set theory ($\mathsf{FM}$ set theory). But, now we
make the atoms of the red set theory be black sets, and the atoms of
the black set theory be red sets.

Again, computer science provides a unique perspective on these age old
notions. The idea of copies of a theory comes about when we reify meta
theory as theory, something implicit in Awodey's work. Computer
science also provides a very practical means for realizing it: we have
different constructors (red braces versus black braces); and different
accessor (a red element-of operation versus a black element of
operation). Further, computer science is quite comfortable with these
kinds of mutually recursive structures. The mutual recursion bottoms
out in the red empty set and the black empty set.

What motivates such a strange construction? It turns out that
$\mathsf{FM}$ set theory has been successfully applied to reasoning
about nominal phenomena, such as binders in the $\pi$-calculus. But,
once again, one is left with a dissatisfying theory of atoms without
structure. This theory explains where atoms come from and how they can
have exactly the same structure as sets. In other words, it reconciles
$\mathsf{FM}$ set theory with $\mathsf{ZF}$ set theory.

\subsection{Reflective graphs}

TBD
